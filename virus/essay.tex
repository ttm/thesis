\documentclass[a4paper]{article}

% \usepackage[portuguese]{babel}
\usepackage[utf8]{inputenc}
\usepackage{amsmath}
\usepackage{graphicx}
\usepackage[colorinlistoftodos]{todonotes}
\providecommand{\keywords}[1]{\textbf{\textit{Index terms---}} #1}

\usepackage{tocloft}
\addtocontents{toc}{\cftpagenumbersoff{section}}

\usepackage{listings}
\usepackage{fancyhdr}
\usepackage{hyperref}
\hypersetup{
        colorlinks,
        linkcolor={red!50!black},
        citecolor={blue!50!black},
        urlcolor={blue!80!black}
}

% todo:
% make index and links beautiful, maybe pnud5
% put bibliography
% include one or two images?
% review all
% make listings
% make zip with images of the cow
% see README.md and thesis again and dissertation
% send to Cristina, and gang of avatars
% send to VIRUS journal:
%   send in Vim printed PDF
%   send in HTML rendered and then PDF
%   send in Latex

\title{
Memory and self-dismantling in social experiments:\\
anthropological physics and technoxamanic heys (ebós)
}
\author{Renato Fabbri}

\date{\today, version 0.1-alfa}

\usepackage{etoolbox}

\makeatletter
\pretocmd{\chapter}{\addtocontents{toc}{\protect\addvspace{15\p@}}}{}{}
\pretocmd{\section}{\addtocontents{toc}{\protect\vspace{-3mm}}}{}{}
\makeatother


\begin{document}
\maketitle



\begin{abstract}
Both mythological and hacker histories have recognized roles for
self-dismantling: it protects the messenger, allows the detachment of the self
from self image, is an artistic technique, etc.  Brazil has a pronounced role
in this context, for it yields religious freedom since the colonization and
beforehand, and holds a renowned and visceral hacker behavior: the kludge
culture (aka. 'cultura da gambiarra').  This article exposes this legacy by two
means: the description of social experiments made by many participants at once,
memorials of images, videos, texts, music, webpages, groups,
avatars/nicks/pseudonyms, presentations, etc.  This text is itself an
experiment, and will be fed back to the community for comments before
publishing, as usual with any \emph{anthropological physics} experiment.  The
materials herein are no secret, and are usually not unpublished, although most
of it have not been bind to a DOI or an ISBN/ISSN.  Further directions are given as
seminal ideas because next steps will be given by the community upon diverse
interests and context stonework.
\end{abstract}

\keywords{
anthropological physics, technoxamanism, memorial, complex networks,
data mining
}

\tableofcontents

\section{What? or motivation}
The main motivation of what is described here was and is to enable
members/participants to take action in their networks by means of scientific
knowledge.  Such networks are complex and social networks, topological
structures that are embedded in, and embed, other complex systems.  The ethic
issues that arise when experimenting with other humans were ameliorated with
the rise of the \emph{anthropological physics}~\cite{anPhy,anPhy2,thesis}:
researchers or activists should keep the processes as open as possible (texts,
software, data, processes, outcomes, people involved, etc) while studying and
experimenting in their own networks; a trace inherited from ethnography and
similar to the technique/strategy of writing diaries.

\section{How? or social, technoxamanic experiments of Collection and Diffusion of information}
Many experiments were carried out by diverse human agents, either directly or
through a second/fake/pseudonym/avatar/nick profile, i.e. people, for various
reasons, made conscious efforts in order to interact with their networks to
achieve specific goals or inspect the outcome.  Two examples are very efficient
in exposing the procedures and potentials: one that is continuous within few
months, one that is ephemeral and occurs in only a few hours or less.  In such
a diversity-rich setting, these experimental procedures were called
\emph{technoxamanic} experiments (or 'yeys' for Brazilian Portuguese '{\bf
ebós}').

\subsection{the Cow of the End of the World (continuous experiment):
progressive network activsdation from Peripherals to Hubs for a crowdfunding}\label{sec:colDif}

This is maybe the most powerful mechanism by which we performed collection and diffusion of information.
The results were very effective in spreading information about social networks,
in gathering knowledge from diverse parties and in modifying the social structures
in which I participate.
Most concretely, academics came to São Carlos for formal meetings,
new collaborations were established (such as the UNDP consultating described in Section~\ref{sec:undp}),
money was obtained (various contributors transferred a total of about 3000 Brazilian reais)
and my Facebook network increased about 50\% with individuals interested in the research.
The process consisted in:
\begin{enumerate}
	\item Downloading my Facebook friendship network. This was done by means of the Netvizz software,
		which is not possible nowadays and requires scrapping of Facebook pages because of new usage terms.
	\item Sorting my friends from the less connected to the more connected, i.e. from my friends that have less friends in common with me to the ones that have more friends in common; i.e. from periphery to hubs.
	\item Sending private messages for each of my friends, in such order.
		The messages were derived from a template I conceived in which I exposed the research and the information diffusion process.
	\item Making steps 1-3 for three times.
\end{enumerate}
In each cycle of steps 1-3, my friendship network grew about 15\% and there were typical reactions in each cycle.
In the first cycle, my Facebook contacts reacted with estrangement and replies such as ``what are these network structures?'',
``what are you doing? I can't understand!'', ``I never though of such a thing as these networks''.
In the second cycle, they replied with interest and support.
In the third cycle, they engaged in establishing collaborations with visits, in the elaboration of documents and technologies and
in co-working proposals.

The data related to performing these three cycles can be organized by downloading my personal data in the Facebook interface.
The experiment was carried in scientific terms and initial hypothesis were confirmed by these results.
Even so, these results are not still confirmed by performing the experiment again,
which poses both a problem and a potential scientific undertake.
Given that the diffusion process was done in Dec/2012-Jan/2013, it was frequently considered by fellow specialists as
having some influence in the civil society mobilization that occurred in Brazil in Mar/2013 and thereafter.
A very simple PDF document was built afterwards for delivering back these results to the networks~\cite{docDif}.

\subsection{Betweenness VS Closeness centralities (ephemeral experiment)}
This was first thought about in meetings with the artist and activist Pedro Paulo Rocha.
The idea was to activate the network not by means of a longstanding process such
as described in the last section, but by an ephemeral endeavor.
There were some artistic performances with this proposal, in which I did
not participate.
Nevertheless, there was one of these instantaneous activation processes that
I have done in conjunction with other specialists which was rather interesting.
In analyzing Facebook ego friendship networks, I fount that the set of $\approx 50$ members with
the greatest betweenness centrality was disjoint with the set of $\approx 50$ members
with the greatest closeness centrality, which is very unexpected.
Therefore I proposed that one should send the same message to both set of friends separately.
The messages were different for each person performing the experiment,
and it was about something they were interested in and wanted to spread and get feedback.
The result was systematic: the set of friends with greatest betweenness always reacted very friendly
with encouraging messages and sharing the original message in their timelines.
The set of friends with greatest closeness always reacted with many leaving the chat group
and with no replies.
We hypothesize that these reactions are because the large betweenness set of friends is more
likely to have control over the information flowing in the corresponding ego network while
the large closeness set of friends is more likely to observe/receive influence by the information.
This experiment was performed by partners related to the consulting reported in Section~\ref{sec:undp}
and other partners involved in making an international technoshamanic festival.

\subsection{massive tagging (semi-ephemeral experiment)}
One very simple process by which we performed collection and diffusion of information
was by tagging many friends in Facebook posts.
Currently, one can tag up to 99 friends in a post
and we did not find any limit for tagging friends in comments.
If one makes abusive use of tagging (too many posts or too many comments)
the Facebook platform sometimes restricts the permissions of that user.
Even so, I have made many posts with up to 99 friends tagged and tagged more
friends in the comments and made experiments such as the ones described in the
last sections and never got restricted.
It seems that the platform has some automated behavior but employees actually
perform the restrictions at least in some cases.
The employees might check the posts, tagging and messages to see if it is
really spam or in anyway abusive.
In~\cite{anExp} are some notes and data of one of these experiments (and a preliminary script for analysis).

Another powerful way by which I many times performed diffusion and collection
of information is by crossposting, i.e. by sending a message to many email lists
at the same time.
I find this very effective but the email list users often report that they understand such practice
as abusive.
Even so, no one has ever sent me a message reporting discomfort with my crossposts.
There was one occasion some years ago when a user replied with a challenge for
arguing why the crosspost what appropriate and then made some good contributions.
I personally perceive that this prejudice against crosspost is one of the main reasons
why email groups are losing users to other communication protocols such as Facebook, Whatsapp, Telegram and Diaspora.

\subsection{video-conferences, etherpads, websites, gadgets, and whatnot}

      
\section{so What and How? Or the outcomes}

\section{by Whom? Or galleries / memorial}
posts with many people marked.
limits on marking people, citing them in comments,
private messages.

\section{What shall we remember? or Memory and Narrative}

\appendix
This appendix holds a memorial through short contextualizations
and listings.
Following the \emph{anthropological physics} guidelines to ameliorate
ethic issues, the materials are as related to the author as possible,
considering the reasons and relevance of the exposition.

\section{avatars/nicks/pseudonyms}
\section{websites}
\section{groups}
labmacambira, metareciclagem, submidialogia, tecnomagias
grupos do fb
canais de irc
\section{videos}
\section{image galleries}
\section{texts}
nuvens cognitivas
tese e dissertacao
\section{musical pieces}
half shape


\end{document}


