\documentclass[a4paper]{article}

% \usepackage[portuguese]{babel}
\usepackage[utf8]{inputenc}
\usepackage{amsmath}
\usepackage{graphicx}
\usepackage[colorinlistoftodos]{todonotes}
\providecommand{\keywords}[1]{\textbf{\textit{Index terms---}} #1}

\usepackage{tocloft}
\addtocontents{toc}{\cftpagenumbersoff{section}}

\usepackage{listings}
\usepackage{fancyhdr}
\usepackage{hyperref}
\hypersetup{
        colorlinks,
        linkcolor={red!50!black},
        citecolor={blue!50!black},
        urlcolor={blue!80!black}
}

% todo:
% make index and links beautiful, maybe pnud5

\title{
Memory and self-dismantling in social experiments:\\
anthropological physics and technoxamanic heys (ebós)
}
\author{Renato Fabbri}

\date{\today, version 0.1-alfa}

\usepackage{etoolbox}

\makeatletter
\pretocmd{\chapter}{\addtocontents{toc}{\protect\addvspace{15\p@}}}{}{}
\pretocmd{\section}{\addtocontents{toc}{\protect\vspace{-3mm}}}{}{}
\makeatother


\begin{document}
\maketitle



\begin{abstract}
Both mythological and hacker histories have recognized roles for
self-dismantling: it protects the messenger, allows the detachment of the self
from self image, is an artistic technique, etc.  Brazil has a pronounced role
in this context, for it yields religious freedom since the colonization and
beforehand, and holds a renowned and visceral hacker behavior: the kludge
culture (aka. 'cultura da gambiarra').  This article exposes this legacy by two
means: the description of social experiments made by many participants at once,
memorials of images, videos, texts, music, webpages, groups,
avatars/nicks/pseudonyms, presentations, etc.  This text is itself an
experiment, and will be fed back to the community for comments before
publishing, as usual with any \emph{anthropological physics} experiment.  The
materials herein are no secret, and are usually not unpublished, although most
of it have not been bind to a DOI or an ISBN/ISSN.  Further directions are given as
seminal ideas because next steps will be given by the community upon diverse
interests and context stonework.
\end{abstract}

\keywords{
anthropological physics, technoxamanism, memorial, complex networks,
data mining
}

\tableofcontents

\section{What? Or motivation}
The main motivation of what is described here was and is to enable
members/participants to take action in their networks by means of scientific
knowledge.  Such networks are complex and social networks, topological
structures that are embedded in, and embed, other complex systems.  The ethic
issues that arise when experimenting with other humans were ameliorated with
the rise of the \emph{anthropological physics}~\cite{anPhy,anPhy2,thesis}:
researchers or activists should keep the processes as open as possible (texts,
software, data, processes, outcomes, people involved, etc) while studying and
experimenting in their own networks; a trace inherited from ethnography and
similar to the technique/strategy of writing diaries.

\section{How? Or social, technoxamanic experiments}
Many experiments were carried out by diverse human agents, either directly or
through a second/fake/pseudonym/avatar/nick profile, i.e. people, for various
reasons, made conscious efforts in order to interact with their networks to
achieve specific goals or inspect the outcome.  Two examples are very efficient
in exposing the procedures and potentials: one that is continuous within few
months, one that is ephemeral and occurs in only a few hours or less.  In such
a diversity-rich setting, these experimental procedures were called
\emph{technoxamanic} experiments (or 'yeys' for Brazilian Portuguese '{\bf
ebós}').

\subsection{The Cow of the End of the World (continuous experiment)}

\subsection{Betweenness VS strength (ephemeral experiment)}

\subsection{Massive tagging (semi-ephemeral experiment)}

\subsection{Video-conferences, Etherpads, websites, gadgets, and whatnot}

      
\section{So what and how? Or the outcomes}

\section{By whom? Or galleries / memorial}
posts with many people marked.
limits on marking people, citing them in comments,
private messages.

\section{What shall we remember? Or memory and narrative}


\end{document}


