%% USPSC-modelo.tex
% ---------------------------------------------------------------
% USPSC: Modelo de Trabalho Academico (tese de doutorado, dissertacao de
% mestrado e trabalhos monograficos em geral) em conformidade com 
% ABNT NBR 14724:2011: Informacao e documentacao - Trabalhos academicos -
% Apresentacao
%----------------------------------------------------------------
%% Esta é uma customização do abntex2-modelo-trabalho-academico.tex de v-1.9.5 laurocesar 
%% para as Unidades do Campus USP de São Carlos:
%% EESC - Escola de Engenharia de São Carlos
%% IAU - Instituto de Arquitetura e Urbanismo
%% ICMC - Instituto de Ciências Matemáticas e de Computação
%% IFSC - Instituto de Física de São Carlos
%% IQSC - Instituto de Química de São Carlos
%%
%% Este trabalho utiliza a classe USPSC.cls que é mantida pela seguinte equipe:
%% 
%% Programação:
%%   - Marilza Aparecida Rodrigues Tognetti - marilza@sc.usp.br (PUSP-SC)
%%   - Ana Paula Aparecida Calabrez - aninha@sc.usp.br (PUSP-SC)
%% Normalização e Padronização:
%%   - Brianda de Oliveira Ordonho Sigolo - brianda@usp.br (IAU)
%%   - Elena Luzia Palloni Gonçalves - elena@sc.usp.br (EESC)
%%   - Eliana de Cássia Aquareli Cordeiro - eliana@iqsc.usp.br (IQSC)
%%   - Flávia Helena Cassin - cassinp@sc.usp.br (EESC)
%%   - Maria Cristina Cavarette Dziabas - mcdziaba@ifsc.usp.br (IFSC)
%%   - Regina Célia Vidal Medeiros - rcvmat@icmc.usp.br (ICMC)
%%
%% O USPSC-modelo.tex utiliza:	
%%  USPSC.cls e USPSC1.cls
%% 	USPSC-modelo-references.bib
%%	USPSC-modelo.tex
%%	USPSC-unidades.tex
%%	Um dos arquivos com dados pre-textuais abaixo, em conformidade com a Unidade de vínculo do autor:
%%				USPSC-pre-textual-EESC.tex
%%				USPSC-pre-textual-IAU.tex
%%				USPSC-pre-textual-ICMC.tex
%%				USPSC-pre-textual-IFSC.tex
%%				USPSC-pre-textual-IQSC.tex
%%				USPSC-pre-textual-OUTRO.tex
%%	USPSC-fichacatalografica.tex ou fichacatalografica.pdf
%%	folhadeaprovacao.pdf
%%	USPSC-Cap1-Introducao.tex
%%	USPSC-Cap2-Desenvolvimento.tex
%%	USPSC-Cap3-Citacoes.tex
%%	USPSC-Cap4-referencias.tex
%%	USPSC-Cap5-Conclusao.tex
%%	USPSC-Apendices.tex
%%	USPSC-Anexos.tex
%%	USPSC-AcentuacaoLaTeX.tex
%%	USPSC-LetrasGregas.tex
%%	USPSC-SimbolosUteis.tex

%----------------------------------------------------------------
%% Sobre a classe abntex2.cls:
%% abntex2.cls, v-1.9.5 laurocesar
%% Copyright 2012-2015 by abnTeX2 group at https://www.abntex.net.br/ 
%%
%----------------------------------------------------------------

\documentclass[
% -- opções da classe memoir --
12pt,		% tamanho da fonte
openright,	% capítulos começam em pág ímpar (insere página vazia caso preciso)
twoside,  % para impressão em anverso (frente) e verso. Oposto a oneside - Nota: utilizar \imprimirfolhaderosto*
%oneside, % para impressão em páginas separadas (somente anverso) -  Nota: utilizar \imprimirfolhaderosto
% inclua uma % antes do comando twoside e exclua a % antes do oneside 
a4paper,			% tamanho do papel. 
% -- opções da classe abntex2 --
chapter=TITLE,		% títulos de capítulos convertidos em letras maiúsculas
% -- opções do pacote babel --
english,			% idioma adicional para hifenização
french,				% idioma adicional para hifenização
spanish,			% idioma adicional para hifenização
brazil				% o último idioma é o principal do documento
% {USPSC} configura o cabeçalho contendo apenas o número da página
]{USPSC}
%]{USPSC1}
% Inclua % antes de ]{USPSC} e retire a % antes de %]{USPSC1}
% para utilizar o cabeçalho diferenciado para as páginas pares e ímpares como indicado abaixo:
%- páginas ímpares: cabeçalho com seções ou subseções e o número da página
%- páginas pares: cabeçalho com o número da página e o título do capítulo 
% ---

% ---
% Pacotes básicos - Fundamentais 
% ---

% \DeclareUrlCommand\url{\def\UrlLeft{<}\def\UrlRight{>}%
% \DeclareUrlCommand\url{
% \renewcommand\UrlFont{\color{blue}}
% \renewcommand\UrlLeft{}
% \renewcommand\UrlRight{}
% }
% \def\url@#1{\hyper@linkurl{\ULurl@@{#1}}{#1}}
% \DeclareRobustCommand*\url{\hyper@normalise\url@}
% \urlstyle{tt}}
% \def\UrlLeft#1\UrlRight{%
% 	\special{html:<a href="#1">}#1\special{html:</a>}}
% \def\UrlLeft#1\UrlRight{%
% \special{html:<a href="#1">}#1\special{html:</a>}}

\usepackage{amsmath}
\usepackage{placeins}
\renewcommand{\bf}{\bfseries}
\usepackage[T1]{fontenc}		% Seleção de códigos de fonte.
\usepackage[utf8]{inputenc}		% Codificação do documento (conversão automática dos acentos)
\usepackage{lmodern}			% Usa a fonte Latin Modern
% Para utilizar a fonte Times New Roman, inclua uma % no início do comando acima  "\usepackage{lmodern}"
% Abaixo, tire a % antes do comando  \usepackage{times}
%\usepackage{times}		    	% Usa a fonte Times New Roman	
% \usepackage[sorting=none]{biblatex}
% Lembre-se de alterar a fonte no comando que imprime o preâmbulo no arquivo da Classe USPSC.cls				
\usepackage{lastpage}			% Usado pela Ficha catalográfica
\usepackage{notoccite}
\usepackage{indentfirst}		% Indenta o primeiro parágrafo de cada seção.
\usepackage{color}				% Controle das cores
\usepackage[usenames,dvipsnames]{xcolor}
\usepackage{graphicx}			% Inclusão de gráficos
\usepackage{float} 				% Fixa tabelas e figuras no local exato
\usepackage{chemfig,chemmacros} % Para escrever reações químicas
\usepackage{microtype} 			% para melhorias de justificação
\usepackage{pdfpages}
\usepackage{makeidx}            % para gerar índice remissimo
\usepackage{subfig}

\usepackage{pdfpages} % TTM
\usepackage{incgraph,tikz}
% \usepackage[titles]{tocloft}

\newcommand{\ops}{\textsc{ops}}
\newcommand{\opsi}{O\textsc{ps}}
\newcommand{\vcps}{\textsc{vcps}}
\newcommand{\owl}{\textsc{owl}}
\newcommand{\owli}{O\textsc{wl}}
\newcommand{\sparql}{\textsc{s}par\textsc{ql}}
\newcommand{\bfo}{\textsc{bfo}}
\newcommand{\dbpedia}{\textsc{db}pedia}
\newcommand{\foaf}{\textsc{foaf}}
\newcommand{\ict}{\textsc{ict}}
\newcommand{\html}{\textsc{html}}
\newcommand{\node}{\textsc{n}ode.js}
\newcommand{\facebook}{\textsc{f}acebook}
\newcommand{\twitter}{\textsc{t}witter}
\newcommand{\wwwc}{\textsc{w3c}}
\newcommand{\skos}{\textsc{skos}}
\newcommand{\etherpad}{\textsc{e}therpad}
\newcommand{\ogp}{\textsc{ogp}}
\newcommand{\iri}{\textsc{iri}}
\newcommand{\uri}{\textsc{uri}}
\newcommand{\urii}{U\textsc{ri}}
\newcommand{\urll}{\textsc{url}}
\newcommand{\ngo}{\textsc{ngo}}
\newcommand{\http}{\textsc{http}}
\newcommand{\opa}{\textsc{op}a}
\newcommand{\ocd}{\textsc{ocd}}
\newcommand{\ontologiaa}{\textsc{o}ntologiaa}
\newcommand{\obs}{\textsc{obs}}
\newcommand{\pubby}{\textsc{p}ubby}
\newcommand{\rdf}{\textsc{rdf}}
\newcommand{\mysql}{\textsc{m}y\textsc{sql}}
\newcommand{\aan}{\textsc{aa}}
\newcommand{\cidadedemocratica}{\textsc{c}idade \textsc{d}emocr\'atica}
\newcommand{\participa}{\textsc{p}articipa.br}
\newcommand{\ontop}{\textsc{o}n\textsc{t}op}
\newcommand{\quest}{\textsc{q}uest}
\newcommand{\webprotege}{\textsc{w}ebprotege}
\newcommand{\obda}{\textsc{obda}}
\newcommand{\pnud}{\textsc{undp}}
\newcommand{\onu}{\textsc{un}}
\newcommand{\vbs}{\textsc{vbs}}
\newcommand{\lod}{\textsc{lod}}
\newcommand{\corais}{\textsc{c}orais}
\newcommand{\serpro}{\textsc{s}erpro}
\newcommand{\python}{\textsc{p}ython}
\newcommand{\protege}{\textsc{p}rot\`eg\`e}
% ---

% ---
% Pacotes de citações
% Citações padrão ABNT
% ---
% Sistemas de chamada: autor-data ou numérico.
% Sistema autor-data
%\usepackage[alf,abnt-emphasize=bf, abnt-thesis-year=both, abnt-repeated-author-omit=yes, abnt-last-names=abnt, abnt-etal-cite,abnt-etal-list=3, abnt-etal-text=default, abnt-and-type=e, abnt-doi=doi, abnt-url-package=none, abnt-verbatim-entry=no]{abntex2cite}

% Para o IQSC, que indica todos os autores nas referências, incluir % no início do comando acima e retirar a % do comando abaixo 

%\usepackage[alf,abnt-emphasize=bf, abnt-thesis-year=both, abnt-repeated-author-omit=yes, abnt-last-names=abnt, abnt-etal-cite,abnt-etal-list=0, abnt-etal-text=default, abnt-and-type=e]{abntex2cite}

% Sistema Numérico
% Para citações numéricas, sistema adotado pelo IFSC, incluir % no início do comando acima e retirar a % do comando abaixo 
\usepackage[num,overcite,abnt-emphasize=bf, abnt-thesis-year=both, abnt-repeated-author-omit=yes, abnt-last-names=abnt, abnt-etal-cite,abnt-etal-list=0, abnt-etal-text=default, abnt-and-type=e]{abntex2cite}
% Complementarmente, verifique as instruções abaixo sobre os Pacotes de Nota de rodapé
% ---
% Pacotes de Nota de rodapé
% Configurações de nota de rodapé

% O presente modelo adota o formato numérico para as notas de rodapés quando utiliza o sistema de chamada autor-data para citações e referências. Para utilizar o sistema de chamada numérico para citações e referências, habilitar um dos comandos abaixo.
% Há diversa opções para nota de rodapé no Sistema Numérico.  Para o IFSC, habilitade o comando abaixo.

\renewcommand{\thefootnote}{\fnsymbol{footnote}}  %Comando para inserção de símbolos em nota de rodapé

% Outras opções para nota de rodapé no Sistema Numérico:
%\renewcommand{\thefootnote}{\alph{footnote}}      %Comando para inserção de letras minúscula em nota de rodapé
%\renewcommand{\thefootnote}{\Alph{footnote}}      %Comando para inserção de letras maiúscula em nota de rodapé
%\renewcommand{\thefootnote}{\roman{footnote}}     %Comando para inserção de números romanos minúsculos  em nota de rodapé
%\renewcommand{\thefootnote}{\Roman{footnote}}     %Comando para inserção de números romanos minúsculos  em nota de rodapé

% \renewcommand{\footnotesize}{\small} %Comando para diminuir a fonte das notas de rodapé

 % ---
 % Pacote para agrupar a citação numérica consecutiva
 % Quando for adotado o Sistema Numérico, a exemplo do IFSC, habilite 
 % o pacote cite abaixo retirando a porcentagem antes do comando abaixo
 \usepackage[superscript]{cite}	

% ---
% Pacotes adicionais, usados apenas no âmbito do Modelo Canônico do abnteX2
% ---
\usepackage{lipsum}				% para geração de dummy text
% ---

\usepackage{placeins}				% para geração de dummy text



% pacotes de tabelas
\usepackage{multicol}	% Suporte a mesclagens em colunas
\usepackage{multirow}	% Suporte a mesclagens em linhas
\usepackage{longtable}	% Tabelas com várias páginas
\usepackage{threeparttablex}    % notas no longtable
\usepackage{array}

%---
% Configurações para o pacote chemfig
%\chemsetup[chemformula]{format=\sffamily}
\renewcommand*\printatom[1]{\ensuremath{\mathsf{#1}}}
\setatomsep{2em}
\setdoublesep{.6ex}
\setbondstyle{semithick}
%---
% Configurando o ambiente para utilizar os recursos de frases pre-prontas do mhchem
\newenvironment{rslist}%
{%
	\begin{labeling}% environment from KOMA-script
		{\rsnumber{R39/23/24/25}}% R39/23/24/25 is longest label
	}{%
\end{labeling}%
}%
% Definição de comando para utilizar os recursos de frases pre-prontas do mhchem
\newcommand{\rs}[2][]{\item[{\rsnumber[#1]{#2}}] \rsphrase{bb}}
% ---
% \usepackage[font={footnotesize},justification=raggedleft,singlelinecheck=false,indention=true]{caption}
\usepackage[font={footnotesize},format=hang,labelsep=endash]{caption}
% \captionsetup{labelsep=endash}% ---
% DADOS INICIAIS - Define sigla com título, área de concentração e opção do Programa 
% Consulte a tabela referente aos Programas, áreas e opções de sua unidade contante do
% arquivo USPSC-Siglas estabelecidas para os Programas de Pós-Graduação por Unidade.xlsx 
% ou nos APÊNDICES A-E
\siglaunidade{IFSC}
\programa{DFAFCe}
% Os demais dados deverão ser fornecidos no arquivo USPSC-pre-textual-UUUU, onde UUUU é a sigla da Unidade. 
% Exemplo:USPSC-pre-textual-IFSC.tex
% ---
% Configurações de aparência do PDF final
% alterando o aspecto da cor azul
\definecolor{blue}{RGB}{41,5,195}

% informações do PDF
\makeatletter
\hypersetup{
	%pagebackref=true,
	pdftitle={\@title}, 
	pdfauthor={\@author},
	pdfsubject={\imprimirpreambulo},
	pdfcreator={LaTeX with abnTeX2},
	pdfkeywords={abnt}{latex}{abntex}{USPSC}{trabalho acadêmico}, 
	colorlinks=true,       		% false: boxed links; true: colored links
	linkcolor=blue,          	% color of internal links
	citecolor=blue,        		% color of links to bibliography
	filecolor=magenta,      		% color of file links
	urlcolor=blue,
	allbordercolors=black,
	bookmarksdepth=4
}
\newcommand{\uurl}[1]{\href{#1}{\color{black}<\url{#1}>}}
\makeatother
% --- 
% --- 
% Espaçamentos entre linhas e parágrafos 
% --- 

% O tamanho do parágrafo é dado por:
\setlength{\parindent}{1.3cm}

% Controle do espaçamento entre um parágrafo e outro:
\setlength{\parskip}{0.2cm}  % tente também \onelineskip

% ---
% compila o sumário e índice
\makeindex
% ---

% ----
% Início do documento
% ----
\begin{document}

% Seleciona o idioma do documento (conforme pacotes do babel)
% \selectlanguage{brazil}
% Se o idioma do texto for inglês, inclua uma % antes do 
%      comando \selectlanguage{brazil} e 
%      retire a % antes do comando abaixo
\selectlanguage{english}

% Retira espaço extra obsoleto entre as frases.
\frenchspacing 

% --- Formatação dos Títulos
\renewcommand{\ABNTEXchapterfontsize}{\fontsize{12}{12}\bfseries}
\renewcommand{\ABNTEXsectionfontsize}{\fontsize{12}{12}\bfseries}
\renewcommand{\ABNTEXsubsectionfontsize}{\fontsize{12}{12}\normalfont}
\renewcommand{\ABNTEXsubsubsectionfontsize}{\fontsize{12}{12}\normalfont}
\renewcommand{\ABNTEXsubsubsubsectionfontsize}{\fontsize{12}{12}\normalfont}

\newcommand{\textttt}[1] {\texttt{\footnotesize#1}} % TTM
\newcommand{\h} {\hphantom ~ }

% \renewcommand{\url}[1]{\url{#1}}

% ----------------------------------------------------------
% ELEMENTOS PRÉ-TEXTUAIS
% ----------------------------------------------------------
% ---
% Capa
% ---
\imprimircapa
% ---
% Folha de rosto
% (o * indica impressão em anverso (frente) e verso )
% ---
\imprimirfolhaderosto*
%\imprimirfolhaderosto
% ---

% ---
% Inserir a ficha catalográfica em pdf
% ---
% A biblioteca da sua Unidade lhe fornecerá um PDF com a ficha
% catalográfica definitiva. 
% Quando estiver com o documento, salve-o como PDF no diretório
% do seu projeto como fichacatalografica.pdf e iclua o arquivo
% utilizando o comando abaixo:
\begin{fichacatalografica}
   \includepdf{fichacatalografica___.pdf}
\end{fichacatalografica}
% Se você optar por elaborar a ficha catalográfica, deverá 
% incluir uma % antes das 3 linhas acima e tirar a % antes
% do comando \include{USPSC-fichacatalografica}
%\include{USPSC-fichacatalografica}
% As informações que compõem a ficha catalográfica estão 
% definidos no arquivo USPSC-pre-textual-UUUU.tex
% ---


% ---
% Inserir errata
% ---

% \begin{errata}
% 	\OnehalfSpacing 			
% 	A errata é um elemento opcional, que consiste de uma lista de erros da obra, precedidos pelas folhas e linhas onde eles ocorrem e seguidos pelas correções correspondentes. Deve ser inserida logo após a folha de rosto e conter a referência do trabalho para facilitar sua identificação, conforme a ABNT NBR 14724. \cite{nbr14724}.
% 	
% 	Modelo de Errata:
% 		
% 	\begin{flushleft}\footnotesize 
% 			\setlength{\absparsep}{0pt} % ajusta o espaçamento da referência	
% 			\SingleSpacing 
% 			\imprimirautorabr~ ~\textbf{\imprimirtitulo}.	\imprimirdata. \pageref{LastPage}p. 
% 			%Substitua p. por f. quando utilizar oneside em \documentclass
% 			%\pageref{LastPage}f.
% 			\imprimirtipotrabalho~-~\imprimirinstituicao, \imprimirlocal, \imprimirdata. 
%  	\end{flushleft}
% \vspace{\onelineskip}
% \OnehalfSpacing 
% \center
% \textbf{ERRATA}
% \vspace{\onelineskip}
% \OnehalfSpacing 
% \begin{table}[htb]
% 	\center
% 	\footnotesize
% 	\begin{tabular}{p{1.4cm} p{1cm} p{3cm} p{3cm} }
% 		\hline
% 		\textbf{Folha} & \textbf{Linha}  & \textbf{Onde se lê}  & \textbf{Leia-se}  \\
% 			\hline
% 			1 & 10 & auto-conclavo & autoconclavo\\
% 		\hline
% 	\end{tabular}
% \end{table}
% 
% 
% \end{errata}
% ---

% ---
% Inserir folha de aprovação
% ---

% A Folha de aprovação é um elemento obrigatório da NBR 4724/2011 (seção 4.2.1.3). 
% Após a defesa/aprovação do trabalho, gere o arquivo folhadeaprovacao.pdf da página assinada pela banca 
% e iclua o arquivo utilizando o comando abaixo:
% \includepdf{folhadeaprovacao.pdf}
% \includepdf{PaginaEmBranco.pdf}

% Alternativa para a Folha de Aprovação:
% Se for a sua opção elaborar uma folha de aprovação, insira uma % antes do comando acima que inclui o arquivo folhadeaprovacao.pdf,
% tire o % do comando abaixo e altere o arquivo folhadeaprovacao.tex conforme suas necessidades
%\include{folhadeaprovacao}

% \includepdf{PaginaEmBranco.pdf}

% ---
% Dedicatória
% ---
\begin{dedicatoria}
   \vspace*{\fill}
   \centering
   \noindent
%   \textit{ Este trabalho é dedicado à Deus e à minha família.} \vspace*{\fill}
   \textit{ This work is dedicated to God and my family, whose constant support made it possible.} \vspace*{\fill}
\end{dedicatoria}
% ---

% ---
% Agradecimentos
% ---=====
\begin{agradecimentos}
I thank my advisor
Prof. Dr. Osvaldo Novais de Oliveira Junior
for his patience and prompt responses.	
I thank Prof. Dr. Luciano da Fontoura Costa
whose research also inspired this thesis
and for introducing me to complex networks.
I thank Prof. Dr. Leonardo Paulo Maia for important discussions regarding statistics and
written discourse.
I thank the Institute of Advanced Studies (IEA/USP)
for enabling me to receive guest researchers and hosting meetings in the years of 2013 and 2014.
I thank Prof. Dr. Yvonne Primerano Mascarenhas (IFSC/USP) for her time in articulating with the IEA.
I thank Prof. Dr. Dilvan de Abreu Moreira (ICMC/USP) for guiding me in understanding linked data/semantic web.
I thank the members of the São Carlos Institute of Physics,
including the instructors, the administration and the secretariat
for the mindfulness and patience whenever I needed to reach them.

	I thank the Brazilian Presidency and the United Nations Development Program (UNDP) for
the partnership established with this research in the year of 2014 (consulting contract 2013/00056, project BRA/12/018).
I thank Ricardo Poppi for his time, experience and mentoring while he was the general coordinator of new media of the Brazilian Presidency.
	I thank the Military from the Brazilian Army Prof. Ronald Emerson Scherolt da Costa for assisting the mentorship.
	I thanks Fabricio Solagna, Grazielle Machado, Ana Célia Costa,
	Joenio Marques da Costa, Daniela Feitosa, Prof. Dr. Paulo Meirelles (UnB), Prof. Dr. Fernando William Cruz (UnB), Antonio Carlos Wosgrau,
	Cintia Cinquini, Anjuli Tostes Faria Osterne and Prof. Dr. Frederick van Amstel (UFPR) for
a series of occasions in which I had the opportunity to learn and maybe contribute to State affairs
regarding interfaces with the civil society and the introduction of scientific knowledge and technologies.
I thank the government and State specialists and authorities for the interviews in which they
allowed me to formalize conceptualizations of social participation instances and mechanisms:
conference, forum, council, ombudsman, public consultation, dialog table, monitoring table.
	Namely, I thank Pedro Pontual, Dra. Maria Carmen Romcy de Carvalho (Biblioteca Digital de Participação Social - PR), Prof. Dr. Fernando Cruz, Clovis Souza, Paula Pompeu, Lígia Maria Alves Pereira,
Valéssio Brito, Fernanda Lobato, Fabiano Rangel, Anjuli Osterne and Paulo Guimarães.
Some of the input was obtained in a workshop held by the presidency,
reason why I don't have the full name of the following contributors
which I also thank: Silvia, Roberto and Márcia.

I thank the Nexos research network for the precious support in deepening concepts related
to the study of human individuals and groups.
A special thanks here goes to the philosopher Prof. Dr. Marília Mello Pisani (CCNH/UFABC) and the social psychologist 
	Prof. Dr. Deborah Christina Antunes (Psychology Course/UFC/Sobral Campus).
I also thank the anthropologist Prof. Dr. Massimo Canevacci (Università di Roma "La Sapienza" and IEA/USP)
for his presence and tolerance in meetings and artistic presentations.

I thank the labMacambira.sf.net collective for numerous discussions and guidance
concerning free and digital culture.
Namely, I thank Prof. Dr. Ricardo Fabbri (IPRJ/UERJ), Vilson Vieira da Silva Junior, Daniel Penalva, Caleb Mascarenhas Luporini, Danilo Shiga,
Dr. Antonio Pessotti, Guilherme Lunhani, Geraldo Magela de Castro Rocha Junior and Dr. Carlos Lobo for insights and fruitful discussions.
I thank the architect and cultural activist Daniel Marostegan e Carneiro for his invaluable support in
the foundation of the labMacambira.sf.net and in the advancement of digital culture conceptualizations.
I thank the Ethymos company and the cultural center Teia Casa de Criação/Pontão Nós Digitais for hosting experiments, webpages, bots and documents in their online infrastructure for years.
I thank João Paulo Mehl, Marco Antonio Konopacki and Jacson Passold for their time, patience and trust in allowing me to benefit from the facilities of Ethymos.

I thank the Metareciclagem network for enabling me to start this research by proposing
a study about the ``end of the world''.
	Special thanks here goes to Prof. Guilherme Rafael (Glerm) Soares (CECULT/UFRB), Simone Bittencourt and Adriano Belisário
for sharing their time and understandings.
I distinctively admire and benefit from what Prof. Glerm yields.

I thank the linguist Dr. Chandra Wood Viegas for her stimulant interest
which culminated in collaborations and citations of preliminary results of this research in her doctorate thesis
	about the indigenous Kokama culture.

I thank my family for the invaluable support in every step.
I thank my parents, the mathematician Cláudia Maria Costa Lopes (teaches at Petrópolis/RJ) and the
physicist Prof. Dr. Maurício Fabbri (LAS/INPE and USF), for the good education that
allowed me to strive in a number of fields of knowledge 
and develop technological, social and artistic artifacts.
I also thank my stepfather, the biophysicist Prof. Dr. Francisco José Pereira Lopes (UFRJ),
for his time in a number of discussions.
I value exceptionally the influence my brother Prof. Dr. Ricardo Fabbri (IPRJ/UERJ)
has in my current appreciation
of the sciences and the academia and computer technologies.
He encouraged me for decades to pursue a deeper background and believed in my ability to acquire and yield knowledge.

I thank the National Counsel of Technological and Scientific Development (CNPq)
for the financial support (140860/2013-4, project 870336/1997-5).
I thank and admire the researchers that keep their works freely available online,
specially those who make video lectures e.g. on Coursera, edX and Complexity Explorer (Santa Fe Institute).
I thank the open source and free software communities for sharing their work,
which made possible the developments presented in this thesis.
% SGPR, PNUD
% labmacambira
% wife and family
% O Grupo desenvolvedor do Pacote USPSC, atualmente composto da Classe USPSC e do  Modelo para teses e dissertações em \LaTeX\ utilizando a classe USPSC, agradece especialmente ao Luis Olmes, doutorando do Instituto de Ciências Matemáticas e de Computação (ICMC) da Universidade de São Paulo (USP), pelas primeiras orientações sobre o \LaTeX\ .
% 
% Agradecemos ao Lauro César Araujo pelo desenvolvimento da classe  \abnTeX, modelos canônicos e tantas outras contribuições que nos permitiu o desenvolvimento da classe USPSC e seus modelos.
% 
% Os nossos agradecimentos aos integrantes do primeiro
% projeto abn\TeX\, Gerald Weber, Miguel Frasson, Leslie H. Watter, Bruno Parente Lima, Flávio de Vasconcellos Corrêa, Otavio Real
% Salvador, Renato Machnievscz, e a todos que contribuíram para que a produção de trabalhos acadêmicos em conformidade com
% as normas ABNT com \LaTeX\ fosse possível.
% 
% Agradecemos ao grupo de usuários
% \emph{latex-br}{\url{http://groups.google.com/group/latex-br}}, aos integrantes do grupo
% \emph{\abnTeX}{\url{http://groups.google.com/group/abntex2}  e \url{http://www.abntex.net.br/}}~que contribuem para a evolução do \abnTeX.
\end{agradecimentos}
% ---

% ---
% Epígrafe
% ---
\begin{epigrafe}
    \vspace*{\fill}
	\begin{flushright}
		% \textit{``The heart of the discerning acquires knowledge, \\
		% for the ears of the wise seek it out.''\\
		% Proverbs 18:15}
		\textit{``Call to me and I will answer you and tell you\\
		great and unsearchable things you do not know.''\\
		Jeremiah 33:3}
	\end{flushright}
\end{epigrafe}
% ---


% ---
% Abstract
% ---
\autor{Fabbri, R.}
\begin{resumo}[Abstract]
	\begin{flushleft}\footnotesize 
		\setlength{\absparsep}{0pt} % ajusta o espaçamento dos parágrafos do resumo		
 		\SingleSpacing 
 		\imprimirautorabr~ ~\textbf{\imprimirtitleabstract}.	\imprimirdata.  \pageref{LastPage}p. 
		%Substitua p. por f. quando utilizar oneside em \documentclass
		%\pageref{LastPage}f.
		\imprimirtipotrabalho~-~\imprimirinstituicao, \imprimirlocal, 	\imprimirdata. 
 	\end{flushleft}
	\OnehalfSpacing 
	 This work reports on stable (or invariant) topological properties and textual differentiation in human interaction networks,
	 with benchmarks derived from public email lists.
	 Activity along time and topology were observed in snapshots in a timeline, and at
	 different scales. Our analysis shows that activity is practically the same for all networks across timescales
	 ranging from seconds to months. The principal components of the participants in the topological metrics
	 space remain practically unchanged as different sets of messages are considered.
	 The activity of participants
	 follows the expected scale-free outline, thus yielding the hub, intermediary and peripheral classes of vertices by
	 comparison against the Erdös-Rényi model.
	 The relative sizes of these three sectors are essentially the same
	 for all email lists and the same along time.
	 Typically, 3-12\% of the vertices are hubs, 15-45\% are intermediary
	 and 44-81\% are peripheral vertices.
	 Texts from each of such sectors are shown to be very different through direct measurements and through an adaptation of the Kolmogorov-Smirnov test.
	 These properties are consistent with the literature and may be general for human
	 interaction networks, which has important implications for establishing a typology of participants based on
	 quantitative criteria.
	 For guiding and supporting this research, we also developed a visualization method of dynamic networks through animations.
	 To facilitate verification and further steps in the analyses, we supply a linked data representation of data related to our results.
   \vspace{\onelineskip}
 
   \noindent 
   \textbf{Keywords}: Complex networks. Text mining. Pattern recognition. Social network analysis. Linked data.
\end{resumo}

% ---
% Resumo
% ---
% Se o idioma do texto for em inglês, o abstract deve preceder o resumo
% resumo em português
\setlength{\absparsep}{18pt} % ajusta o espaçamento dos parágrafos do resumo		
\titulo{Estabilidade topológica e diferenciação textual em redes de interação humana: análise estatística, visualização e dados ligados}
     	\tipotrabalho{Tese (Doutorado em Ci\^encias)}
        \area{F\'isica Aplicada}
        \opcao{F\'isica Computacional}
\begin{resumo}[Resumo]
 \begin{otherlanguage*}{portuguese}
	\begin{flushleft}\footnotesize 
			\setlength{\absparsep}{0pt} % ajusta o espaçamento da referência	
			\SingleSpacing 
			\imprimirautorabr~ ~\textbf{\imprimirtitulo}.	\imprimirdata. \pageref{LastPage}p. 
			%Substitua p. por f. quando utilizar oneside em \documentclass
			%\pageref{LastPage}f.
			\imprimirtipotrabalho~-~\imprimirinstituicao, \imprimirlocal, \imprimirdata. 
 	\end{flushleft}
\OnehalfSpacing 			
	 Este trabalho relata propriedades topológicas estáveis (ou invariantes) e diferenciação textual em redes de interação humana,
	 com referências derivadas de listas públicas de e-mail.
	 A atividade ao longo do tempo e a topologia foram observadas em instantâneos ao longo de uma linha do tempo e em
	 diferentes escalas.
	 A análise mostra que a atividade é praticamente a mesma para todas as redes em escalas temporais
	 de segundos a meses.
	 As componentes principais dos participantes no espaço das métricas topológicas
	 mantêm-se praticamente inalteradas quando diferentes conjuntos de mensagens são considerados.
	 A atividade dos participantes
	 segue o esperado perfil livre de escala, produzindo, assim, as classes de vértices dos hubs, dos intermediários e dos periféricos em
	 comparação com o modelo Erdös-Rényi.
	 Os tamanhos relativos destes três setores são essencialmente os mesmos
	 para todas as listas de e-mail e ao longo do tempo.
	 Normalmente, 3-12\% dos vértices são hubs, 15-45\% são intermediários
	 e 44-81\% são vértices periféricos.
	 Os textos de cada um destes setores são considerados muito diferentes através de uma adaptação dos testes de Kolmogorov-Smirnov.
	 Estas propriedades são consistentes com a literatura e podem ser gerais para
	 redes de interação humana, o que tem implicações importantes para o estabelecimento de uma tipologia dos participantes com base em
	 critérios quantitativos.
	 De modo a guiar e apoiar esta pesquisa, também desenvolvemos um método de visualização para redes dinâmicas através de animações.
	 Para facilitar a verificação e passos seguintes nas análises, fornecemos uma representação em dados ligados dos dados relacionados aos nossos resultados.

 \textbf{Palavras-chave}: Redes complexas. Mineração de texto. Reconhecimento de padrões. Análise de redes sociais. Dados ligados.
 \end{otherlanguage*}
\end{resumo}


% ---
% inserir lista de ilustrações
% ---
\pdfbookmark[0]{\listfigurename}{lof}
\listoffigures*
\cleardoublepage
% ---

% ---
% inserir lista de tabelas
% ---
\pdfbookmark[0]{\listtablename}{lot}
\listoftables*
\cleardoublepage
% ---

% ---
% inserir lista de quadros
% ---
% \pdfbookmark[0]{\listofquadroname}{loq}
% \listofquadro*
% \cleardoublepage
% ---

% ---
% inserir lista de abreviaturas e siglas
% ---
\begin{siglas}
	\item[AA] Algorithmic Autoregulation
	\item[HTML] HyperText Markup Language
	\item[HTTP] HyperText Transfer Protocol
	\item[IRC] Internet Relay Chat
	\item[LOSD] Linked Open Social Database
	\item[PCA] Principal Component Analysis
	\item[POS] Part-Of-Speech
	\item[RDF] Resource Description Framework
	\item[SPARQL] SPARQL Protocol and RDF Query Language
	\item[URI] Uniform Resource Identifier
	\item[Versinus] Line and sine (Latin versus+sinus) layout for graph visualization.
	\item[W3C] World Wide Web Consortium
\end{siglas}
% \begin{siglas}
%     \item[ABNT] Associação Brasileira de Normas Técnicas
%     \item[abnTeX] ABsurdas Normas para TeX
% 	\item[EESC] Escola de Engenharia de São Carlos
% 	\item[IAU] Instituto de Arquitetura e Urbanismo
% 	\item[IBGE] Instituto Brasileiro de Geografia e Estatística
% 	\item[ICMC] Instituto de Ciências Matemáticas e de Computação
% 	\item[IFSC] Instituto de Física de São Carlos
% 	\item[IQSC] Instituto de Química de São Carlos
% 	\item[USP] Universidade de São Paulo
% 	\item[USPSC] Campus USP de São Carlos
% 	
% \end{siglas}
% ---

% ---
% inserir lista de símbolos
% ---
% \begin{simbolos}
%   \item[$ \Gamma $] Letra grega Gama
%   \item[$ \Lambda $] Lambda
%   \item[$ \zeta $] Letra grega minúscula zeta
%   \item[$ \in $] Pertence
% \end{simbolos}
% ---
% ---
% inserir o sumario
% ---
\pdfbookmark[0]{\contentsname}{toc}
\tableofcontents*
\cleardoublepage
% ---
% ----------------------------------------------------------
% ELEMENTOS TEXTUAIS
% ----------------------------------------------------------
\textual
% Os capítulos são inseridos como arquivos externos 

% Capítulo 1 - Introdução
% ---
%% USPSC-Introducao.tex

% ----------------------------------------------------------
% Introdução (exemplo de capítulo sem numeração, mas presente no Sumário)
% ----------------------------------------------------------
% \chapter[Introdução]{Introdução}
\chapter{Introduction}\label{ch:int}

The first studies dealing explicitly with human interaction networks
date from the nineteenth century while the foundation of
social network analysis is generally attributed to the psychiatrist Jacob Moreno in mid twentieth century~\cite{moreno,newmanBook}.
With the increasing availability of data related to human interactions, research about these networks has grown continuously.
Contributions can now be found in a variety of fields, from social sciences and humanities~\cite{latour2013} to computer science~\cite{bird} and physics~\cite{barabasiHumanDyn,newmanFriendship}, given the multidisciplinary nature of the topic.
One of the approaches from an exact science perspective is to represent interaction networks as complex networks~\cite{barabasiHumanDyn,newmanFriendship}, with which 
several features of human interaction have been revealed.
For example, the topology of human interaction networks exhibits a scale-free outline,
which points to the existence of a small number of highly connected hubs and a large number of poorly connected nodes.
The dynamics of complex networks representing human interaction has also been addressed~\cite{barabasiEvo,newmanEvolving}, but only to a limited extent, since research is normally focused on a particular metric or task, such as accessibility or community detection~\cite{access,newmanModularity}. 

There are numerous articles, books, websites and software tools about complex and social networks and about text mining in social media.
There are fewer endeavours to characterize these networks beyond general features such as the scale-free 
aspect or to deal with text produced by social networks from the complex networks background.
Research on network evolution is often restricted to network growth, in which there is a monotonic increase in the number of events~\cite{barabasiEvo}.
Network types have been discussed with regard to the number of participants, intermittence of their activity and network longevity~\cite{barabasiEvo}. Two topologically different networks emerged from human interaction networks, depending on whether the frequency of interactions follows a generalized power law or an exponential connectivity distribution~\cite{barabasiTopologicalEv}. In email list networks, scale-free properties were reported with $\alpha \approx 1.8$~\cite{bird} (as in web browsing and library loans~\cite{barabasiHumanDyn}), and different linguistic traces were related to weak and strong ties~\cite{Gmane2}.

The fact that unreciprocated edges often exceed 50\% in human interaction networks~\cite{newmanEvolving} motivated the inclusion of symmetry metrics in our analysis.
No correlation of topological characteristics and geographical coordinates was found~\cite{barabasiGeo},
therefore geographical positions were not considered in our study.
Gender related behavior in mobile phone datasets was indeed reported~\cite{barabasiSex}
but it is not relevant for the present work because email messages and addresses have no gender related metadata~\cite{gmanePack}.

\section{Related knowledge}
\subsection{Complex networks}
Although not universally accepted, it is commonplace to define a complex network to be
a ``graph with non-trivial topological features''.
We might add to this definition that a complex network is also a large graph (even while there
seems not to be a consensus to what \emph{large} means in such context)
and that it is a graph representation of a system found in nature or in real or empirical systems.
Another way to approach the definition of ``complex networks'' is to define it as
complex systems modeled as networks.
This second definition is also useful but is even more problematic as
there is no consensus of what a \emph{complex system} is.
Even so, one should keep in mind that authors often define a complex system
to be a system composed with many parts in which ``the whole is more than
the sum of its parts''.
Authors also often consider complex systems to have capabilities
to ``process information'', to adapt, to reproduce. 

A graph is a structure that consists of a set of objects (called vertices)
and a set of binary/dual relations of the objects (called edges).
Such graph might be unweighted and undirected (the simplest possibility),
weighted and undirected, unweighted and directed, or weighted and directed.

The most usual representations of graphs (and networks) are the matrix, list and node-edge representations.
In the matrix representation, each entry $a_{ij})$ is non-zero if $i$ is linked to $j$;
entries might be other than 0 and 1 in weighted graphs; undirected graphs yield symmetric matrices.
There are two common list representations of graphs, one lists each pair of vertices that are connected,
the other holds a list for each vertex in which are all the vertices connected to it (a list of lists).
In the node-edge representation, each node i represented as a point while each edge is represented
by a line between correspondent nodes.
The matrix representation is essential for algebraic reasoning and for deriving measures
while the node-edge representation is important for illustration and intuitive guidance
of the characterization of the systems.

\subsubsection{A good justification for the complex networks theory}
The estimated number of atoms in the universe is often used as a reference of largeness
and is $10^80$.
Let us find the number of vertices needed to reach such number of possible networks.
Let also us consider the simplest case of the unweighted and undirected networks.
Each edge can exist or not (i.e. it is a Bernoulli variable) and with $n$ vertices there are
at most ${n \choose 2}$ edges.
Therefore:
\begin{align}
	2^{n \choose 2} > 10^{80} \Rightarrow 
	log_2[2^{n \choose 2}] > log_2(10^{80}) \Rightarrow
	{n \choose 2} > \frac{log_{10}(10^{80})}{log_{10}2} \Rightarrow \nonumber\\
	\Rightarrow \frac{n.(n-1)}{2} > \frac{80}{log_{10}2} \Rightarrow
		N > 23,5988 \;\;\;\;\;\;\;\;\;\;\;\;\;\;\;\;\;\;\;\;\;
			\nonumber
\end{align}
That is, with only 24 vertices we have more possible networks than
the estimated number of atoms in the universe.
We should also add that the number of possible networks grows
very fast with the number of vertices.
This is a good reason for characterizing such systems by means
of paradigmatic networks and generic measures for nodes and the network (and less often for the edges).

\subsubsection{Basic measures}
Section~\ref{measures} gives a mathematical account of the following measures,
which are here for pointing the characteristics of basic types
of networks presented in the next section.
Such measures are:
\begin{itemize}
	\item Degree     $k_i$: number of edges linked to vertex $i$.
	\item In-degree  $k_i^{in}$: number of edges ending at vertex $i$.
	\item Out-degree $k_i^{out}$: number of edges departing from vertex $i$.
	\item Strength $s_i$: sum of weights of all edges linked to vertex $i$.
	\item In-strength $s_i^{in}$: sum of weights of all edges ending at vertex $i$.
	\item Out-strength $s_i^{out}$: sum of weights of all edges departing from vertex $i$.
	\item Betweenness centrality $bt_i$: fraction of geodesics that contain vertex $i$.
	\item Clustering coefficient $cc_i$: fraction of pairs of neighbors of $i$ that are linked, i.e. the standard clustering coefficient metric for undirected graphs.
\end{itemize}
% distance between vertices

\subsubsection{Basic types of networks}
Complex networks are often characterized in terms of paradigmatic models.
There are diverse models, but we can glimpse the background theory
with the following ones:
\begin{itemize}
	\item The Erdös-Rényi model\footnote{This name is also used for the model in which, for a fixed number of vertices and a fixed number of edges, all graphs are equally likely. This is the model originally introduces by Paul Erdös and Alfréd Rényi~\cite{erdosOrig}.
		We choose the definition given inline, which is closely related to the one given in this footnote, because it is more commonly used nowadays.}: each pair of nodes is connected with a fixed random probability $p$.
		This model presents a characteristic degree ($n.p$ where $n$ is the number of vertices), low clustering and low average distance between nodes.
	\item Spatial network, also called geographic network or geometric graph: nodes are located in a metric space and the probability that two vertices are connected is greater as the distance between them gets smaller. These networks present characteristic degrees, high clustering and large average distance between nodes.
	\item Small-word network: defined as a network where the typical distance between vertices grows with the logarithm of the number of nodes while the average clustering coefficient is not small (larger than e.g. in the Erdös-Rényi model).
		To construct a small-world network, start with a regular lattice in which each vertex is connected to $k$ nearest neighbors.
		Each edge is then rewired with probability $p$.
		With intermediate values of $p$ such as $0.01<p<0.1$, we obtain a network with both short average distance between vertices (as in the Erdös-Rényi model) and a high average clustering coefficient (as in the spatial network).
		This model presents also a characteristic degree.
	\item Scale-free networks: in which the degree distribution $p(k)$ follows a power law ($p(k)=C.k^{-\gamma}$ where C and k are constants).
		These networks are qualitatively characterized by the presence of a large number of poorly connected and of few highly connected vertices.
		Important is the absence of a characteristic degree, thus the name 'scale-free network'.
	\item Other networks: among important models of networks are exponential networks, networks with community structure and hybrid models.
\end{itemize}
Real networks most often exhibit scale-free and small-world properties.
This is the case of most of e.g. social, gene and food networks.
However, one should be cautious about such statement because
the networks derived from the real systems depend heavily
in what is considered a vertex and an edge,
i.e. on how the system is modeled as a graph.
Another noteworthy remark is that
the Erdös-Rényi networks, i.e. graphs of the Erdös-Rényi model, are frequently pin-pointed as the networks with trivial
topological properties.
Even though, it is posed as a paradigmatic ``complex network'', concept often defined as graphs with non-trivial topological properties,
which is a contradiction and exposes that complex networks is not a very well defined notion,
as is the case with the complexity field in general.

\subsection{Text mining of social data}
Text mining is a multidisciplinary field, 
it is an extension of data mining to (often unstructured) textual data
with the goal of discovering structure and meaning~\cite{customText}.
A general outline of a text mining endeavor involves structuring input text,
deriving patterns and the evaluation of the output.
There are actually numerous models of such outline,
as e.g. considering document collection and obtaining a final report in the
start and end respectively~\cite{textSurvey}.
Text mining tasks include document summarization, sentiment analysis
and natural language processing techniques such as part of speech tagging~\cite{ntlk}.
Among application are social media monitoring, automated ad placement,
publishing and making tools for semantics, sentiment and general natural language~\cite{textSurvey}.
It is believed that applying text mining to social media
can yield interesting findings in human behavior~\cite{customText}.
Although there is no clear cut, text mining is sometimes divided into linguistic and non-linguistic~\cite{customText}.
In the first case, linguistic techniques are present, such as
the analysis of discourse and part of speech tagging,
and it is often mingled with natural language processing or computational linguistics (see Section~\ref{amb} for a coherent distinction of the fields).
In the non-linguistic text mining, text is analyzed by means of statistical features
derived from e.g. the size of tokens and sentences,
and might be more easily related to the intuitive concept of data mining of text.
On this thesis we use both perspectives.

\subsection{Visualization of static and dynamic graphs}
Static graph visualization is achieved in many ways,
most usually through the node-link (often called network diagram)
and matrix representations as illustrated in Figure~\ref{nlmatrix}.
Representing graphs as node-link diagrams has a long tradition which remotes
at least to the works of Ramon Llull in the 13th century~\cite{llull}.
To glimpse at the theory involved in visualizing networks~\cite{eades}, we mention three aspects:
\begin{itemize}
	\item criteria for the quality of layouts include the number of crossing edges, the area of the drawing relative to closest distance between two vertices.
	\item Layout methods are derived e.g. by placing vertices in a circular fashion, by using the eigen vectors from a worked out variant of the adjacency matrix as coordinates, or by force-based methods. For large graphs, including a number of social networks, the force-based networks are reported as useful. Therefore, we illustrate this method with the simplest model we could find in the well known literature. Be $f_a$ the attraction force, $f_r$ the repulsion force, $d$ the distance between the vertices and $k$ a constant. The model introduced by Fruchterman and Reingold~\cite{fr} defines the forces as:
\begin{align}
	f_a = \frac{d^2}{k}\\
	f_d = -\frac{k^2}{d}
\end{align}
	\item Graph drawings are often developed for specific applications e.g. in biology (e.g. protein and gene interactions), social networks, tree diagrams.
\end{itemize}

The core difference of dynamic graphs to static graphs is that vertices and edges
can be added and removed over time.
If we define the static graph G as $G:=(V,E)$ where V are the vertices as E are the edges in G,
a dynamic graph might be defined as $\Gamma:=(G_1,G_2,...,G_n)$ where $G_i:=(V_i,E_i)$
are static graphs and indices refer to a sequence of time steps $(t_1,t_2,...,t_n)$.
In dynamic graph visualization most usually graphs are represented as animated diagrams
or charts based on a timeline~\cite{dynGraph}.
In this thesis we make use of node-link diagrams of both static and dynamic graphs.

\subsection{Linked (open) data}
Linked data refers to data published in the web in such a way that it is
machine readable and complies with a set of best practices.
The web of data is constructed with documents on the web 
such as the web of HTML documents.
In practice, the idea of linked data can me summarized
by 1) the use of RDF to publish data on the web and 2) the use of RDF
links to interlink data from different sources.
The web is expected to be interconnected and to grow by the systematic application of four
steps~\cite{lee1}:
\begin{itemize}
	\item Use URIs to identify things~\cite{uri}.
	\item Use HTTP URIs.
	\item Provide useful information when an URI is accessed via HTTP.
	\item Provide other URIs in the description of resources so human
		and machine agents can perform discovery.
\end{itemize}

The Linked Open Data~\cite{lod} builds an ever growing cloud of data,
the global data space, which is usually
conceived as centered around the DBPedia, a linked data representation
of data from Wikipedia~\cite{dbpedia0,dbpedia}.

The fields of social network analysis and complex networks
are widely researched.
However, there is a lack of open datasets for benchmarking results,
especially associated with the complex networks field,
yielding diverse results from poorly related sources.
Recently, a myriad of results have been reported which are based in
diverse datasets most often not accessible to researchers other than the publishing authors.
In this thesis we present resources for having open databases to provide 
the scientific community with a friendly and common repertoire.
We chose to use the linked data technology and follow W3C best practices for publishing data.

\subsubsection{RDF}
The Resource Description Framework (RDF), a W3C
recommendation, is a model for data
interchange.
It is based on the idea of making statements about resources in the form
of triples, i.e. expressions in the form ``subject - predicate -
object''.
RDF can be serialized in several file formats, including RDF/XML,
Turtle and Manchester, all of which, in essence, represent a labeled and
directed multi-graph.
RDF may be stored in a type of database referred to as a triplestore~\cite{rdf}.

As an example of a RDF statement, the following triple in the Turtle
format asserts that ``the paper has color white'':\\
\texttt{http://example.org/Things\#Paper http://example.org/hasColor\\
http://example.org/Colors\#White .}
% This work makes available an open database with diverse provenance and
% to furnish the scientific community with a friendly and common repertoire.

Integration and uniformity of access is obtained through linked data
representation, as explained in Section~\ref{queries}.

\subsection{Social participation}
A significant share of our endeavor was oriented towards social participation,
i.e. to facilitate civil engagement in a community, most significantly in State affairs.
More concretely, we published data from a social participation federal portal~\cite{participaData},
applied complex networks and data mining criteria for resources recommendation~\cite{participaRecommend} and
proposed a ranking algorithm for voted proposals in another federal participation portal~\cite{dialogaAlgorithm}.
Such works were performed within a United Nations Development Program consulting contract,
in partnership with the Brazilian Presidency of the Republic and published publicly mainly as technical reports~\cite{participa3,participa4,participa5}.
This aspect of our research was important for maturing topics and understanding
the extent to which they are applied in pragmatic contexts
and is left mostly to auxiliary documents of this thesis to promote simple expositions.

\subsection{Other}
Given the multidisciplinary condition of our work and of the implied topics,
many other fields of knowledge could be further explored in this introduction or
the methods chapter.
To name just a few of the most directly related fields: statistics, principal component analysis,
big data, social network analysis, social media mining, mathematical sociology,
datasets, free culture, open source software, computer programming.

Of particular relevance are the typologies for human personality, such as the ones derived
from the Myers-Briggs type indicator~\cite{mb} and from authoritarian personality~\cite{ap} theories,
because we present a new typology of human participants in social networks in Section~\ref{sec:pty}.
Another topic we should highlight is what we called ``anthropological physics''~\cite{anPhy,ccsSpeach}:
the observation of natural/physical laws in human social systems.
The term should not be confused with physical anthropology, which is a synonym for biological anthropology,
a subfield of anthropology concerned with the evolution of humans~\cite{phyAn}.

\section{Polysemy and synonyms}
In the context of complex networks, the words \emph{network} and \emph{graph}
are often used interchangeably, although the word graph might refer to the
mathematical structure of vertices and edges and the word network might refer to the
real system being represented as a graph or to the graph obtained by means of representing a real system.
Furthermore, the word graph can be used to refer to a \emph{graph of a function} (mathematics) or to an abstract datatype (computer science).
This parallelism between network and graphs also apply to network visualization and graph visualization.
One might add here the term \emph{graph drawing} another synonym for the visualization of graphs, although the term
seem to be more traditional in relation to the achievement of node-edge network diagrams.
Evolutionary graph or network visualization is an example variant of dynamic graph visualization.
The nomenclature of vertices and edges vary widely among interested fields (mathematics, physics, biology, sociology, etc).
A vertex might be called e.g. a node, a point, an agent, a actor, a participant.
An edge might be called e.g. a link, a bond, a relation, a tie, a connection.

% There is also some ambiguity related to measures that can be take with respect to each vertex (or, less often, to each edge),
% such as degree and clustering.
% Such terms might be also used for the average of the measure for the whole network.

The terms \emph{text mining}, \emph{natural language processing} and \emph{computational linguistics}
are often used for similar endeavors.
A distinction might be made in that text mining refers to data mining of text,
natural language processing is concerned with the interactions between computer and human natural languages,
and computational linguistics aims for statistical or rule-based modeling of natural language from a computational perspective.
Such fields are multidisciplinary and there is no sharp distinction between them.

Examined as fields of knowledge, the \emph{linked data} and the \emph{semantic web}
terms are often used without distinction.
Tim Berners-Lee coined both terms: 
the semantic web was conceived as a web of data that can be processed by machines~\cite{lee0},
the expression linked data appeared in a 2006 design note about the Semantic Web project~\cite{lee1}
and refers to structured data that can emphasizes interlinking and usefulness through semantic queries.

\emph{Social participation}, \emph{social envolvement} and \emph{social engagement} are synonyms that
refer to the participation of an individual or group in a community or society.
In Brazilian Portuguese, \emph{controle social} can refer to the antagonist concepts of social participation
or of a social control (played by the State or companies in the civil society).

\subsection{More specific terminology problems in the complex networks field}
Given that this thesis involves multidisciplinary and new knowledge,
it might be of no surprise that the nomenclature is not very well defined.
Here we pin-point some more specific conflicts that arise in the literature of complex networks
to both exemplify this issue and to avoid some problems in interpreting the
methods and results in this thesis:
\begin{itemize}
	\item The \emph{hubs} are, by the usual definition, the more connected vertices.
		In the context of the HITS (Hyperlink-Induced Topic Search) algorithm,
		for attributing centrality to vertices, most traditionally to web pages,
		the hubs are the vertices with greater out-degree
		(greater in-degree yield \emph{authorities}).
	\item In some contexts, the center of network is the collection of vertices whose the
		maximum distance to other vertices is the radius (i.e. the minimum maximum difference between vertices). 
		In the same framework, the periphery of a network is the collection of vertices whose
		the maximum distance to other vertices is the diameter (i.e. the maximum distance between vertices).
		By extension, the intermediary might be regarded as the set of vertices that are not in the center or the periphery.
		These definitions yield fractions of members that do not agree with the literature with respect to hubs, intermediary and periphery.
		We present a suitable method for deriving such memberships, in that it fits the literature prediction, in Section~\label{sectioning}.
	\item Lace, loop, selfloop and autoloop are terms used to designate an edge from a vertex to itself.
\end{itemize}

\section{A historical note}
The knowledge fields involved in this thesis are very recent.
To point just the main areas,
complex networks has emerged in the final years of the 1990s decade~\cite{newmanBook};
text mining first workshops were held in 1999~\cite{textMining};
as an independent field, graph drawing arose in the 1990s~\cite{dynGraph};
the term linked data was coined in 2006~\cite{lee1}.
% fields are recent, point starting works and/or dates for the fields.

\section{Considerations about the presented work}
The initial idea of this research was to enable
the use of complex and social networks scientific knowledge by the participant.
This motivated the open software and texts produced, and the endeavors with
the United Nations, Brazilian Presidency and civil parties.
As this was a practical goal, we found by hands-on processes that
many fields are related to the subject, which reflected in the number of fields tackled in the thesis
and related documents~\cite{pnud3,pnud4,pnud5,anPhy,dialoga,losd,versinus,kolmSmir}.
Furthermore, we understand that the open software, texts, videos and processes proided
by our work contributes for the popularization of the knowledge and technologies implied
by the empowerment of civil individuals and groups through the management of the
networks in which they exist.

\section{Structure of the thesis}
% canonical chapters; stability through circular statistics and topological statistics (PCA, sectioning)
% text mining through measures of sizes of tokens, sentences, messages
% through kolmogorov-smirnov, POS tagging, PCA
% visualization of network evolution through animations (and other gadgets)
% linked data representation of data and ontological organization
% Appendices: supporting tables and diagrams, related results:
% kolmogorov-smirnov, ubiquity of inequality through power laws, list of UNDP products, list of software

% ---

% ---
% Capítulo 2
% ---
%% USPSC-Cap2-Desenvolvimento.tex 

% ---
% Este capítulo, utilizado por diferentes exemplos do abnTeX2, ilustra o uso de
% comandos do abnTeX2 e de LaTeX.
% ---

\chapter{Desenvolvimento}\label{cap_exemplos}
Este capítulo é parte principal da dissertação ou tese e deve conter a exposição ordenada e detalhada do assunto. Divide-se em seções e subseções, em conformidade com a abordagem do tema e do método, abrangendo: revisão bibliográfica, materiais e métodos, técnicas utilizadas, resultados obtidos e discussão.

O conteúdo deste documento visa apresentar um tutorial para utilização da Classe USPSC e seus modelos, utilizando a estrutura de trabalhos acadêmicos, mas por questões didáticas adotou-se capítulo, seções e subseções diferentes das usualmente utilizadas.


\section{Classe USPSC e modelo de trabalho de acadêmico}
A classe USPSC é uma derivada da \textbf{\textsf{abntex2}.cls, v-1.9.5} para as Unidades de Ensino e Pesquisa do Campus USP de São Carlos:
Escola de Engenharia de São Carlos (EESC), Instituto de Arquitetura e Urbanismo (IAU), Instituto de Ciências Matemáticas e de Computação (ICMC), Instituto de Física de São Carlos (IFSC) e Instituto de Química de São Carlos (IQSC).

O objetivo do projeto é disponibilizar um modelo em \LaTeX\  para a elaboração de trabalhos acadêmicos (tese, dissertação, trabalho de conclusão de curso (TCC), dentre outros) em conformidade com a \textbf{ABNT NBR 14724}: informação e documentação: trabalhos acadêmicos: apresentação \cite{nbr14724}, \textbf{Diretrizes para apresentação de dissertações e teses da USP}: documento eletrônico e impresso - Parte I (ABNT) \cite{sibi2009} e normas e padrões estabelecidos pelas Unidades.

Este documento e seu código fonte são exemplos de uso da classe USPSC e do pacote \textsf{abntex2cite}.
Para complementar as instruções contidas neste documento, utilize os manuais \cite{abnetxclasse,abnetxcite,abnetxcitealf} e da classe \textsf{memoir}\cite{memoir2010}. 

O modelo segue a estrutura de trabalhos acadêmicos estabelecida pela ABNT NBR 14724, conforme a \autoref{fig_EstruturaTrabAcad}, e consiste no arquivo USPSC-modelo.tex que utiliza os seguintes arquivos para gerar o documento mediante a compilação utilizando um dos editores \LaTeX\ :
\begin{alineas}	 
				\item USPSC.cls ou USPSC1.cls (classe USPSC); 
				\item USPSC-modelo-references.bib;
				\item USPSC-modelo.tex;
				\item USPSC-unidades.tex;
				\item USPSC-pre-textual-UUUU.tex;
				\item USPSC-fichacatalografica.tex;
				\item fichacatalografica.pdf;
				\item folhadeaprovacao.pdf;
				\item USPSC-Cap1-Introducao.tex;
				\item USPSC-Cap2-Desenvolvimento.tex;
				\item USPSC-Cap3-Citacoes.tex;
				\item USPSC-Cap4-Referencias.tex;
				\item USPSC-Cap5-Conclusao.tex;
				\item USPSC-Apendices.tex;
				\item USPSC-Anexos.tex;
				\item USPSC-AcentuacaoLaTeX.tex
				\item USPSC-LetrasGregas.tex
				\item USPSC-SimbolosUteis.tex
\end{alineas}	 

\begin{figure}[htb]
	\caption{\label{fig_EstruturaTrabAcad}Estrutura do trabalho acadêmico}
	\begin{center}
		\includegraphics[scale=0.5]{USPSC-EstruturaTrabAcad.jpg}
	\end{center}
	\legend{Fonte: \citeonline{nbr14724}}
\end{figure}

				
			 No arquivo USPSC-modelo.tex o autor deverá indicar a sigla da Unidade e a sigla do programa de pós-graduação que está vinculado, a exemplo dos comandos abaixo:
		
			\begin{verbatim}
				\siglaunidade{IQSC}
				\programa{MQOB}
			\end{verbatim}

			Para tanto deverá consultar as siglas estabelecidas para os programas de pós-graduação de cada Unidade \textbf{(APÊNDICES A-E)} ou na planilha \textbf{USPSC-Siglas estabelecidas para os programas de pós-graduação por Unidade.xlsx}, para utilizar um dos arquivos com dados pre-textuais nominados como USPSC-pre-textual-UUUU.tex, onde UUUU é a sigla da Unidade. Inicialmente estão disponibilizados os pré-textuais das Unidades do Campus USP de São Carlos:
			
\begin{alineas}	 
				\item USPSC-pre-textual-EESC.tex;
				\item USPSC-pre-textual-IAU.tex;
				\item USPSC-pre-textual-ICMC.tex;
				\item USPSC-pre-textual-IFSC.tex;
				\item USPSC-pre-textual-IQSC.tex.
\end{alineas}


Foi definido o arquivo USPSC-pre-textual-OUTRO.tex que será executado para siglas de Unidades diferentes das explicitadas acima, visando principalmente alertar o autor para rever a indicação feita no comando \textbf{siglaunidade}.

Quando for indicado uma sigla não prevista de \textbf{programa}, para qualquer um dos arquivos pré-textuais já definidos, o preâmbulo será iniciado por "Dissertação/Tese", mostrando que o autor deverá rever a sigla e utilizar a correta para o programa que está vinculado.

Através do comando \verb+  %% USPSC-unidades.tex
% Camando para definição do programa de Pós-Graduação, Especialidade do Título e Instituição - 21/09/2015
\newcommand{\siglaunidade}[1]{

% EESC ==========================================================================
    \ifthenelse{\equal{#1}{EESC}}{
     			\include{USPSC-pre-textual-EESC}
% ---
    }{
% IAU ===========================================================================
        \ifthenelse{\equal{#1}{IAU}}{
        \include{USPSC-pre-textual-IAU}    
        }{
% ICMC ===========================================================================
        \ifthenelse{\equal{#1}{ICMC}}{
        \include{USPSC-pre-textual-ICMC}    
        }{
% IFSC ===========================================================================
        \ifthenelse{\equal{#1}{IFSC}}{
        %% USPSC-pre-textual-IFSC.tex
%% Camandos para defini��o do tipo de documento (tese ou disserta��o), �rea de concentra��o, op��o, pre�mbulo, titula��o 
%% referentes ao Programa de P�s-Gradua��o o IFSC
\instituicao{Instituto de F\'isica de S\~ao Carlos, Universidade de S\~ao Paulo}
\unidade{INSTITUTO DE F\'ISICA DE S\~AO CARLOS}
\unidademin{Instituto de F\'isica de S\~ao Carlos}
\universidademin{Universidade de S\~ao Paulo}
%\notafolharosto{Vers\~ao original}
%Para vers�o original em ingl�s, comente do comando/declara��o 
%     acima(inclua % antes do comando acima) e tire a % do 
%     comando/declara��o abaixo no idioma do texto
\notafolharosto{Original version} 
%Para vers�o corrigida, comente do comando/declara��o da 
%     vers�o original acima (inclua % antes do comando acima) 
%     e tire a % do comando/declara��o de um dos comandos 
%     abaixo em conformidade com o idioma do texto
%\notafolharosto{Vers\~ao corrigida \\(Vers\~ao original dispon\'ivel na Unidade que aloja o Programa)}
%\notafolharosto{Corrected version \\(Original version available on the Program Unit)}

% ---
% dados complementares para CAPA e FOLHA DE ROSTO
% ---
\universidade{UNIVERSIDADE DE S\~AO PAULO}
%\titulo{Modelo para teses e disserta\c{c}\~oes em \LaTeX\ utilizando a classe USPSC para o IFSC}
%\titleabstract{Model for theses and dissertations in \LaTeX\ using the USPSC class to the IFSC}
\titulo{Topological stability and textual differentiation in human interaction networks: linked data and statistical analysis}
\titleabstract{Topological stability and textual differentiation in human interaction networks: linked data and statistical analysis}
% \autor{Jos\'e da Silva}
% \autorficha{Silva, Jos\'e da}
% \autorabr{SILVA, J.}
\autor{Renato Fabbri}
\autorficha{Fabbri, Renato}
\autorabr{FABBRI, R.}

\cutter{S856m}
% Para gerar a ficha catalogr�fica sem o C�digo Cutter, basta 
% incluir uma % na linha acima e tirar a % da linha abaixo
%\cutter{ }

\local{S\~ao Carlos}
\data{2016}
% Quando for Orientador, basta incluir uma % antes do comando abaixo
%\renewcommand{\orientadorname}{Orientadora:}
% Quando for Coorientadora, basta tirar a % utilizar o comando abaixo
%\newcommand{\coorientadorname}{Coorientador:}
% \orientador{Profa. Dra. Elisa Gon\c{c}alves Rodrigues}
% \orientadorcorpoficha{orientadora Elisa Gon\c{c}alves Rodrigues}
% \orientadorficha{Rodrigues, Elisa Gon\c{c}alves, orient}
\orientador{Prof. Dr. Osvaldo Novais de Oliveira Junior}
\orientadorcorpoficha{orientador Osvaldo Novais de Oliveira Junior}
\orientadorficha{Oliveira Junior, Osvaldo Novais de}
%Se houver co-orientador, inclua % antes das duas linhas (antes dos comandos \orientadorcorpoficha e \orientadorficha) 
%          e tire a % antes dos 3 comandos abaixo
%\coorientador{Prof. Dr. Jo\~ao Alves Serqueira}
%\orientadorcorpoficha{orientadora Elisa Gon\c{c}alves Rodrigues ;  co-orientador Jo\~ao Alves Serqueira}
%\orientadorficha{Rodrigues, Elisa Gon\c{c}alves, orient. II. Serqueira, Jo\~ao Alves, co-orient}

\notaautorizacao{AUTORIZO A REPRODU\c{C}\~AO E DIVULGA\c{C}\~AO TOTAL OU PARCIAL DESTE TRABALHO, POR QUALQUER MEIO CONVENCIONAL OU ELETR\^ONICO PARA FINS DE ESTUDO E PESQUISA, DESDE QUE CITADA A FONTE.}
\notabib{Ficha catalogr\'afica revisada pelo Servi\c{c}o de Biblioteca e Informa\c{c}\~ao Prof. Bernhard Gross, com os dados fornecidos pelo(a) autor(a)}

\newcommand{\programa}[1]{

% DFA ==========================================================================
    \ifthenelse{\equal{#1}{DFA}}{
     	\tipotrabalho{Tese (Doutorado em Ci\^encias)}
        \area{F\'isica Aplicada}
		%\opcao{Nome da Op��o}
        % O preambulo deve conter o tipo do trabalho, o objetivo, 
		% o nome da institui��o, a �rea de concentra��o e op��o quando houver
		\preambulo{Tese apresentada ao Programa de P\'os-Gradua\c{c}\~ao em F\'isica do Instituto de F\'isica de S\~ao Carlos da Universidade de S\~ao Paulo, para obten\c{c}\~ao do t\'itulo de Doutor em Ci\^encias.}
		\notaficha{Tese (Doutorado - Programa de P\'os-Gradua\c{c}\~ao em F\'isica Aplicada)}
    }{
% MFA ===========================================================================
        \ifthenelse{\equal{#1}{MFA}}{
	        \tipotrabalho{Disserta\c{c}\~ao (Mestrado em Ci\^encias)}
	        \area{F\'isica Aplicada}
			%\opcao{Nome da Op��o}
	        % O preambulo deve conter o tipo do trabalho, o objetivo, 
			% o nome da institui��o, a �rea de concentra��o e op��o quando houver
			\preambulo{Disserta\c{c}\~ao apresentada ao Programa de P\'os-Gradua\c{c}\~ao em F\'isica do Instituto de F\'isica de S\~ao Carlos da Universidade de S\~ao Paulo, para obten\c{c}\~ao do t\'itulo de Mestre em Ci\^encias.}
			\notaficha{Disserta\c{c}\~ao (Mestrado - Programa de P\'os-Gradua\c{c}\~ao em F\'isica Aplicada)}
        }{
% DFAFC ==========================================================================
    \ifthenelse{\equal{#1}{DFAFC}}{
     	\tipotrabalho{Tese (Doutorado em Ci\^encias)}
        \area{F\'isica Aplicada}
        \opcao{F\'isica Computacional}
        % O preambulo deve conter o tipo do trabalho, o objetivo, 
		% o nome da institui��o, a �rea de concentra��o e op��o quando houver
		\preambulo{Tese apresentada ao Programa de P\'os-Gradua\c{c}\~ao em F\'isica do Instituto de F\'isica de S\~ao Carlos da Universidade de S\~ao Paulo, para obten\c{c}\~ao do t\'itulo de Doutor em Ci\^encias.}
		\notaficha{Tese (Doutorado - Programa de P\'os-Gradua\c{c}\~ao em F\'isica Aplicada)}
    }{
% DFAFCe
        \ifthenelse{\equal{#1}{DFAFCe}}{
		\tipotrabalho{Thesis (Doctor in Science)}
	        \area{Applied Physics}
			%\opcao{Nome da Op��o}
        \opcao{Computational Physics}
	        % O preambulo deve conter o tipo do trabalho, o objetivo, 
			% o nome da institui��o, a �rea de concentra��o e op��o quando houver
			\preambulo{Thesis presented to the Graduate Program in Physics at the Instituto de F\'isica de S\~ao Carlos, Universidade de S\~ao Paulo, to obtain the degree of Doctor in Science.}
			\notaficha{Thesis (Doctorate - Graduate Program in Applied Physics)}
        }{
% MFAFC ===========================================================================
        \ifthenelse{\equal{#1}{MFAFC}}{
			\tipotrabalho{Disserta\c{c}\~ao (Mestrado em Ci\^encias)}
	        \area{F\'isica Aplicada}
	        \opcao{F\'isica Computacional}
	        % O preambulo deve conter o tipo do trabalho, o objetivo, 
			% o nome da institui��o, a �rea de concentra��o e op��o quando houver
			\preambulo{Disserta\c{c}\~ao apresentada ao Programa de P\'os-Gradua\c{c}\~ao em F\'isica do Instituto de F\'isica de S\~ao Carlos da Universidade de S\~ao Paulo, para obten\c{c}\~ao do t\'itulo de Mestre em Ci\^encias.}
			\notaficha{Disserta\c{c}\~ao (Mestrado - Programa de P\'os-Gradua\c{c}\~ao em F\'isica Aplicada)}
        }{
% DFAFBp ===========================================================================
        \ifthenelse{\equal{#1}{DFAFBp}}{
			\tipotrabalho{Tese (Doutorado em Ci\^encias)}
	        \area{F\'isica Aplicada}
	        \opcao{F\'isica Biomolecular}
	        % O preambulo deve conter o tipo do trabalho, o objetivo, 
			% o nome da institui��o, a �rea de concentra��o e op��o quando houver
			\preambulo{Tese apresentada ao Programa de P\'os-Gradua\c{c}\~ao em F\'isica do Instituto de F\'isica de S\~ao Carlos da Universidade de S\~ao Paulo, para obten\c{c}\~ao do t\'itulo de Doutor em Ci\^encias.}
			\notaficha{Tese (Doutorado - Programa de P\'os-Gradua\c{c}\~ao em F\'isica Aplicada)}
        }{				
% DFAFBe ===========================================================================
        \ifthenelse{\equal{#1}{DFAFBe}}{
			\renewcommand{\areaname}{Concentration area:}
			\renewcommand{\opcaoname}{Option:}
			\tipotrabalho{Thesis (Doctor in Science)}
	        \area{Applied Physics}
	        \opcao{Biomolecular Physics}
	        % O preambulo deve conter o tipo do trabalho, o objetivo, 
			% o nome da institui��o, a �rea de concentra��o e op��o quando houver
			\preambulo{Thesis presented to the Graduate Program in Physics at the Instituto de F\'isica de S\~ao Carlos, Universidade de S\~ao Paulo, to obtain the degree of Doctor in Science.}
			\notaficha{Thesis (Doctorate - Graduate Program in Applied Physics)}
        }{				
% MFAFB ===========================================================================
        \ifthenelse{\equal{#1}{MFAFB}}{
	        \tipotrabalho{Disserta\c{c}\~ao (Mestrado em Ci\^encias)}
	        \area{F\'isica Aplicada}
	        \opcao{F\'isica Biomolecular}
	        % O preambulo deve conter o tipo do trabalho, o objetivo, 
			% o nome da institui��o, a �rea de concentra��o e op��o quando houver
			\preambulo{Disserta\c{c}\~ao apresentada ao Programa de P\'os-Gradua\c{c}\~ao em F\'isica do Instituto de F\'isica de S\~ao Carlos da Universidade de S\~ao Paulo, para obten\c{c}\~ao do t\'itulo de Mestre em Ci\^encias.}
			\notaficha{Disserta\c{c}\~ao (Mestrado - Programa de P\'os-Gradua\c{c}\~ao em F\'isica Aplicada)}
        }{
				
% DFB ==========================================================================
    \ifthenelse{\equal{#1}{DFB}}{
     	\tipotrabalho{Tese (Doutorado em Ci\^encias)}
        \area{F\'isica B\'asica}
		%\opcao{Nome da Op��o}
        % O preambulo deve conter o tipo do trabalho, o objetivo, 
		% o nome da institui��o, a �rea de concentra��o e op��o quando houver				
		\preambulo{Tese apresentada ao Programa de P\'os-Gradua\c{c}\~ao em F\'isica do Instituto de F\'isica de S\~ao Carlos da Universidade de S\~ao Paulo, para obten\c{c}\~ao do t\'itulo de Doutor em Ci\^encias.}
		\notaficha{Tese (Doutorado - Programa de P\'os-Gradua\c{c}\~ao em F\'isica B\'asica)}
    }{
% MFB ===========================================================================
        \ifthenelse{\equal{#1}{MFB}}{
	        \tipotrabalho{Disserta\c{c}\~ao (Mestrado em Ci\^encias)}
	        \area{F\'isica B\'asica}
			%\opcao{Nome da Op��o}
	        % O preambulo deve conter o tipo do trabalho, o objetivo, 
			% o nome da institui��o, a �rea de concentra��o e op��o quando houver				
			\preambulo{Disserta\c{c}\~ao apresentada ao Programa de P\'os-Gradua\c{c}\~ao em F\'isica do Instituto de F\'isica de S\~ao Carlos da Universidade de S\~ao Paulo, para obten\c{c}\~ao do t\'itulo de Mestre em Ci\^encias.}
			\notaficha{Disserta\c{c}\~ao (Mestrado - Programa de P\'os-Gradua\c{c}\~ao em F\'isica B\'asica)}
        }{                
% Outros
				\tipotrabalho{Disserta\c{c}\~ao/Tese (Mestrado/Doutorado)}
				\area{Nome da \'Area}
				\opcao{Nome da Op\c{c}\~ao}
		        % O preambulo deve conter o tipo do trabalho, o objetivo, 
				% o nome da institui��o, a �rea de concentra��o e op��o quando houver
				\preambulo{Disserta\c{c}\~ao/Tese apresentada ao Programa de P\�{\o}s-Gradua\c{c}\~ao em F\�{\i}sica do Instituto de F\�{\i}sica de S\~ao Carlos da Universidade de S\~ao Paulo, para obten�\c{c}\~ao do t\�{\i}tulo de Mestre/Doutor em Ci\^encias.}
				\notaficha{Disserta\c{c}\~ao/Tese (Mestrado/Doutorado - Programa de P\'os-Gradua\c{c}\~ao em Nome da \'Area)}
	}}}}}}}}}}}
				
				






    
        }{
% IQSC ===========================================================================
        \ifthenelse{\equal{#1}{IQSC}}{
        \include{USPSC-pre-textual-IQSC}    
        }{

                   
% Outros ========================================================================
                       % ------------------------------------------------------------------------
        \include{USPSC-pre-textual-OUTRO} 
								  }
                }
							}
						}
				}
			}
         
    + presente em USPSC.cls e USPSC1.cls,  o arquivo USPSC-unidades.tex efetua as chamadas dos arquivos pré-textuais, portanto quando for feita uma customização incluindo novos arquivos pré-textuais e/ou outra Unidade USP e/ou outra instituição de ensino e pesquisa, será necessário fazer as devidas indicações em tais arquivos. 
	 
\subsection{Alternativas de formatação}
O modelo foi concebido de forma a atender as especificidades de cada Unidade e atualmente disponibiliza as seguintes alternativas de formatação:
\subsubsection{Opções de fonte} 
No arquivo USPSC-modelo.tex é possível optar pela fonte desejada, conforme a programação abaixo reproduzida:
				\begin{verbatim}
				\usepackage{lmodern}			% Usa a fonte Latin Modern
					% Para utilizar a fonte Times New Roman, inclua 
					% uma % no início do comando acima  "\usepackage{lmodern}"
					% Abaixo, tire a % antes do comando  \usepackage{times}
					%\usepackage{times}			% Usa a fonte Times New Roman
					% Lembre-se de alterar a fonte no comando que imprime 
					% o preâmbulo no arquivo da Classe USPSC.cls					
				\end{verbatim}
\subsubsection{Impressão anverso e verso ou somente anverso}
No arquivo USPSC-modelo.tex é possível optar por impressão em páginas ou em folhas, conforme a seguinte programação:
			  \begin{verbatim}
			  twoside,  % para impressão em anverso (frente) e verso. Oposto 
			            a oneside - Nota: utilizar \imprimirfolhaderosto*
			  %oneside, % para impressão em páginas separadas (somente 
			            anverso) -  Nota: utilizar \imprimirfolhaderosto
			            % inclua uma % antes do comando twoside e exclua a % 
			            antes do oneside 
			  \end{verbatim}			  
\subsubsection{Opção de p. ou f. na referência da Errata, do Resumo e do Abstract} 
 No arquivo USPSC-modelo.tex, indicar p. ou f. em conformidade com a opção de impressão anverso e verso ou somente anverso, conforme a seguinte programação:
			  \begin{verbatim}
			  \pageref{LastPage}p. 
			  %Substitua p. por f. quando utilizar oneside em \documentclass
			  %\pageref{LastPage}f.
			  \end{verbatim}			  
\subsubsection{Tipos de cabeçalhos de páginas} 
No arquivo USPSC-modelo.tex é possível optar por dois tipos de cabeçalhos em conformidade com o definido abaixo:
			  \begin{verbatim}
			  \documentclass[
			  ...
			  % {USPSC} configura o cabeçalho contendo apenas o número da página
			  ]{USPSC}
			  %]{USPSC1}
			  % Inclua % antes de ]{USPSC} e retire a % antes de %]{USPSC1}
			  % para utilizar o cabeçalho diferenciado para as páginas pares 
			    e ímpares como indicado abaixo:
			  %- páginas ímpares: cabeçalho com a seções ou subseções e o número 
			     da página
			  %- páginas pares: cabeçalho com o o número da página e o título do 
			     capítulo 
			  \end{verbatim}
\subsubsection{Opções de idiomas do texto} 
No arquivo USPSC-modelo.tex há duas opções de idiomas do texto: português ou inglês, conforme programação abaixo:			  
			  \begin{verbatim}
			  % Seleciona o idioma do documento (conforme pacotes do babel)
			  \selectlanguage{brazil}
			  % Se o idioma do texto for inglês, inclua uma % antes do 
			  %      comando \selectlanguage{brazil} e 
			  %      retire a % antes do comando abaixo
			  %\selectlanguage{english}			  
			  \end{verbatim}
\subsubsection{Utilização de pacotes para a indicação de número de autores nas referências e para citações alfabéticas ou numéricas}
É possível indicar todos os autores nas referências ou utilizar \textbf{et al} quando houver mais de três autores. Como somente o IQSC indica todos os autores, adotamos o \textbf{et al} e incluímos a orientação de como proceder para alterar a programação no arquivo USPSC-modelo.tex para indicar todos.

Outra possibilidade é de optar por citações alfabéticas ou numéricas, conforme a seguinte orientação contida em USPSC-modelo.tex:	
		  
			 \begin{verbatim}
			 % ---
			 % Pacotes de citações
			 % Citações padrão ABNT
			 % ---
			 % Sistemas de chamada: autor-data ou numérico.
			 % Sistema autor-data
			 \usepackage[alf,abnt-emphasize=bf, abnt-thesis-year=both,
			 abnt-repeated-author-omit=yes, abnt-last-names=abnt,
			 abnt-etal-cite,abnt-etal-list=3, abnt-etal-text=default, 
			 abnt-and-type=e, abnt-doi=doi, abnt-url-package=none,
			 abnt-verbatim-entry=no]{abntex2cite}
			 
			 % Para o IQSC, que indica todos os autores nas referências, incluir % no 
			 início do comando acima e retirar a % do comando abaixo 
			 
			 %\usepackage[alf,abnt-emphasize=bf, abnt-thesis-year=both,
			 abnt-repeated-author-omit=yes, abnt-last-names=abnt,
			 abnt-etal-cite,abnt-etal-list=0, abnt-etal-text=default,
			 abnt-and-type=e]{abntex2cite}
			 
			 % Sistema Numérico
			 %Para citações numéricas, sistema adotado pelo IFSC, incluir % no início 
			 do comando acima e retirar a % do comando abaixo 
			 %\usepackage[num,overcite,abnt-emphasize=bf, abnt-thesis-year=both,
			 abnt-repeated-author-omit=yes, abnt-last-names=abnt,
			 abnt-etal-cite,abnt-etal-list=0, abnt-etal-text=default,
			 abnt-and-type=e]{abntex2cite}
			  
			  %Complementarmente, verifique as instruções abaixo sobre os Pacotes de 
			  Nota de rodapé
			  % ---
			  % Pacotes de Nota de rodapé
			  % Configurações de nota de rodapé
			  
			  %O presente modelo adota o formato numérico para as notas de rodapés 
			  quando utiliza o sistema de chamada autor-data para citações e 
			  referências. Para utilizar o sistema de chamada numérico para 
			  citações e referências, habilitar um dos comandos abaixo.
			  % Há diversa opções para nota de rodapé no Sistema Numérico.  Para
			  o IFSC, habilitade o comando abaixo.
			  
			  %\renewcommand{\thefootnote}{\fnsymbol{footnote}} %Comando para 
			  inserção de símbolos em nota de rodapé
			  
			  % Outras opções para nota de rodapé no Sistema Numérico:
			  %\renewcommand{\thefootnote}{\alph{footnote}}     %Comando para 
			  inserção de letras minúscula em nota de rodapé
			  %\renewcommand{\thefootnote}{\Alph{footnote}}     %Comando para 
			  inserção de letras maiúscula em nota de rodapé
			  %\renewcommand{\thefootnote}{\roman{footnote}}    %Comando para 
			  inserção de números romanos minúsculos  em nota de rodapé
			  %\renewcommand{\thefootnote}{\Roman{footnote}}    %Comando para 
			  inserção de números romanos minúsculos  em nota de rodapé
			  
			  \renewcommand{\footnotesize}{\small} %Comando para diminuir a fonte 
			  das notas de rodapé	
			  
			  % ---
			  % Pacote para agrupar a citação numérica conusecutiva
			  % Quando for adotado o Sistema Numérico, a exemplo do IFSC, habilite 
			  % o pacote cite abaixo retiruando a porcentagem antes do comando abaixo
			  %\usepackage[superscript]{cite}
			  	
			 \end{verbatim}
			 
Sugerimos que quando for alterada a programação do Sistema autor-data para o numérico e/ou vice-versa, o arquivo original USPSC-modelo.tex seja renomeado, pois durante a compilação são gerados arquivos temporários que podem interferir nas mudanças desejadas.			 
\subsubsection{Possibilidades de preâmbulos}

Inicialmente disponibiliza 76 possibilidades de preâmbulos codificados nos arquivos pré-textuais, em conformidade com as siglas estabelecidas para os programas de pós-graduação das Unidades do Campus USP de São Carlos \textbf{(APÊNDICES B-F)} ou na planilha \textbf{USPSC-Siglas estabelecidas para os programas de pós-graduação por Unidade.xlsx}:
	
				  
			   \begin{alineas}
			   	\item EESC – 43;
				\item IAU – 4;
				\item  ICMC – 14;
				\item  IFSC – 9;
				\item  IQSC - 6;
			  \end{alineas}							
\subsubsection{Versão original ou final/corrigida}
Nos arquivos com os elementos pré-textuais das Unidades é possível especificar a versão do trabalho acadêmico produzido, a exemplo do contido em USPSC-pre-textual-IFSC.tex:	  
			  \begin{verbatim}
			  \notafolharosto{Vers\~ao original}
			  %Para versão original em inglês, comente do comando/declaração 
			  %     acima(inclua % antes do comando acima) e tire a % do 
			  %     comando/declaração abaixo no idioma do texto
			  %\notafolharosto{Original version} 
			  %Para versão corrigida, comente do comando/declaração da 
			  %     versão original acima (inclua % antes do comando acima) 
			  %     e tire a % do comando/declaração de um dos comandos 
			  %     abaixo em conformidade com o idioma do texto
			  %\notafolharosto{Vers\~ao corrigida \\(Vers\~ao original dispon\'ivel na
			  Unidade que aloja o Programa)}
			  %\notafolharosto{Corrected version \\(Original version available on the
			  Program Unit)}
			  \end{verbatim}
			  
\subsubsection{Ficha catalográfica}
É possível elaborar a ficha catalográfica em \LaTeX\ ou incluir a fornecida pela Biblioteca. Para tanto observe a programação contida nos arquivos USPSC-modelo.tex e USPSC-fichacatalografica.tex e/ou gere o arquivo fichacatalografica.pdf.
	  
No arquivo USPSC-modelo.tex faça a sua opção conforme orientações reproduzidas abaixo:

			 \begin{verbatim}
			 % ---
			 % Inserir a ficha catalográfica em pdf
			 % ---
			 % A biblioteca da sua Unidade lhe fornecerá um PDF com a ficha
			 % catalográfica definitiva. 
			 % Quando estiver com o documento, salve-o como PDF no diretório
			 % do seu projeto como fichacatalografica.pdf e iclua o arquivo
			 % utilizando o comando abaixo:
			 %\begin{fichacatalografica}
			 %   \includepdf{fichacatalografica.pdf}
			 %\end{fichacatalografica}
			 % Se você optar por elaborar a ficha catalográfica, deverá 
			 % incluir uma % antes das 3 linhas acima e tirar a % antes
			 % do comando \include{USPSC-fichacatalografica}
			 \include{USPSC-fichacatalografica}
			 % As informações que compõem a ficha catalográfica estão 
			 % definidos no arquivo USPSC-pre-textual-UUUU.tex
			 % ---
			 \end{verbatim} 
			 				
É possível incluir ou não o Código Cutter na ficha catalográfica, conforme a seguinte orientação nos respectivos arquivos pré-textuais:

\begin{verbatim}
\cutter{S856m}
% Para gerar a ficha catalográfica sem o Código Cutter, basta 
% incluir uma % na linha acima e tirar a % da linha abaixo
%\cutter{ } 
\end{verbatim} 

Através do arquivo fichacatalografica.tex é possível elaborar a ficha catalográfica em \LaTeX\ . Caso o trabalho possua co-orientador será necessário seguir as orientações contidas também no arquivo com os elementos pré-textuais.	 


\section{Resultados de comandos}\label{sec-divisoes}

O conteúdo desta seção foi baseado no item \textbf{1 Resultados de comandos} do \textbf{Modelo canônico de trabalho acadêmico com abnTEX2} \cite{equipeabntex2}.

% ---
\subsection{Codificação dos arquivos: UTF8}
% ---

A codificação \texttt{UTF8} deve ser utilizada para todos os arquivos do \abnTeX\ . Utilize a mesma codificação nos documentos que escrever, incluindo nos arquivos de base bibliográficas |.bib|. Para tanto, o arquivo USPSC-modelo.tex deve conter o seguinte pacote:
\begin{verbatim}
\usepackage[utf8]{inputenc}	 % Codificacao do documento (conversão
                               automática dos acentos)
\end{verbatim}

% ---
\subsection{Diferentes idiomas e hifenizações}
\label{sec-hifenizacao}
% ---

Para usar hifenizações de diferentes idiomas, inclua nas opções do documento o
nome dos idiomas que o seu texto contém. Os usuários da Classe USPSC devem utilizar:

\begin{verbatim}
\documentclass[
% -- opções da classe memoir --
12pt,		% tamanho da fonte
openright,	% capítulos começam em pág ímpar (insere página vazia caso 
preciso)
twoside,  % para impressão em anverso (frente) e verso. Oposto a oneside - 
Nota: utilizar \imprimirfolhaderosto*
%oneside, % para impressão em páginas separadas (somente anverso) -  
Nota: utilizar \imprimirfolhaderosto
% inclua uma % antes do comando twoside e exclua a % antes do oneside 
a4paper,			% tamanho do papel. 
% -- opções da classe abntex2 --
chapter=TITLE,		% títulos de capítulos convertidos em letras 
maiúsculas
% -- opções do pacote babel --
english,			% idioma adicional para hifenização
french,				% idioma adicional para hifenização
spanish,			% idioma adicional para hifenização
brazil				% o último idioma é o principal do documento
% {uspsc} configura o cabeçalho contendo apenas o número da página
]{uspsc}
%]{uspsc1}
% Inclua % antes de ]{uspsc} e retire a % antes de %]{uspsc1}
% para utilizar o cabeçalho diferenciado para as páginas pares e ímpares 
como indicado abaixo:
%- páginas ímpares: cabeçalho com seções ou subseções e o número da página
%- páginas pares: cabeçalho com o número da página e o título do capítulo 
% ---
\end{verbatim}

Desta forma o texto deverá ser escrito idioma português-brasileiro (\texttt{brazil}), podendo ter citações em inglês, francês e espanhol.

O idioma português-brasileiro (\texttt{brazil}) é incluído automaticamente pela
classe \textsf{abntex2}. Porém, mesmo assim a opção \texttt{brazil} deve ser
informada como a última opção da classe para que todos os pacotes reconheçam o
idioma. Vale ressaltar que a última opção de idioma é a utilizada por padrão no
documento. 

Portanto, para Classe USPSC, caso deseje escrever um texto em inglês que tenha
citações em espanhol, português e francês, você deverá usar:

\begin{verbatim}
\documentclass[
% -- opções da classe memoir --
12pt,		% tamanho da fonte
openright,	% capítulos começam em pág ímpar (insere página vazia caso 
preciso)
twoside,  % para impressão em anverso (frente) e verso. Oposto a oneside - 
Nota: utilizar \imprimirfolhaderosto*
%oneside, % para impressão em páginas separadas (somente anverso) -  
Nota: utilizar \imprimirfolhaderosto
% inclua uma % antes do comando twoside e exclua a % antes do oneside 
a4paper,			% tamanho do papel. 
% -- opções da classe abntex2 --
chapter=TITLE,		% títulos de capítulos convertidos em letras 
maiúsculas
% -- opções do pacote babel --
spanish,			% idioma adicional para hifenização
french,				% idioma adicional para hifenização
brazil,				% o último idioma é o principal do documento
english 			% idioma adicional para hifenização
% {uspsc} configura o cabeçalho contendo apenas o número da página
]{uspsc}
%]{uspsc1}
% Inclua % antes de ]{uspsc} e retire a % antes de %]{uspsc1}
% para utilizar o cabeçalho diferenciado para as páginas pares e ímpares 
como indicado abaixo:
%- páginas ímpares: cabeçalho com seções ou subseções e o número da página
%- páginas pares: cabeçalho com o número da página e o título do capítulo 
% ---
\end{verbatim}

A lista completa de idiomas suportados, bem como outras opções de hifenização,
estão disponíveis em \citeonline[p.~7-8]{babel2011}.

Exemplo de hifenização em inglês\footnote{Extraído de:
	\url{http://en.wikibooks.org/wiki/LaTeX/Internationalization}}:

\begin{otherlanguage*}{english}
	\textit{Text in English language. This environment switches all language-related
		definitions, like the language specific names for figures, tables etc. to the other
		language. The starred version of this environment typesets the main text
		according to the rules of the other language, but keeps the language specific
		string for ancillary things like figures, in the main language of the document.
		The environment hyphenrules switches only the hyphenation patterns used; it can
		also be used to disallow hyphenation by using the language name
		`nohyphenation'.}
\end{otherlanguage*}

Exemplo de hifenização em francês\footnote{Extraído de:
	\url{http://bigbrowser.blog.lemonde.fr/2013/02/17/tu-ne-tweeteras-point-le-vatican-interdit-aux-cardinaux-de-tweeter-pendant-le-conclave/}}:

\begin{otherlanguage*}{french}
	\textit{Texte en français. Pas question que Twitter ne vienne faire une
		concurrence déloyale à la traditionnelle fumée blanche qui marque l'élection
		d'un nouveau pape. Pour éviter toute fuite précoce, le Vatican a donc pris un
		peu d'avance, et a déjà interdit aux cardinaux qui prendront part au vote
		d'utiliser le réseau social, selon Catholic News Service. Une mesure valable
		surtout pour les neuf cardinaux – sur les 117 du conclave – pratiquants très
		actifs de Twitter, qui auront interdiction pendant toute la période de se
		connecter à leur compte.}
\end{otherlanguage*}

Exemplo de hifenização em espanhol\footnote{Extraído de:
	\url{http://internacional.elpais.com/internacional/2013/02/17/actualidad/1361102009_913423.html}}:

\foreignlanguage{spanish}{\textit{Decenas de miles de personas ovacionan al pontífice en su
		penúltimo ángelus dominical, el primero desde que anunciase su renuncia. El Papa se
		centra en la crítica al materialismo}}.

O idioma geral do texto pode ser alterado como no exemplo seguinte:

\begin{verbatim}
\selectlanguage{english}

\end{verbatim}

Isso altera automaticamente a hifenização e todos os nomes constantes de
referências do documento para o idioma inglês. Consulte o manual da classe para obter orientações adicionais sobre internacionalização de documentos produzidos com \textsf{\abnTeX} \cite{abnetxclasse}.

% ---
\subsection{Enumerações}
% ---

\index{alíneas}\index{subalíneas}\index{incisos}Quando for necessário enumerar
os diversos assuntos de uma seção que não possua título, esta deve ser
subdividida em alíneas \cite[4.2]{nbr6024}:

\begin{alineas}

  \item os diversos assuntos que não possuam título próprio, dentro de uma mesma
  seção, devem ser subdivididos em alíneas; 
  
  \item o texto que antecede as alíneas termina em dois pontos;
  \item as alíneas devem ser indicadas alfabeticamente, em letra minúscula
  seguida de parêntese. Utilizam-se letras dobradas, quando esgotadas as
  letras do alfabeto;

  \item as letras indicativas das alíneas devem apresentar recuo em relação à
  margem esquerda;

  \item o texto da alínea deve começar por letra minúscula e terminar em
  ponto-e-vírgula, exceto a última alínea que termina em ponto final;

  \item o texto da alínea deve terminar em dois pontos, se houver subalínea;

  \item a segunda e as seguintes linhas do texto da alínea começa sob a
  primeira letra do texto da própria alínea;
  
  \item subalíneas \cite{nbr6024} devem ser conforme as alíneas a
  seguir:

  \begin{alineas}
     \item as subalíneas devem começar por travessão seguido de espaço;

     \item as subalíneas devem apresentar recuo em relação à alínea;

     \item o texto da subalínea deve começar por letra minúscula e terminar em
     ponto-e-vírgula. A última subalínea deve terminar em ponto final, se não
     houver alínea subsequente;

     \item a segunda e as seguintes linhas do texto da subalínea começam sob a
     primeira letra do texto da própria subalínea.
  \end{alineas}
  
  \item no \abnTeX\ estão disponíveis os ambientes \texttt{incisos} e
  \texttt{subalineas}, que em suma é o mesmo que se criar outro nível de
  \texttt{alineas}, como nos exemplos à seguir:
  
  \begin{incisos}
    \item \textit{Um novo inciso em itálico};
  \end{incisos}
  
  \item Alínea em \textbf{negrito}:
  
  \begin{subalineas}
    \item \textit{Uma subalínea em itálico};
    \item \underline{\textit{Uma subalínea em itálico e sublinhado}}; 
  \end{subalineas}
  
  \item Última alínea com \emph{ênfase}.
  
\end{alineas}

% ---
\subsection{Espaçamento entre parágrafos e linhas}\label{sec_espacamento}
% ---

\index{espaçamento!dos parágrafos}O tamanho do parágrafo, espaço entre a margem
e o início da frase do parágrafo, é definido por:

\begin{verbatim}
   \setlength{\parindent}{1.3cm}
\end{verbatim}

\index{espaçamento!do primeiro parágrafo}Por padrão, não há espaçamento no
primeiro parágrafo de cada início de divisão do documento
(\autoref{sec-divisoes-b}). Porém, você pode definir que o primeiro parágrafo
também seja indentado, como é o caso deste documento. Para isso, apenas inclua o
pacote \textsf{indentfirst} no preâmbulo do documento:

\begin{verbatim}
   \usepackage{indentfirst} % Indenta o primeiro parágrafo de cada seção.
\end{verbatim}

\index{espaçamento!entre os parágrafos}O espaçamento entre um parágrafo e outro
pode ser controlado por meio do comando:

\begin{verbatim}
  \setlength{\parskip}{0.2cm}  % tente também \onelineskip
\end{verbatim}

\index{espaçamento!entre as linhas}O controle do espaçamento entre linhas é
definido por:
\begin{verbatim}
  \OnehalfSpacing       % espaçamento um e meio (padrão); 
  \DoubleSpacing        % espaçamento duplo
  \SingleSpacing        % espaçamento simples	
\end{verbatim}

Para isso, também estão disponíveis os ambientes:
\begin{verbatim}
  \begin{SingleSpace} ...\end{SingleSpace}
  \begin{Spacing}{hfactori} ... \end{Spacing}
  \begin{OnehalfSpace} ... \end{OnehalfSpace}
  \begin{OnehalfSpace*} ... \end{OnehalfSpace*}
  \begin{DoubleSpace} ... \end{DoubleSpace}
  \begin{DoubleSpace*} ... \end{DoubleSpace*} 
\end{verbatim}

% ---
\subsection{Tabelas}

As tabelas e os quadros apresentam os dados de modo resumido, oferecendo uma visão geral do conteúdo em questão, visando facilitar a compreensão do fenômeno em estudo. A diferença básica entre ambas está relacionada ao conteúdo e a formatação. 

Tabela é o conjunto de dados estatísticos, dispostos em determinada ordem de classificação, que expressam as variações qualitativas de um fenômeno. Sua finalidade básica é resumir ou sintetizar dados \cite{sibi2009}.

A construção de tabelas deve obedecer os critérios estabelecidos pelo Instituto Brasileiro de Geografia e Estatística (IBGE) e requerido pelas normas da ABNT para documentos técnicos e acadêmicos.

A \autoref{tab-ibge} é um exemplo de tabela alinhada que pode ser longa ou curta, conforme padrão do IBGE.

\begin{table}[htb]
	%\begin{table}[H]
	\IBGEtab{%
		\caption{Frequência anual por categoria de usuários}%
		\label{tab-ibge}
	}{%
	\begin{tabular}{ccc}
		\toprule
		Categoria de Usuários & Frequência de Usuários \\
		\midrule \midrule
		Graduação & 72\% \\
		\midrule 
		Pós-Graduação & 15\% \\
		\midrule 
		Docente & 10\% \\
		\midrule 
		Outras & 3\% \\
		\bottomrule
	\end{tabular}%
}{%
\fonte{Produzido pelos autores.}%
\nota{Exemplo de uma nota.}%
\nota[Anotações]{Uma anotação adicional, que pode ser seguida de várias
	outras.}%

}
\end{table}


\begin{table}[H]
	\IBGEtab{%
		\caption{Níveis descritivos dos testes de comparação de médias entre grupos para profundidade da lesão junto à restauração}%
		\label{tabela-ibge}
	}{%
	\begin{tabular}{p{5.5cm}|p{5.5cm}}
		\hline
		\textbf{Resultado} & \textbf{Nível Descritivo} \\ 
		\hline 
		CIC < Ariston & < 0,0001  \\
		Ariston < Am & 0,0118  \\
		Am = Helio & 0,4576  \\
		-100 = Helio & 0,3360  \\
		\hline
	\end{tabular}%
}{%
\fonte{\citeonline{sibi2009}}%
}
\end{table} 

Os \textbf{(APÊNDICES G-H)} exemplificam outras formatações de tabelas.
% ---
\subsection{Figuras}\label{sec_figuras}
% ---

\index{figuras}Figuras podem ser criadas diretamente em \LaTeX,
como o exemplo da \autoref{fig_circulo}. \\ 


\begin{figure}[htb]
	\caption{\label{fig_circulo}A delimitação do espaço}
	\begin{center}
		\setlength{\unitlength}{9cm}
		\begin{picture}(1,1)
		\put(0,0){\line(0,1){1}}
		\put(0,0){\line(1,0){1}}
		\put(0,0){\line(1,1){1}}
		\put(0,0){\line(1,2){.5}}
		\put(0,0){\line(1,3){.3333}}
		\put(0,0){\line(1,4){.25}}
		\put(0,0){\line(1,5){.2}}
		\put(0,0){\line(1,6){.1667}}
		\put(0,0){\line(2,1){1}}
		\put(0,0){\line(2,3){.6667}}
		\put(0,0){\line(2,5){.4}}
		\put(0,0){\line(3,1){1}}
		\put(0,0){\line(3,2){1}}
		\put(0,0){\line(3,4){.75}}
		\put(0,0){\line(3,5){.6}}
		\put(0,0){\line(4,1){1}}
		\put(0,0){\line(4,3){1}}
		\put(0,0){\line(4,5){.8}}
		\put(0,0){\line(5,1){1}}
		\put(0,0){\line(5,2){1}}
		\put(0,0){\line(5,3){1}}
		\put(0,0){\line(5,4){1}}
		\put(0,0){\line(5,6){.8333}}
		\put(0,0){\line(6,1){1}}
		\put(0,0){\line(6,5){1}}
		\end{picture}
	\end{center}
	\legend{Fonte: \citeonline{equipeabntex2}}
\end{figure}

Outra opção é incorporar a figura utilizando um arquivo externo, como é o caso da \autoref{fig_grafico}. Se a figura que for incluída se tratar de um diagrama, um gráfico ou uma ilustração, que você mesmo produza, priorize o uso de imagens vetoriais no formato PDF. Com isso, o tamanho do arquivo final do trabalho será menor e as imagens terão uma apresentação melhor, principalmente quando impressas, uma vez que imagens vetoriais são perfeitamente escaláveis para qualquer dimensão. Nesse caso, se for utilizar o Microsoft Excel para produzir gráficos, ou o Microsoft Word para ilustrações, exporte-os como PDF e os incorpore ao documento conforme o exemplo abaixo. No entanto, para manter a
coerência no uso de software livre (já que você está usando \LaTeX\  e \abnTeX),
teste a ferramenta \textsf{InkScape}\index{InkScape}
(\url{http://inkscape.org/}). Ela é uma excelente opção de código-livre para
produzir ilustrações vetoriais, similar ao CorelDraw\index{CorelDraw} ou ao Adobe
Illustrator\index{Adobe Illustrator}. De todo modo, caso não seja possível
utilizar arquivos de imagens como PDF, utilize qualquer outro formato, como
JPEG, GIF, BMP, etc. Nesse caso, você pode tentar aprimorar as imagens
incorporadas com o software livre \textsf{Gimp}\index{Gimp}
(\url{http://www.gimp.org/}). Ele é uma alternativa livre ao Adobe
Photoshop\index{Adobe Photoshop}. \\

\begin{figure}[H]
	\caption{\label{fig_grafico}Gráfico produzido em Excel e salvo como PDF}
	\includegraphics[scale=0.5]{USPSC-modelo-img-grafico.pdf}
	\begin{flushleft}
		Fonte: \citeonline[p. 24]{araujo2012}
	\end{flushleft}	
\end{figure}


A formatação do quadro é similar à tabela, mas deve ter suas laterais fechadas e conter as linhas horizontais.

% o comando \newpage foi utilizado para forçar a quebra de página

\begin{quadro}[htb]
	\caption{\label{quadro_modelo}Níveis de investigação}
	\begin{tabular}{|p{2.6cm}|p{6.0cm}|p{2.25cm}|p{3.40cm}|}
		\hline
		\textbf{Nível de Investigação} & \textbf{Insumos}  & \textbf{Sistemas de Investigação}  & \textbf{Produtos}  \\
		\hline
		Meta-nível & Filosofia\index{filosofia} da Ciência  & Epistemologia &
		Paradigma  \\
		\hline
		Nível do objeto & Paradigmas do metanível e evidências do nível inferior &
		Ciência  & Teorias e modelos \\
		\hline
		Nível inferior & Modelos e métodos do nível do objeto e problemas do nível inferior & Prática & Solução de problemas  \\
		\hline
	\end{tabular}
	\begin{flushleft}
		%\fonte{\citeonline{van1986}}
		Fonte: \citeonline{van1986}
	\end{flushleft}
\end{quadro} 


Os \textbf{(APÊNDICES B-F)} são exemplos de quadros.

% ---
\subsubsection{Figuras em minipages}
% ---

As ilustrações devem sempre ter numeração contínua e única em todo o documento:

% O comando \newpage força a quebra de página

\begin{citacao}
	Qualquer que seja o tipo de ilustração, sua identificação aparece na parte
	superior, precedida da palavra designativa (desenho, esquema, fluxograma,
	fotografia, gráfico, mapa, organograma, planta, quadro, retrato, figura,
	imagem, entre outros), seguida de seu número de ordem de ocorrência no texto,
	em algarismos arábicos, travessão e do respectivo título. Após a ilustração, na
	parte inferior, indicar a fonte consultada (elemento obrigatório, mesmo que
	seja produção do próprio autor), legenda, notas e outras informações
	necessárias à sua compreensão (se houver). A ilustração deve ser citada no
	texto e inserida o mais próximo possível do trecho a que se
	refere \cite{nbr14724}.
\end{citacao}

\emph{Minipages} são usadas para inserir textos ou outros elementos em quadros
com tamanhos e posições controladas. Veja o exemplo da
\autoref{fig_minipage_imagem1} e da \autoref{fig_minipage_grafico2}.

\begin{figure}[H]
	\label{teste}
	\centering
	\begin{minipage}{0.4\textwidth}
		\centering
		\caption{Imagem 1 da minipage} \label{fig_minipage_imagem1}
		\includegraphics[scale=0.9]{USPSC-modelo-img-marca.pdf}
		\legend{Fonte: \citeonline{equipeabntex2}}
	\end{minipage}
	\hfill
	\begin{minipage}{0.4\textwidth}
		\centering
		\caption{Grafico 2 da minipage} \label{fig_minipage_grafico2}
		\includegraphics[scale=0.2]{USPSC-modelo-img-grafico.pdf}
		\legend{Fonte: \citeonline[p. 24]{araujo2012}}
	\end{minipage}
\end{figure}

% ---
\subsection{Expressões matemáticas}
% ---

\index{expressões matemáticas}Use o ambiente \texttt{equation} para escrever
expressões matemáticas numeradas:

\begin{equation}
\forall x \in X, \quad \exists \: y \leq \epsilon
\end{equation}

Escreva expressões matemáticas entre \$ e \$, como em $ \lim_{x \to \infty}
\exp(-x) = 0 $, para que fiquem na mesma linha.

Também é possível usar colchetes para indicar o início de uma expressão
matemática que não é numerada.

\[
\left|\sum_{i=1}^n a_ib_i\right|
\le
\left(\sum_{i=1}^n a_i^2\right)^{1/2}
\left(\sum_{i=1}^n b_i^2\right)^{1/2}
\]

Consulte mais informações sobre expressões matemáticas em
\url{https://github.com/abntex/abntex2/wiki/Referencias}.


% ---
\subsection{Estruturas, reações e mecanismos de reações químicas}
% ---
O pacote chemfig permite o desenho de estruturas, reações e mecanismos de reações químicas em latex. Abaixo relacionamos alguns exemplos de utilização de seus recursos e indicamos a consulta do Chemfig Manual para mais informações.

A fórmula estrutural do metano é:
\begin{center}
	\chemfig{C(-[5]H)(-[2]H)(<[:-70]H)(<:[:-20]H)}
	
\end{center}


A fórmula estrutural do 1-hexeno é:
\begin{center}
	\chemfig{H_3C-[,1.5]{{(CH_2)}_3}-[,1.5]CH=CH_2}
	
\end{center}



Molecula da Adrenalina
\begin{center}
\chemfig{*6((-HO)-=-(-(<[::60]OH)-[::-60]-[::-60,,,2]
	HN-[::+60]CH_3)=-(-HO)=)}


\end{center}



Exemplo de reações químicas:

\begin{center}
	\schemestart
	\chemfig{-[:30](-[2])-[:-30]OH}
	\arrow
	\chemfig{-[:30](-[2])=^[:-30]O}
	\schemestop
	
\end{center}

Mais um exemplo de reações químicas:
\begin{center}\small\setatomsep{1.5em}
	\schemestart
	\chemfig{*6(=-=(-(=[2]O)-[::-60]O-[0]O-[::30](=[2]O)-[::-60]*6(=-=-=-))-=-)}
	\arrow{->[$\Delta$]}
	2 \chemfig{*6(=-=(-(=[2]O)-[::-60]\lewis{0.,O})-=-)}
	\arrow
	2 \chemfig{*6(=-=(-[,.15,,,draw=none]\lewis{0.,})-=-)}\+\ch{2 CO2 ^}
	\schemestop
	
	
\end{center}

Os exemplos abaixo ilustram que é possível modificar cores, utilizar o recurso de perspectiva, dentre outros recursos que obter destaques.


Mudando as cores das moleculas
\begin{center}
\chemfig{C|{\color{blue}H_3}-C(=[1]O)-[7]O|{\color{red}H}}


\end{center}


Moleculas em perspectiva
\begin{center}
	\setcrambond{2pt}{}{}
	\chemfig{HO-[2,0.5,2]?<[7,0.7]-[,,,,
		line width=2pt]>[1,0.7]-[3,0.7]O-[4]?}
	
	
\end{center}


% ---
\subsection{Inclusão de outros arquivos}\label{sec-include}
% ---

É uma boa prática dividir o seu documento em diversos arquivos, e não
apenas escrever tudo em um único. Esse recurso foi utilizado neste
documento. Para incluir diferentes arquivos em um arquivo principal,
de modo que cada arquivo incluído fique em uma página diferente, utilize o
comando:

\begin{verbatim}
   \include{documento-a-ser-incluido}      % sem a extensão .tex
\end{verbatim}

Para incluir documentos sem quebra de páginas, utilize:

\begin{verbatim}
   \input{documento-a-ser-incluido}      % sem a extensão .tex
\end{verbatim}
% ---
\subsection{Índice(s)}
% ---
Elemento opcional, que consiste em lista de palavras ou frases ordenadas alfabeticamente (autor, título ou assunto) ou sistematicamente (ordenação por classes, numérica ou cronológica); localiza e remete para as informações contidas no texto. A paginação deve ser contínua, dando seguimento ao texto principal \cite{sibi2009}.

Para criar índice remissivo no \LaTeX\  utilize o pacote makeidx, que deve estar declarado com os demais pacotes. No presente modelo está declarado no arquivo USPSC-modelo.tex, conforme indicado abaixo:

\begin{verbatim}
% ---
% Pacotes básicos - Fundamentais 
% ---
\usepackage[T1]{fontenc}		% Selecao de codigos de fonte.
\usepackage[utf8]{inputenc}		% Codificacao do documento (conversão 
automática dos acentos)
\usepackage{lmodern}			% Usa a fonte Latin Modern
% Para utilizar a fonte Times New Roman, inclua uma % no início do comando 
acima  "\usepackage{lmodern}"
% Lembre-se de alterar a fonte no comando que imprime o preâmbulo no 
arquivo da Classe USPSC.cls
%\usepackage{times}			% Usa a fonte Times New Roman					
\usepackage{lastpage}			% Usado pela Ficha catalográfica
\usepackage{indentfirst}		% Indenta o primeiro parágrafo de cada seção.
\usepackage{color}				% Controle das cores
\usepackage{graphicx}			% Inclusão de gráficos
\usepackage{float} 				%Fixa tabelas e figuras no local exato
\usepackage{microtype} 			% para melhorias de justificação
\usepackage{pdfpages}
\usepackage[labelsep=endash]{caption}
\usepackage{makeidx}            % para gerar índice remissimo
% ---
\end{verbatim}

A habilitação dos comandos de indexação foi incluída no arquivo USPSC-modelo.tex da seguinte forma:


\begin{verbatim}
% compila o sumário e índice
\makeindex
% ---
\end{verbatim}

O presente modelo inclui um exemplo de índice, gerado a partir da utilização de comandos similares aos reproduzidos abaixo:

\begin{verbatim}
\index{InkScape}
\index{CorelDraw}
\index{Adobe Illustrator}
\index{Gimp}
\index{Adobe Photoshop}
\index{espaçamento!do primeiro parágrafo}
\index{espaçamento!dos parágrafos}
\index{espaçamento!entre as linhas}
\index{espaçamento!entre os parágrafos}
\end{verbatim}

Os comandos acima estão no arquivo USPSC-Cap2-Desenvolvimento.tex, em textos na  \autoref{sec_figuras} e em \autoref{sec_espacamento}.

Para imprimir o índice, no final do arquivo USPSC-modelo.tex foi incluído:

\begin{verbatim}

%---------------------------------------------------------------------
% INDICE REMISSIVO
%--------------------------------------------------------------------
\phantompart
\printindex
%---------------------------------------------------------------------
\end{verbatim}

Para que o índice seja gerado e incluído corretamente no texto é necessário compilá-lo separadamente. No \textbf{TeXstudio 2.9.4}, na barra de menu, selecione \textbf{Tools} e execute \textbf{Index}.


% ---
\subsection{Compilar o documento \LaTeX}
% ---

Geralmente os editores \LaTeX, como o
TeXlipse\footnote{\url{http://texlipse.sourceforge.net/}}, o
Texmaker\footnote{\url{http://www.xm1math.net/texmaker/}}, entre outros,
compilam os documentos automaticamente, de modo que você não precisa se
preocupar com isso.

No entanto, você pode compilar os documentos \LaTeX\ usando os seguintes
comandos, que devem ser digitados no \emph{Prompt de Comandos} do Windows ou no
\emph{Terminal} do Mac ou do Linux:

\begin{verbatim}
   pdflatex ARQUIVO_PRINCIPAL.tex
   bibtex ARQUIVO_PRINCIPAL.aux
   makeindex ARQUIVO_PRINCIPAL.idx 
   makeindex ARQUIVO_PRINCIPAL.nlo -s nomencl.ist -o ARQUIVO_PRINCIPAL.nls
   pdflatex ARQUIVO_PRINCIPAL.tex
   pdflatex ARQUIVO_PRINCIPAL.tex
\end{verbatim}

% ---
\subsection{Remissões internas}
% ---

Ao nomear a \autoref{tab-ibge} e a \autoref{fig_circulo}, apresentamos um exemplo de remissão interna, que também pode ser feita quando indicamos o
\autoref{cap_exemplos}, que tem o nome \emph{\nameref{cap_exemplos}}. O número
do capítulo indicado é \ref{cap_exemplos}, que se inicia à
\autopageref{cap_exemplos}\footnote{O número da página de uma remissão pode ser
	obtida também assim:
	\pageref{cap_exemplos}.}.
Veja a \autoref{sec-divisoes-b} para outros exemplos de remissões internas entre
seções, subseções e subsubseções.

O código usado para produzir o texto desta seção é:

\begin{verbatim}
Ao nomear a \autoref{tab-nivinv} e a \autoref{fig_circulo}, apresentamos 
um exemplo de remissão interna, que também pode ser feita quando indicamos 
o \autoref{cap_exemplos}, que tem o nome \emph{\nameref{cap_exemplos}}. O
número do capítulo indicado é \ref{cap_exemplos}, que se inicia à 
\autopageref{cap_exemplos}\footnote{O número da página de uma remissão 
pode ser obtida também assim: \pageref{cap_exemplos}.}. Veja a 
\autoref{sec-divisoes-b} para outros exemplos de remissões internas entre 
seções, subseções e subsubseções.
\end{verbatim}

% ---
\section{Divisões do documento}\label{sec-divisoes-b}
Esta seção exemplifica o uso de divisões de documentos em conformidade com a ABNT NBR 6024  \cite{nbr6024}.
% ---
% ---
\subsection{Divisões do documento: subseção}\label{sec-divisoes-subsection}
% ---

Um exemplo de seção é a \autoref{sec-divisoes-b}. Esta é a \autoref{sec-divisoes-subsection}.

\subsubsection{Divisões do documento: subsubseção}\label{sec-divisoes-subsubsection}

Isto é uma \texttt{subsubsection} do \LaTeX, mas é denominada de ``subseção'' porque no português não temos a palavra ``subsubseção''.

\subsubsection{Divisões do documento: subsubseção}

Isto é outra subsubseção.

\subsection{Divisões do documento: subseção}\label{sec-exemplo-subsec}

Isto é uma subseção.

\subsubsection{Divisões do documento: subsubseção}

Isto é mais uma subsubseção da \autoref{sec-exemplo-subsec}.


\subsubsubsection{Esta é uma subseção de quinto
nível}\label{sec-exemplo-subsubsubsection}

Esta é uma seção de quinto nível. Ela é produzida com o seguinte comando:

\begin{verbatim}
\subsubsubsection{Esta é uma subseção de quinto
nível}\label{sec-exemplo-subsubsubsection}
\end{verbatim}

\subsubsubsection{Esta é outra subseção de quinto nível}\label{sec-exemplo-subsubsubsection-outro}

Esta é outra seção de quinto nível.


\paragraph{Este é um parágrafo numerado}\label{sec-exemplo-paragrafo}

Este é um exemplo de parágrafo nomeado. Ele é produzido com o comando de
parágrafo:

\begin{verbatim}
\paragraph{Este é um parágrafo nomeado}\label{sec-exemplo-paragrafo}
\end{verbatim}

A numeração entre parágrafos numerados e subsubsubseções são contínuas.

\paragraph{Esta é outro parágrafo numerado}\label{sec-exemplo-paragrafo-outro}

Este é outro parágrafo nomeado.

% ---
\subsection{Este é um exemplo de nome de subseção longa que se aplica a seções e demais divisões do documento. Ele deve estar alinhado à esquerda e a segunda e demais linhas devem iniciar logo abaixo da primeira palavra da primeira linha} 

Observe que o alinhamento do título obedece esta regra também no sumário.
	

% ---
\section{Manual da classe \textsf{\abnTeX}}
% ---

O manual da classe \textsf{\abnTeX} possui uma referência completa das macros e ambientes disponíveis \cite{abnetxclasse}.

Contém informações adicionais sobre as normas ABNT
observadas pelo \textsf{\abnTeX} e considerações sobre eventuais requisitos específicos
não atendidos, como o caso da ABNT NBR 14724 \cite{nbr14724}, que
especifica o espaçamento entre os capítulos e o início do texto, regra
propositalmente não atendida pelo presente modelo.

% ---
\section{Precisa de ajuda sobre \textsf{\abnTeX}?}
% ---

Consulte a FAQ com perguntas frequentes e comuns no portal do \textsf{\abnTeX}:
\url{https://github.com/abntex/abntex2/wiki/FAQ}.

Inscreva-se no grupo de usuários \LaTeX:
\url{http://groups.google.com/group/latex-br}, tire suas dúvidas e ajude
outros usuários.

Participe também do grupo de desenvolvedores do \textsf{\abnTeX}:
\url{http://groups.google.com/group/abntex2} e faça sua contribuição à
ferramenta.

% ---
\section{Você pode ajudar?}
% ---

Sua contribuição é muito importante! Você pode ajudar na divulgação, no
desenvolvimento e de várias outras formas. Veja como contribuir com o \abnTeX\
em \url{https://github.com/abntex/abntex2/wiki/Como-Contribuir}.

% ---
\section{Quer customizar os modelos do \abnTeX\ para sua instituição ou
universidade?}
% ---

Veja como customizar o \abnTeX\ em:
\url{https://github.com/abntex/abntex2/wiki/ComoCustomizar}.

% ---
\section{Precisa de ajuda sobre a Classe USPSC e Modelos?}
% ---
Para obter ajuda sobre a Classe USPSC e o Modelo para teses e dissertações em \LaTeX\  utilizando a classe USPSC, consulte a Seção de Referência da Biblioteca de sua instituição.

No Campus USP de São Carlos, consulte uma das seguintes equipes de referência:
\begin{verbatim}
EESC - Serviço de Biblioteca Prof. Dr. Sérgio Rodrigues Fontes 
Atendimento ao Usuário
biblioteca.atendimento@eesc.usp.br
(16) 3373-8860

IAU - Biblioteca
Atendimento ao Usuário
bibiau@sc.usp.br
(16) 3373-9282

ICMC - Biblioteca Prof. Achille Bassi
Seção de Atendimento ao Usuário
biblio@icmc.usp.br
(16) 3373-8619

IFSC - Serviço de Biblioteca e Informação Prof. Bernhard Gross
Seção de Atendimento ao Usuário
comut@ifsc.usp.br
(16) 3373-9778

IQSC - Serviço de Biblioteca e Informação Prof. Johannes Rüdiger Lechat
Seção de Atendimento ao Usuário
bibiqsc@iqsc.usp.br
(16) 3373-9936
\end{verbatim}


O Grupo desenvolvedor da Classe USPSC e deste Modelo esclarece que seu objetivo é oferecer um facilitador para os pós-graduandos, mas não se compromete a ensinar a Linguagem de Programação \LaTeX .  

% ---
\section{Customize a Classe USPSC e Modelos para sua instituição}
% ---

Para customizar o \textbf{Modelo para teses e dissertações em \LaTeX\ utilizando a Classe USPSC} para outras Unidades da USP e demais instituições interessadas em adotar essas normas e padrões, basta criar um arquivo pré-textual contemplando os programas de pós-graduação vigentes e incluir a chamada do mesmo em USPSC-unidades.tex.

Para solicitar orientações como proceder, contactar as responsáveis pela programação:

\begin{verbatim}
Biblioteca da Prefeitura do Campus USP de São Carlos - PUSP-SC/USP
Marilza Aparecida Rodrigues Tognetti
Ana Paula Aparecida Calabrez
biblioteca.prefeitura@sc.usp.br
(16) 3373-8316
\end{verbatim}






% ---
% Capítulo 3
% ---
% ---
%% USPSC-Cap3-Citacoes.tex
% --
% Este capítulo traz os exemplos de citações das "Diretrizes para apresentação de dissertações e teses da USP: documento eletrônico e impresso - Parte I (ABNT)" disponílvel em: http://biblioteca.puspsc.usp.br/pdfFiles_Caderno_Estudos_9_PT_1.pdf


% --- 
\chapter{Results and Discussion}
\label{ch:disc}
% --- 
Citação é a menção no texto de informações extraídas de uma fonte documental que tem o propósito de esclarecer ou fundamentar as ideias do autor. A fonte de onde foi extraída a informação deve ser citada obrigatoriamente, respeitando-se os direitos autorais, conforme ABNT NBR 10520 \cite{nbr10520}.

As citações mencionadas no texto devem, obrigatoriamente, seguir a mesma forma de entrada utilizada nas Referências, no final do trabalho e/ou em Notas de Rodapé.

Todos os documentos relacionados nas Referências devem ser citados no texto, assim como todas as citações do texto devem constar nas Referências. 

Os textos que constam desse manual e os exemplos de citações e referências foram elaborados com base nas \textbf{Diretrizes para apresentação de dissertações e teses da USP}: documento eletrônico e impresso - Parte I (ABNT) \cite{sibi2009}.

Para elaborar as citações utilizando a Classe USPSC é necessário a instalação do pacote: 

\begin{alineas}
	\item \textbf{usepackage[num]abntex2cite:} para gerar citações e referências em estilo numérico;
	\item \textbf{usepackage[alf]abntex2cite:} para gerar citações e referências em estilo alfabético.
\end{alineas}

As explicações para utilização do pacote abntex2cite e exemplos de como elaborar citações e referências de acordo com as normas da ABNT está presente nos manuais: \textbf{O pacote abntex2cite}: estilos bibliográficos compatíveis com a ABNT NBR 6023 \cite{abnetxcite} e  \textbf{O pacote abntex2cite}: tópicos específicos da ABNT NBR 10520:2002 e o estilo bibliográfico alfabético (sistema autor-data) \cite{abnetxcitealf}.

Abaixo seguem alguns exemplos de citações, mas se o exemplo que você precisa não estiver contemplado aqui, acesse o manual \textbf{O pacote abntex2cite} que possui aproximadamente 240 modelos de referências.

Em todo esse documento e especificamente nos exemplos abaixo, foi utilizado o ponto final após o comando \verb+\cite{}+, em conformidade com sistema autor-data. Para o sistema numérico é necessário utilizar o ponto final antes do comando \verb+\cite{}+. 

Alertamos que se este documento for alterado para sistema numérico a pontuação final ficará incorreta. \\

\section{Citação direta}

É a transcrição (reprodução integral) de parte da obra consultada, conservando-se a grafia, pontuação, idioma etc.

A reprodução de um texto de até três linhas deve ser incorporada ao parágrafo entre aspas duplas, mesmo que compreenda mais de um parágrafo. As aspas simples são utilizadas para indicar citação no interior da citação.

\textbf{Exemplos:}

\begin{alineas} 
\item 
\begin{verbatim}
\citeonline[p.~27]{KOK2013} refere ao "Texto texto texto texto 
texto texto texto texto texto texto texto texto texto texto."
\end{verbatim}
Que corresponde: \\
\citeonline[p.~27]{KOK2013} refere ao "Texto texto texto texto texto texto texto texto texto texto texto texto texto texto."
\item 
\begin{verbatim}
"Texto texto texto texto texto texto texto texto texto texto texto 
texto texto texto texto texto texto texto." \cite[p.~67]{Krauss1997}.
\end{verbatim}
Que corresponde: \\
"Texto texto texto texto texto texto texto texto texto texto texto texto texto texto texto texto texto texto." \cite[p.~67]{Krauss1997}.

\item 
\begin{verbatim}
Segundo \citeonline [p.~618]{Moss1999}: "[\ldots] texto texto texto 
texto texto texto texto texto texto texto texto texto [\ldots]".
\end{verbatim}
Que corresponde: \\
Segundo \citeonline [p.~618]{Moss1999}: "[\ldots] texto texto texto texto texto texto texto texto texto texto texto texto [\ldots]".

\item 
\begin{verbatim}
"Texto texto texto texto texto texto texto texto\textbf{texto texto}
texto."\cite[v.~2, p.18, grifo do autor]{ROMANO1996}. 
\end{verbatim}
Que corresponde: \\
"Texto texto texto texto texto texto texto texto texto texto texto\textbf{texto texto} texto texto texto texto texto texto texto texto." \cite[v.~2, p.18, grifo do autor]{ROMANO1996}. 

\end{alineas}

As transcrições com mais de três linhas devem figurar abaixo do texto, com recuo de 4 cm da margem esquerda, com letra menor que a do texto utilizado e sem aspas.Utilize o ambiente citação para incluir citações diretas com mais de três linhas.

Use o ambiente assim: 

\verb+\begin{citação}+

Texto texto texto texto texto texto texto texto texto.

\verb+\end{citação}+

O ambiente citação pode receber como parâmetro opcional um nome de idioma previamente carregado nas opções da classe. Nesse caso, o texto da citação é automaticamente escrito em itálico e a hifenização é ajustada para o idioma selecionado na opção do ambiente.\\
 Por exemplo:
 
\verb+\begin{citacao}[english]+
 
 Text in English language in italic with correct hyphenation.
 
\verb+\end{citacao}+
 
Tem como resultado:
\begin{citacao}[english]
Text in English language in italic with correct hyphenation. \\
\end{citacao}

\textbf{Exemplos:} \\

\begin{alineas} 

\item 
\begin{verbatim}
Texto texto texto texto texto texto texto texto texto texto texto. 
\begin{citacao}
Texto texto texto texto texto texto [\ldots] textos textos textos
texto texto texto texto texto texto texto texto texto texto texto 
texto texto texto texto texto texto texto texto texto texto texto 
texto texto texto texto texto texto texto texto texto texto texto
texto texto texto. \cite[p.~10]{Farias2001}.
\end{citacao}
\end{verbatim}
Que corresponde: \\
Texto texto texto texto texto texto texto texto texto texto texto. 
\begin{citacao}
Texto texto texto texto texto texto  [\ldots] textos textos textos Texto texto texto texto texto texto texto texto texto texto texto texto texto texto texto texto texto texto texto texto texto texto texto texto texto texto texto texto texto  texto texto texto texto. \cite[p.~10]{Farias2001}.
\end{citacao}	
\item
\begin{verbatim}
Valendo-se de várias hipóteses \citeonline[p.~21]{Gubitoso1989} 
constata que: 
\begin{citacao}
Texto texto texto texto texto texto texto texto texto texto texto
texto texto texto texto texto texto texto texto texto texto texto 
texto texto texto texto texto texto texto texto texto texto texto 
texto texto texto texto texto texto texto texto texto texto texto.
\end{citacao}
\end{verbatim}
Que corresponde: \\
Valendo-se de várias hipóteses \citeonline[p.~21]{Gubitoso1989} constata que:
\begin{citacao}
Texto texto texto texto texto texto texto texto texto texto texto. Texto texto texto texto texto texto texto texto texto texto texto texto texto texto texto texto texto texto texto texto texto texto texto texto texto texto texto texto texto  texto texto texto texto.\\
\end{citacao}
\item
\begin{verbatim}
De acordo com \citeonline[p.~S4]{Hood1999}
\begin{citacao}[english]
Text in English. Text in English. Text in English. Text in
English. Text in English. Text in English. Text in English. 
Text in English. Text in English. Text in English. Text in
English. Text in English.
\end{citacao}
\end{verbatim}
Que corresponde: \\
 De acordo com \citeonline[p.~S4]{Hood1999}
\begin{citacao}[english]
	Text in English. Text in English. Text in English. Text in English. Text in English. Text in English. Text in English. Text in English. Text in English. Text in English Text in English. Text in English.
\end{citacao}

\end{alineas}

\section{Citação indireta}

É o texto criado com base na obra de autor consultado, em que se reproduz o conteúdo e ideias do documento original; dispensa o uso de aspas duplas.

\textbf{Exemplos:}\\
\begin{alineas}
\item
\begin{verbatim}
Texto texto texto texto texto texto texto \cite{Naves25abr.1999}.
\end{verbatim}
Que corresponde: \\
Texto texto texto texto texto texto texto \cite{Naves25abr.1999}.
\item
\begin{verbatim}
Para \citeonline{Sukikara2007} texto texto texto texto texto texto.
\end{verbatim}
Que corresponde: \\
Para \citeonline{Sukikara2007} texto texto texto texto texto texto.
\item
\begin{verbatim}
Conforme \citeonline[p.~53]{Catani1989} texto texto texto texto.
\end{verbatim}
Que corresponde: \\
Conforme \citeonline[p.~53]{Catani1989} texto texto texto texto.\\
\end{alineas} 


\section{Citação de citação}

É a citação direta ou indireta de um texto que se refere ao documento original, que não se teve acesso.
Indicar no texto o sobrenome do(s) autor(es) do documento não consultado, seguido da data, da expressão latina apud (citado por) e do sobrenome do(s) autor(es) do documento consultado, data e página. 
Este tipo de citação só deve ser utilizada nos casos em que o documento original não foi recuperado (documentos muito antigos, dados insuficientes para a localização do material etc.).

Para elaboração de citação de citação são disponibilizados os seguintes comandos: \verb+\apud e \apudonline+.

\textbf{Exemplos:}

\begin{alineas}

\item
\begin{verbatim}
"[\ldots] texto texto..." \apud[p.~54]{Castro1990}{Alves2002}. 
\end{verbatim}
Que corresponde: \\
"[\ldots] texto texto texto texto texto texto texto texto texto texto texto. Texto texto texto texto texto texto texto texto texto texto texto texto texto texto texto." \apud[p.~54]{Castro1990}{Alves2002}.

\item
\begin{verbatim}
\apudonline {Gomes1992}{Azevedo2015} texto texto texto texto texto.
\end{verbatim}
Que corresponde:

\apudonline{Gomes1992}{Azevedo2015} texto texto texto texto texto texto texto texto texto texto texto. Texto texto texto texto texto texto texto texto texto texto texto texto texto texto texto.
 
\end{alineas}

Ressaltamos que os comandos \verb+\apud e \apudonline+ estão em conformidade com ABNT NBR 10520 e não permitem a inserção de notas de rodapés nos sobrenomes dos autores citados. Para elaborar a citação de citação conforme as Diretrizes da USP, que sugere a inclusão da citação da obra consultada nas referências e mencionar, em nota de rodapé, a referência do trabalho não consultado, é necessário criar a citação conforme abaixo:, esse recurso não deve ser utilizado para citações com sistema numérico, já que as notas de rodapé estão configuradas com símbolos. 

\begin{alineas}
\item
\begin{verbatim}
Saadi\footnote{SAADI, S.\textbf{O jardim das rosas.} Tradução 
de Aurélio Buarque de Holanda. Rio de Janeiro: J. Olympio, 1944.
124 p.(Coleção Rubayat). Versão francesa de Franz Toussaint do 
original àrabe.} (1944 apud \citeauthor{Alves2002}, 2002, p.15) 
texto texto texto texto texto texto texto texto texto texto texto. 
\end{verbatim}
Que Corresponde: \\

Saadi\footnote{SAADI, S.\textbf{O jardim das rosas.} Tradução de Aurélio Buarque de Holanda. Rio de Janeiro: J. Olympio, 1944. 124 p.(Coleção Rubayat). Versão francesa de Franz Toussaint do original àrabe.} (1944 apud \citeauthor{Alves2002}, 2002, p.15) texto texto texto texto texto texto texto texto texto texto texto. 

\item
\begin{verbatim}
"[\ldots] texto texto texto texto texto texto texto texto texto 
texto texto texto texto texto texto texto texto texto texto texto"
(ESPÍRITO SANTO\footnote{ESPÍRITO SANTO, A. \textbf{Essências de
metodologia científica:} aplicada à educação. Londrina: 
Universidade Estadual, 1987}, 1987 p.15 apud \citeauthor
{Azevedo2015}, 2015, p.101).
\end{verbatim}
Que corresponde: \\
"[\ldots] texto texto texto texto texto texto texto texto texto texto texto texto texto texto texto texto texto texto texto texto". (ESPÍRITO SANTO\footnote{ESPÍRITO SANTO, A. \textbf{Essências de metodologia científica:} aplicada à educação. Londrina: Universidade Estadual, 1987}, 1987 p.15 apud \citeauthor
{Azevedo2015}, 2015, p.101).
\end{alineas}

\textbf{Observação:}

Também é possível escolher dentre os dois comandos: \verb+\footciteref{}+ e o comando \verb+\footnote{\citetext{}}+ para inserir referências em notas de rodapés, mas ao utilizar esses comandos a referência é automaticamente inserida na lista final de referências, constando tanto das notas de rodapés quanto da lista de referências.

\section{Citação de fontes informais}

\textbf{Informação Verbal}

Quando obtidas através de comunicações pessoais, anotações de aulas, trabalhos de eventos não publicados (conferências, palestras, seminários, congressos, simpósios etc.), indicar entre parênteses a expressão (informação verbal), mencionando os dados disponíveis somente em nota de rodapé.

\textbf{Exemplos:}

\begin{alineas}
\item
\begin{verbatim}
Silva (1983) texto texto texto texto texto texto [\ldots] 
(informação verbal).\footnote{Informação fornecida por 
Silva em Belo Horizonte, em 1983.}
\end{verbatim}
Que corresponde:\\
Silva (1983) texto texto texto texto texto texto [\ldots] (informação verbal).\footnote{Informação fornecida por Silva em Belo Horizonte, em 1983.} \\
\item
\begin{verbatim}
Fukushima e Hagiwara (1979) texto texto texto texto texto texto 
texto texto texto texto [\ldots] (informação verbal).\footnote
{Informação fornecida por Fukushima e Hagiwara na Conferência 
Anual da Sociedade Paulista de Medicina Veterinária, em 1979.}
\end{verbatim}
Que corresponde: \\
Fukushima e Hagiwara (1979) texto texto texto texto texto texto texto texto texto texto texto [\ldots] (informação verbal).\footnote{Informação fornecida por Fukushima e Hagiwara na Conferência Anual da Sociedade Paulista de Medicina Veterinária, em 1979.}\\
\end{alineas}

\textbf{Informação Pessoal}

Indicar, entre parênteses, a expressão (informação pessoal) para dados obtidos de comunicações pessoais, correspondências pessoais (postal ou e-mail), mencionando-se os dados disponíveis em nota de rodapé.

\textbf{Exemplos:}


\begin{alineas}
\item
\begin{verbatim}
Bruckman citou texto texto texto texto texto texto texto texto 
texto. (informação pessoal)\footnote{\citetext{Bruckman2002}}.
\end{verbatim}
Que corresponde:\\
Bruckman citou texto texto texto texto texto texto texto texto texto texto. (informação pessoal)\footnote{\citetext{Bruckman2002}}.
\item
\begin{verbatim}
SCIENCEDIRECT MESSAGE CENTER traz a informação texto texto texto
texto texto. (informação pessoal)\footnote{\citetext{science2006}}.
\end{verbatim}
Que correspode:\\
SCIENCEDIRECT MESSAGE CENTER traz a informação texto texto texto texto texto texto texto texto texto. (informação pessoal)\footnote{\citetext{science2006}}\\
\end{alineas}

\textbf{Em fase de elaboração}

Trabalhos em fase de elaboração devem ser mencionados apenas em nota de rodapé. 

\textbf{Exemplo:}
\begin{alineas}
\item
\begin{verbatim}
Barbosa estudou texto texto texto texto texto texto texto texto 
texto. (em fase de elaboração)\footnote{\citetext{Barbosa2002}}.\\
\end{verbatim}
Que correspode:\\
Barbosa estudou texto texto texto texto texto texto texto texto texto. (em fase de elaboração)\footnote{\citetext{Barbosa2002}}.
\end{alineas}

\section{Citação de website}

O endereço eletrônico é indicado nas Referências. No texto, a citação é referente ao autor ou ao título do trabalho. 

\textbf{Exemplos:}
\begin{alineas}
\item
Texto texto texto texto texto texto texto texto texto texto texto texto texto texto. \cite{galeria1998}.
\item 
Texto texto texto texto texto texto texto texto texto. \cite{usp2006}.
\end{alineas}

\section{Destaque e supressões no texto}

Utilizar os comandos abaixo durante a redação das citações com destaques e supressões.

\verb+\underline{}+: para grifar.

\verb+\textbf{}+: para colocar em negrito.

\verb+\textit{}+: para colocar em itálico.

\verb+[\ldots]+: para supressões [...]. \\

\textbf{Exemplos:}

\begin{alineas}
\item
Usar \underline{grifo} ou \textbf{negrito} ou \textit{itálico} para ênfases ou destaques. Na citação, indicar (grifo nosso) entre parênteses, logo após a data.
\begin{verbatim}
Texto texto \underline{texto} texto texto. \cite[~p.129, grifo nosso]
{Piccini1999}.
\end{verbatim}	
Que corresponde: \\
Texto texto \underline{texto} texto texto. \cite[~p.129, grifo nosso]{Piccini1999}.\\
\item
Usar a expressão “grifo do autor” caso o destaque seja do autor consultado.
\begin{verbatim}
Texto texto \underline{texto} texto texto. \cite[~p.57, grifo do autor]
{Dias1994}.
\end{verbatim}
Que corresponde: \\
Texto texto \underline{texto} texto texto. \cite[~p.57, grifo do autor]{Dias1994}.\\
\item
Indicar as supressões por reticências dentro de colchetes, estejam elas no início, no meio ou no fim do parágrafo e/ou frase.
\begin{verbatim}
Segundo \citeonline[~p.140]{Tollivet1994} "[\ldots]texto texto 
texto texto [\ldots] texto texto". 
\end{verbatim}
Que corresponde:\\
Segundo \citeonline[~p.140]{Tollivet1994} "[\ldots] texto texto texto texto [\ldots] texto texto".\\ 
\item
Indicar as interpolações, comentários próprios, acréscimos e explicações dentro de colchetes, estejam elas no início ou no fim do parágrafo e/ou frase.
\begin{verbatim}
"Texto texto texto [comentário comentário] texto texto texto texto 
texto texto." \cite[~p.8]{Naves25abr.1999}.
\end{verbatim}
Que corresponde:\\
"Texto texto texto [comentário comentário] texto texto texto texto texto texto".  \cite[~p.8]{Naves25abr.1999}.\\
\item
Quando a citação incluir um texto traduzido pelo autor, acrescentar a chamada da citação seguida da expressão “tradução nossa”, tudo entre parênteses.
\begin{verbatim}
"Texto texto texto". \cite[~p.102, tradução nossa]{Malinowski2000}.
\end{verbatim}
Que corresponde:\\
"Texto texto texto". \cite[~p.102, tradução nossa]{Malinowski2000}.\\
\end{alineas}

\section{Notas de rodapé}
As notas de rodapé são observações ou esclarecimentos, cujas inclusões no texto são feitas pelo autor do trabalho. Inclui dados obtidos por fontes informais tais como: informação verbal, pessoal, trabalhos em fase de elaboração ou não consultados diretamente.
Classificam-se em:\\
\begin{alineas}
\item
\textbf{Notas explicativas} constituem-se em comentários, complementações ou traduções que interromperiam a sequência lógica se colocadas no texto.
\item
\textbf{Notas de referências} indicam documentos consultados ou remetem a outras partes do texto onde o assunto em questão foi abordado. \\
\end{alineas}

Devem ser digitadas em fontes menores, dentro das margens, ficando separadas do texto por um espaço simples de entrelinhas e por filete de aproximadamente 5 cm, a partir da margem esquerda.

As notas de rodapé podem ser indicadas por numeração consecutiva, com números sobrescritos dentro do capítulo ou da parte (não se inicia a numeração a cada folha).\\

\textbf{Notas}

Os exemplos de inserção de notas de rodapé já foram expostos nos itens 3.3 e 3.4.

Se a opção for pelo sistema de chamada numérico, a indicação da nota de rodapé deverá ser por símbolos (ex.: asterisco etc.). 
Este modelo está com o sistema numérico para nota de rodapés para mudar para simbólico é necessário ativar o comando \verb+\renewcommand{\thefootnote}{\fnsymbol{footnote}}+

\section{Exemplos de citações}

\textbf{Um autor}

Pelo sobrenome\\

\cite{Abreu2015}

ou

\citeonline{Abreu2015}\\

\textbf{Dois autores}

Os sobrenomes dos autores entre parênteses devem ser separados por ponto e vírgula. Quando citados fora de parênteses devem ser separados pela letra “e”\\

\cite{simone1977}

ou 

\citeonline{simone1977}\\


\textbf{Três autores}

Os sobrenomes dos autores citados entre parênteses devem ser separados por ponto e vírgula. Quando citados fora de parênteses, os autores devem ser separados por vírgula sendo o último separado pela letra “e”.\\

\cite{Giannini2000}

ou

\citeonline{Giannini2000}\\

\textbf{Quatro ou mais autores}

Indicar o sobrenome do primeiro autor seguido da expressão latina et al., sem itálico.\\

\cite{Meyaard2003}

ou

\citeonline{Meyaard2003}\\


\textbf{Citações consecutivas em Sistema Numérico}

Para agrupar a citação numérica quando for consecutiva:

Adicionar o pacote “cite” junto aos demais pacotes listados inicialmente:

\verb+\usepackage{cite}+ \\

Ao citar a referência:

Para 2 referências consecutivas: 

\verb+\cite{bibtexkey}-\cite{bibtexkey}+ \\

Para 3 ou mais: 

\verb+~\cite{bibtexkey}+ \\

\textbf{Documentos de mesmo autor publicado no mesmo ano}


Acrescentar letras minúsculas após o ano, sem espaço.\\

\cite{Hennekens1987b}  \textbf{\underline{outra obra}}   \cite{Hennekens1987a}

ou

\citeonline{Hennekens1987b}  \textbf{\underline{outra obra}}   \citeonline{Hennekens1987a}

\textbf{Autoria desconhecida}

Citar pela primeira palavra do título, seguida de reticências e do ano de publicação.\\

\cite{fgv1984}

ou 

\citeonline{fgv1984}\\

\textbf{Entidade coletivas}

Citar pela forma em que aparece na referência.\\

\cite{CETESB1994}

ou 

\citeonline{CETESB1994}\\

Na lista de referência do trabalho a entrada será feita pelo nome por extenso da entidade coletiva conforme abaixo:\\

\begin{tabular}{|l|c|} \hline
	COMPANHIA ESTADUAL DE TECNOLOGIA DE SANEAMENTO \\AMBIENTAL.
	Bacia hidrográfica do Ribeirão Pinheiros: relatório técnico.\\ São Paulo: CETESB,
	1994. 39 p. \\\hline
\end{tabular}\\

\textbf{Campos em LATEX:}

\begin{verbatim}
@Book{CETESB1994,
Title                    = {Bacia hidrográfica do Ribeirão Pinheiros},
Address                  = {São Paulo},
Organization             = {Companhia Estadual de Tecnologia de 
Saneamento Ambiental},
Pages                    = {39},
Publisher                = {CETESB},
Subtitle                 = {relatório técnico},
Year                     = {1994},
Owner                    = {apcalabrez},
Timestamp                = {2015.09.17}
}
\end{verbatim}

Para as unidades que desejarem citar no texto a sigla da entidade coletiva ao invés do nome completo, é necessário acrescentar na referência o campo Org-Short no arquivo.bib em BibTeX e acrescentar a sigla da entidade coletiva neste campo. As referências que possuírem esse campo serão citadas pela sigla e a referência será organizada no final do trabalho pelo nome por extenso da entidade.

\cite{cetesb94}

ou 

\citeonline{cetesb94}

Na lista de referência do trabalho a entrada será feita pelo nome por extenso da entidade coletiva conforme abaixo:

\begin{tabular}{|l|c|} \hline
COMPANHIA ESTADUAL DE TECNOLOGIA DE SANEAMENTO \\AMBIENTAL.
Bacia hidrográfica do Ribeirão Pinheiros: relatório técnico.\\ São Paulo: CETESB,
1994. 39 p. \\\hline 
\end{tabular}\\

\textbf{Campos em LATEX:}

\begin{verbatim}
@Book{cetesb94,
Title                    = {Bacia hidrográfica do Ribeirão Pinheiros},
Address                  = {São Paulo},
Org-short                = {CETESB},
Organization             = {Companhia Estadual de Tecnologia de 
Saneamento Ambiental},
Owner                    = {apcalabrez},
Pages                    = {39},
Publisher                = {CETESB},
Subtitle                 = {relatório técnico},
Timestamp                = {2015.09.17},
Year                     = {1994}
}
\end{verbatim}

\textbf{Eventos}

Mencionar o nome completo do evento, desde que considerado no todo, seguido do ano de publicação.\\

\cite{iniciacao1996}

ou

\citeonline{iniciacao1996}\\

\textbf{Vários trabalhos de autores diferentes}

Indicar, em ordem alfabética, os sobrenomes dos autores seguidos de vírgula e data.\\

\cite{Farias2001,ROMANO1996,SEKEFF2002} 
	
ou

\citeonline{Farias2001,ROMANO1996,SEKEFF2002} \\


\section{Comandos em \LaTeX\ para citações}


No texto você deve inserir as citações com os comandos relacionados abaixo:

\begin{alineas}
\item
\begin{verbatim}
\cite
\end{verbatim}

Utilizado para inserir o sobrenome do autor dentro de parênteses seguido da informação do ano.

\textbf{Exemplos} 

\begin{verbatim}
\cite{ASPLUND2006}
\end{verbatim}
\cite{ASPLUND2006}

\begin{verbatim}
\cite{Paula2001}
\end{verbatim}
\cite{Paula2001}

\begin{verbatim}
\cite{Demakopoulou2000}
\end{verbatim}
\cite{Demakopoulou2000}

\begin{verbatim}
\cite{PhillipiJunior2000}
\end{verbatim}
\cite{PhillipiJunior2000}

\begin{verbatim}
\cite{resprin1997}
\end{verbatim}
\cite{resprin1997}

\begin{verbatim}
\cite{saopaulo1963}
\end{verbatim}
\cite{saopaulo1963}

\begin{verbatim}
\cite{resolucao1991}
\end{verbatim}
\cite{resolucao1991}

\begin{verbatim}
\cite{codigo1985}
\end{verbatim}
\cite{codigo1985}

\begin{verbatim}
\cite{constituicao1988}
\end{verbatim}
\cite{constituicao1988}

\begin{verbatim}
\cite{buscopan2013}
\end{verbatim}
\cite{buscopan2013}

\begin{verbatim}
\cite{Pasquarelli1987}
\end{verbatim}
\cite{Pasquarelli1987}\\

\item
\begin{verbatim}
\citeonline
\end{verbatim}

É utilizado quando você menciona explicitamente o autor da referência na sentença.

\textbf{Exemplos}

\begin{verbatim}
\citeonline{Novak1967}
\end{verbatim}
\citeonline{Novak1967}

\begin{verbatim}
\citeonline{Dood2002}
\end{verbatim}
\citeonline{Dood2002}

\begin{verbatim}
\citeonline{biblioteca1985}
\end{verbatim}
\citeonline{biblioteca1985}

\begin{verbatim}
\citeonline{usp2001}
\end{verbatim}
\citeonline{usp2001}

\begin{verbatim}
\citeonline{educacao2005}
\end{verbatim}
\citeonline{educacao2005}

\begin{verbatim}
\citeonline{brasil1981}
\end{verbatim}
\citeonline{brasil1981}

\begin{verbatim}
\citeonline{brasil1986}
\end{verbatim}
\citeonline{brasil1986}

\begin{verbatim}
\citeonline{Gomes1980}
\end{verbatim}
\citeonline{Gomes1980}\\

\item
\begin{verbatim}
\citeyear
\end{verbatim}

Apenas o \textbf{ano} da obra constará do texto, suprimindo-se os outros dados presentes na citação e os dados bibliográficos continuará constando da lista de referências. 

\textbf{Exemplos}

\begin{verbatim}
\citeyear{law1967}
\end{verbatim}
\citeyear{law1967}

\begin{verbatim}
\citeyear{Agencia2003}
\end{verbatim}
\citeyear{Agencia2003}

\begin{verbatim}
\citeyear{Dorlands2000}
\end{verbatim}
\citeyear{Dorlands2000}

\begin{verbatim}
\citeyear{abetter2004}
\end{verbatim}
\citeyear{abetter2004}

\begin{verbatim}
\citeyear{abetter2004}
\end{verbatim}
\citeyear{council2001}

\begin{verbatim}
\citeyear{Thome1999}
\end{verbatim}
\citeyear{Thome1999}

\begin{verbatim}
\citeyear{Nature1869}
\end{verbatim}
\citeyear{Nature1869}

\begin{verbatim}
\citeyear{Brennan2006}
\end{verbatim}
\citeyear{Brennan2006}

\begin{verbatim}
\citeyear{microsoft1995}
\end{verbatim}
\citeyear{microsoft1995}\\

\item
\begin{verbatim}
\citeauthor
\end{verbatim}

Apenas o \textbf{sobrenome do autor} da obra constará do texto em letras maiúsculas, suprimindo-se os outros dados presentes na citação e os dados bibliográficos continuará constando da lista de referências. 

\textbf{Exemplos}

\begin{verbatim}
\citeauthor{Vicente2010}
\end{verbatim}
\citeauthor{Vicente2010}

\begin{verbatim}
\citeauthor{Miyaura}
\end{verbatim}
\citeauthor{Miyaura}

\begin{verbatim}
\citeauthor{Piccini1996} 
\end{verbatim}
\citeauthor{Piccini1996} 

\begin{verbatim}
\citeauthor{Wendel1992}
\end{verbatim}
\citeauthor{Wendel1992}

\begin{verbatim}
\citeauthor{Elewa2006}
\end{verbatim}
\citeauthor{Elewa2006}

\begin{verbatim}
\citeauthor{Hofling1993}
\end{verbatim}
\citeauthor{Hofling1993}

%\begin{verbatim}
%\citeauthor{bule18}
%\end{verbatim}
%\cite{bule18}\\

\item
\begin{verbatim}
\citeauthoronline
\end{verbatim}

Apenas o \textbf{sobrenome do autor} da obra constará do texto, suprimindo-se os outros dados presentes na citação e os dados bibliográficos continuarão constando da lista de referências.

\textbf{Exemplos}

\begin{verbatim}
\citeauthoronline{Fonseca2000}
\end{verbatim}
\citeauthoronline{Fonseca2000}

\begin{verbatim}
\citeauthoronline{bibliotecanacional2000}
\end{verbatim}
\citeauthoronline{bibliotecanacional2000}

\begin{verbatim}
\citeauthoronline{Demakopoulou2000}
\end{verbatim}
\citeauthoronline{Demakopoulou2000}

\begin{verbatim}
\citeauthoronline{GlasscockIII1987}
\end{verbatim}
\citeauthoronline{GlasscockIII1987}

\begin{verbatim}
\citeauthoronline{delvecchio1995}
\end{verbatim}
\citeauthoronline{delvecchio1995}

\begin{verbatim}
\citeauthoronline{brasil1990}
\end{verbatim}
\citeauthoronline{brasil1990}

\begin{verbatim}
\citeauthoronline{Herbrick1989}
\end{verbatim}
\citeauthoronline{Herbrick1989}

\begin{verbatim}
\citeauthoronline{Mostafavi2014}
\end{verbatim}
\citeauthoronline{Mostafavi2014}\\

\item
\begin{verbatim}
\citetext
\end{verbatim}

Imprimi o conteúdo da referência de uma citação dentro do texto e também na lista de referências. Ao utilizar a macro  \verb+\citetext+ será transcrito o conteúdo da referência com a formatação padrão do documento, ou seja com espaçamento entre as linhas de 1,5 cm e na lista de referências com espaçamento simples.

\textbf{Exemplos}

\begin{verbatim}
\citetext{Lacasse2005}
\end{verbatim}

\citetext{Lacasse2005} \\

Para alterar o espaçamento entre linhas da referência para simples dentro do documento é necessário inserir o comando de formatação para espaços simples \verb+\SingleSpacing+ conforme abaixo:

\begin{verbatim}
\begin{SingleSpace} 
\citetext{Lacasse2005}
\end{SingleSpace}
\end{verbatim}

\begin{SingleSpace} 
	\citetext{Lacasse2005}
\end{SingleSpace}

Os exemplos abaixo estão formatados com espaçamento simples.

\begin{verbatim}
\begin{SingleSpace} 
\citetext{Palagachev2006}
\end{SingleSpace}
\end{verbatim}

\begin{SingleSpace} 
	\citetext{Palagachev2006}
\end{SingleSpace}

\begin{verbatim}
\begin{SingleSpace} 
\citetext{Zelen2000}
\end{SingleSpace}
\end{verbatim}

\begin{SingleSpace} 
	\citetext{Zelen2000}
\end{SingleSpace}

\begin{verbatim}
\begin{SingleSpace} 
\citetext{Boyd1993}
\end{SingleSpace}
\end{verbatim}

\begin{SingleSpace} 
	\citetext{Boyd1993}
\end{SingleSpace} 

\begin{verbatim}
\begin{SingleSpace} 
\citetext{Cochrane1998}
\end{SingleSpace}
\end{verbatim}

\begin{SingleSpace} 
	\citetext{Cochrane1998}
\end{SingleSpace} 

\begin{verbatim}
\begin{SingleSpace} 
\citetext{Oliveira2006}
\end{SingleSpace}
\end{verbatim}

\begin{SingleSpace} 
	\citetext{Oliveira2006}
\end{SingleSpace}

\begin{verbatim}
\begin{SingleSpace} 
\citetext{Harrison2001}
\end{SingleSpace}
\end{verbatim}

\begin{SingleSpace} 
	\citetext{Harrison2001}
\end{SingleSpace}

\begin{verbatim}
\begin{SingleSpace} 
\citetext{usp2006}
\end{SingleSpace}
\end{verbatim}

\begin{SingleSpace} 
	\citetext{usp2006}
\end{SingleSpace} 

\quad

\item
\begin{verbatim}
\Idem comando específico para mesmo autor
\Ibidem comando específico para mesma obra
\opcit comando específico para obra citada
\passim comando específico para aqui e alí
\loccit comando específico para no lugar citado
\cfcite comando específico para confira
\etseq comando específico para e sequencia 
\end{verbatim} 

As expressões latinas podem ser usadas para evitar repetições constantes de fontes citadas anteriormente. A primeira citação de uma obra deve apresentar sua referência completa e as subsequentes podem aparecer sob forma abreviada. Não usar destaque tipográfico quando utilizar expressões latinas. As expressões latinas não devem ser usadas no texto, apenas em nota de rodapé, exceto apud. A presença da referência em nota de rodapé não dispensa sua inclusão nas Referências, no final do trabalho. As expressões idem, ibidem, opus citatum, passim, loco citato, cf. e et seq. só podem ser usadas na mesma página ou folha da citação a que se referem. Para não prejudicar a leitura é recomendado evitar o emprego de expressões latinas.\\

\textbf{Exemplos}

\begin{verbatim}
\Idem[p.~491]{Abend2002}
\end{verbatim}
\Idem[p.~491]{Abend2002}

\begin{verbatim}
\Idem[p.~15]{tratados1999}
\end{verbatim}
\Idem[p.~15]{tratados1999}

\begin{verbatim}
\Idem[p.~18]{central1998}
\end{verbatim}
\Idem[p.~18]{central1998}

\begin{verbatim}
\Ibidem[p.~1]{Emenda1995}
\end{verbatim}
\Ibidem[p.~1]{Emenda1995}

\begin{verbatim}
\Ibidem[p.~15]{Paciornick1978}
\end{verbatim}
\Ibidem[p.~15]{Paciornick1978}

\begin{verbatim}
\Ibidem[p.~15]{atlas1981}
\end{verbatim}
\Ibidem[p.~35]{atlas1981}

\begin{verbatim}
\opcit[p.~23]{Denver1974}
\end{verbatim}
\opcit[p.~23]{Denver1974}

\begin{verbatim}
\opcit[p.~2]{Almeida1995}
\end{verbatim}
\opcit[p.~2]{Almeida1995}

\begin{verbatim}
\opcit[p.~3]{bionline}
\end{verbatim}
\opcit[p.~3]{bionline}

\begin{verbatim}
\passim{Villa-Lobos1916}
\end{verbatim}
\passim{Villa-Lobos1916}

\begin{verbatim}
\passim{Ramos1999}
\end{verbatim}
\passim{Ramos1999}

\begin{verbatim}
\passim{atlas2001}
\end{verbatim}
\passim{atlas2001}

\begin{verbatim}
\loccit{Wu1999}
\end{verbatim}
\loccit{Wu1999}

\begin{verbatim}
\loccit{Costa2002}
\end{verbatim}
\loccit{Costa2002}

\begin{verbatim}
\loccit{Geografico1986}
\end{verbatim}
\loccit{Geografico1986}

\begin{verbatim}
\cfcite[p.~2]{BRAYNER1994}
\end{verbatim}
\cfcite[p.~2]{BRAYNER1994}

\begin{verbatim}
\cfcite[p.~2]{Sabroza1998}
\end{verbatim}
\cfcite[p.~2]{Sabroza1998}

\begin{verbatim}
\cfcite[p.~46]{Oliva1900}
\end{verbatim}
\cfcite[p.~46]{Oliva1900}

\begin{verbatim}
\etseq[p.~2]{Montgomery1992}
\end{verbatim}
\etseq[p.~2]{Montgomery1992}

\begin{verbatim}
\etseq[p.~2]{Dudek2006}
\end{verbatim}
\etseq[p.~2]{Dudek2006}

\begin{verbatim}
\etseq[p.~2]{brasil1990b}
\end{verbatim}
\etseq[p.~2]{brasil1990b}

\end{alineas}





% ---
% Capítulo 4
% ---
%% USPSC-Cap3-Conclusao.tex
% Capítulo 3 - Conclusão
% ---
% Conclusão
% ---
\chapter{Conclusion and Future Work}
% ---
% O comando abaixo insere parágrafos aleatórios só para exemplificar
The very small standard deviations of principal components formation
(see Sections~\ref{sec:pca} and~\ref{prevalence}),
the presence of the Erd\"os sectors even in networks with
few participants (see Sections~\ref{sectioning} and~\ref{subsec:pih}),
and the recurrent activity patterns along different timescales (see Sections~\ref{sec:mtime} and~\ref{constDisc}),
go a step further in characterizing scale-free networks in the context
of the interaction of human individuals.
Furthermore, the importance of symmetry-related metrics,
which surpassed that of clustering coefficient,
with respect to dispersion of the system in the topological measures space,
might add to the current understanding of key-differences between digraphs and
undirected graphs in complex networks.
Noteworthy is also the very stable fraction participants in each Erd\"os sector when the network reaches more than 200 participants.
Benchmarks were derived from email list networks
and the supplied analysis of
networks from Facebook,
Twitter and Participabr in the Supporting Information might ease hypothesizing
about the generality of these characteristics.

Further work should expand the analysis to include
more types of networks and more metrics.
The data and software needed to attain these results
should also receive dedicated and in-depth
documentation as they enable a greater level of transparency
and work share,
which is adequate for both benchmarking
and specifically for the study of systems constituted
by human individuals (see Section~\ref{sec:data}).
The derived typology of hub, intermediary and peripheral participants
has been applied for semantic web and participatory democracy efforts,
and these developments might be enhanced to yield scientific knowledge~\cite{opa}.
Also, we plan to further explore and publish the audiovisualizations
used for this research~\cite{versinus,animacoes} and
the linguistic differences found in each of the Erd\"os sectors~\cite{rcText}.

 

% ---

% Capítulo 5 - Conclusão
% ---
% \include{USPSC-Cap5-Conclusao}
% ---

% ----------------------------------------------------------
% ELEMENTOS PÓS-TEXTUAIS
% ----------------------------------------------------------
\postextual
% ----------------------------------------------------------

% -----------------------------------------------------------
% Referências bibliográficas
% ----------------------------------------------------------
% \bibliography{USPSC-modelo-references}
\begin{thebibliography}{999}

\bibitem{moreno}
	MORENO, J. L. \textbf{Who shall survive?}: a new approach to the problem of human interrelations. Washington: Nervous and Mental Disease Publishing Co, 1934. (Nervous and mental disease monograph series, n. 58).

\bibitem{newmanBook}
	NEWMAN, M. \textbf{Networks}: an introduction. Oxford: Oxford University Press, 2010.

\bibitem{latour2013}
	LATOUR, B. Reassembling the social. an introduction to actor-network-theory. \textbf{Journal of Economic Sociology}, v. 14, n. 2, p. 73–87, 2013.

\bibitem{bird}
	BIRD, C. et al. Mining email social networks. In: INTERNATIONAL WORKSHOP ON MINING SOFTWARE REPOSITORIES. 2006, Shanghai. \textbf{Proceedings...} New York: ACM. p. 137–143, 2006.

\bibitem{barabasiHumanDyn}
	VÁZQUEZ, A. et al. Modeling bursts and heavy tails in human dynamics. \textbf{Physical Review E}, v. 73, n. 3, p. 036127, 2006.

\bibitem{newmanFriendship}
	BALL, B.; NEWMAN, M. E. \textbf{Friendship networks and social status}.  2012. Available from: \url{https://arxiv.org/pdf/1205.6822.pdf}. Accessible  at: 23 Jan. 2016.

\bibitem{barabasiEvo}
	PALLA, G.; BARABÁSI, A.-L.; VICSEK, T. Quantifying social group evolution. \textbf{Nature}, v. 446, n. 7136, p. 664–667, 2007.

\bibitem{newmanEvolving}
	LEICHT, E. A. et al.  Large-scale structure of time evolving citation networks. \textbf{European Physical Journal B}, v. 59, n. 1, p. 75–83, 2007.

\bibitem{access}
	TRAVENÇOLO, B.; COSTA, L. F. Accessibility in complex networks. \textbf{Physics Letters A}, v. 373, n. 1, p. 89–95, 2008.

\bibitem{newmanModularity}
	NEWMAN, M. E. Modularity and community structure in networks. \textbf{Proceedings of the National Academy of Sciences of the United States of America}, v. 103, n. 23, p. 8577–8582, 2006.

\bibitem{barabasiTopologicalEv}
	ALBERT, R.; BARABÁSI, A.-L. Topology of evolving networks: local events and universality. \textbf{Physical Review Letters}, v. 85, n. 24, p. 5234, 2000.

\bibitem{Gmane2}
	MAREK-SPARTZ, K.; CHESLEY, P.; SANDE, H. \textbf{Construction of the gmane corpus for examining the diffusion of lexical innovations}. 2012. Available from: \url{https://pdfs.semanticscholar.org/03c0/8a95695e39595267f786a1040cd002bfea1d.pdf?_ga=1.212330838.358458694.1488839238}. Accessible  at: 23 Jan. 2016.

\bibitem{stab}
	FABBRI, R. et al. \textbf{Temporal stability in human interaction networks}. 2013. Available from: \url{https://arxiv.org/pdf/1310.7769.pdf}. Accessible at: 23 Jan. 2016.

\bibitem{barabasiGeo}
	ONNELA, J.-P. et al. Geographic constraints on social network groups. \textbf{PLoS One}, v. 6, n. 4, p. e16939, 2011.

\bibitem{barabasiSex}
	PALCHYKOV, V. et al. Sex differences in intimate relationships. \textbf{Scientific Reports}, v. 2, 2012. doi:10.1038/srep00370.

\bibitem{gmanePack}
	FABBRI, R. \textbf{Python package to observe temporal stability in the GMANE database}. 2015. Available from: \url{https://github.com/ttm/percolation}. Accessible at 10 mar. 2016.

\bibitem{complexity}
	HOLLAND, J. H. \textbf{Complexity}: a very short introduction. Oxford: Oxford University Press, 2014.

\bibitem{penrose}
	PENROSE, R. \textbf{The road to reality}: a complete guide to the physical universe. London: Jonathan Cape, 2004.

\bibitem{luMeasures}
	COSTA, L. F. et al. Characterization of complex networks: a survey of measurements. \textbf{Advances in Physics}, v. 56, n. 1, p. 167–242, 2007.

\bibitem{erdosOrig}
	ERDÖS, P.; RÉNYI, A. On random graphs I. \textbf{Publicationes Mathematicae, v}. 6, p. 290–297, 1959.

\bibitem{customText}
	ORDENES, F. V. et al. Analyzing customer experience feedback using text mining: a linguistics-based approach. \textbf{Journal of Service Research}, v. 17, n. 3, p. 278–295, 2014.

\bibitem{textSurvey}
	GUPTA, V. et al. A survey of text mining techniques and applications. \textbf{Journal of Emerging Technologies in Web Intelligence}, v. 1, n. 1, p. 60–76, 2009.

\bibitem{nltk}
	BIRD, S.; KLEIN, E.; LOPER, E. \textbf{Natural language processing with Python}: analyzing text with the natural language toolkit. Beijing: O'Reilly, 2009.

\bibitem{llull}
	WILSON, R.; WATKINS, J. J. \textbf{Combinatorics}: ancient \& modern. Oxford: Oxford University Press, 2013.

\bibitem{eades}
	EADES, P.; KLEIN, K. \textbf{Graph visualization}. 2015. Available from: \url{http://edbt2015school.win.tue.nl/material/eades-klein.pdf}. Accessible at: 23 Jan. 2016.

\bibitem{fr}
	FRUCHTERMAN, T. M.; REINGOLD, E. M. Graph drawing by force-directed placement. \textbf{Software}: practice and experience, v. 21, n. 11, p. 1129–1164, 1991.

\bibitem{dynGraph}
	BECK, F. et al. \textbf{A taxonomy and survey of dynamic graph visualization}. 2016. Available from: \url{http://www.visus.uni-stuttgart.de/uploads/tx_vispublications/cgf-dynamicgraphs.pdf}. Accessible at: 23 Jan. 2016.

\bibitem{lee1}
	BERNERS-LEE, T. \textbf{Linked data}. 2006. Available from: \url{https://www.w3.org/DesignIssues/LinkedData.html}. Accessible at: 15 Mar. 2017.

\bibitem{uri}
	MASINTER, L.; BERNERS-LEE, T.; FIELDING, R. T. \textbf{Uniform resource identifier (uri)}: generic syntax. 2005. Available from: \url{https://tools.ietf.org/html/rfc3986}. Accessible at: 15 Mar. 2017.

\bibitem{lod}
	UMBRICH, J. et al. \textbf{Towards dataset dynamics}: change frequency of linked open data sources. 2010. Available from: \url{http://aidanhogan.com/docs/dynamics_ldow2010.pdf}. Accessible at: 23 Jan. 2016.

\bibitem{dbpedia0}
	BIZER, C. et al. DBpedia-a crystallization point for the web of data. \textbf{Web semantics}: science, services and agents on the world wide web, v. 7, n. 3, p. 154–165, 2009.

\bibitem{dbpedia}
	AUER, S. et al. DBpedia: a nucleus for a web of open data. In: ABERER, K. et al. (Ed.). \textbf{The semantic web}. Berlin: Springer, 2007. p. 722–735. (Lecture notes in computer science, v. 4825)

\bibitem{rdf}
	CYGANIAK, R.; WOOD, D.; LANTHALER, M. \textbf{RDF 1.1 concepts and abstract syntax}, 2014. Available from: \url{https://www.w3.org/TR/rdf11-concepts/}. Accessible at: 15 Mar. 2017.

\bibitem{losd}
	FABBRI, R.; OLIVEIRA JUNIOR, O. N. \textbf{Linked open social data for scientific benchmarking}. 2016. Available from: \url{https://github.com/ttm/linkedOpenSocialData/raw/master/paper.pdf}. Accessible at: 30 Oct. 2016.

\bibitem{pnud3}
	FABBRI, R. \textbf{United Nations Development Programme}: tools for content classification in the ParticipaBR Brazilian federal portal of social participation. Available from: \url{https://github.com/ttm/pnud3/blob/master/latex/produto.pdf?raw=true}. Accessible at: 30 Oct. 2016.

\bibitem{pnud4}
	FABBRI, R. \textbf{United Nations Development Programme}: adaptations and increments for the ParticipaBR Brazilian federal portal of social participation). Available from: \url{https://github.com/ttm/pnud4/blob/master/latex/produto.pdf?raw=true}. Accessible at: 30 Oct. 2016.

\bibitem{opa}
	FABBRI, R. \textbf{United Nations Development Programme}: content extraction through API from the Brazilian federal portal of social participation and its tools to a social participation cloud. 2014. Avaliable from: \url{https://github.com/ttm/pnud5/blob/master/latex/produto.pdf?raw=true}. Accessible at: 30 Oct. 2016.

\bibitem{dialogaAlg}
	FABBRI, R.; POPPI, R. \textbf{Continuous voting by approval and participation}. 2015. Available from: \url{https://arxiv.org/pdf/1505.06640.pdf}. Accessible at: 23 Jan. 2016.

\bibitem{mb}
	MACDAID, G. P.; MCCAULLEY, M. H.; KAINZ, R. I. \textbf{Myers-Briggs type indicator atlas of type tables}. Florida: Center for Applications of Psychological Type, 2005.

\bibitem{adorno}
	ADORNO, T. W. et al. \textbf{The authoritarian personality}. New York: Harpers, 1950.

\bibitem{anPhy}
	FABBRI, R. \textbf{What are you and I?} [Anthropological physics fundamentals]. 2015. Available from: \url{https://www.academia.edu/10356773/What_are_you_and_I_anthropological_physics_fundamentals_}. Accessible at: 23 Jan. 2016.

\bibitem{ccs15}
	ANTUNES, D.; FABBRI, R.; PISANI, M. M. Anthropological physics and social psychology in the critical research of networks. In: CONFERENCE ON COMPLEX SYSTEMS, 2015, Tempe. \textbf{Proceedings...} Arizona: CCS, 2015. Available from \url{https://www.youtube.com/watch?v=oeOKYc3-nbM}. Accessible at: 10 Mar. 2017.

\bibitem{phyAn}
	STANFORD, C.; ALLEN, J. S.; ANTÓN, S. C. \textbf{Biological anthropology}: the natural history of humankind. 4 th ed. New York: Pearson, 2016.

\bibitem{lee0}
	BERNERS-LEE, T.; HENDLER, J.; LASSILA, O. et al. The semantic web. \textbf{Scientific American}, v. 284, n. 5, p. 28–37, 2001.

\bibitem{textMining}
	WITTEN, I. H. \textbf{Text mining}. 2004. Available from: \url{http://www.cos.ufrj.br/~jano/LinkedDocuments/_papers/aula13/04-IHW-Textmining.pdf}. Accessible at: 06 Jan. 2015.

\bibitem{versinus}
	FABBRI, R. \textbf{Versinus}: a visualization method for graphs in evolution. 2013. Available from: \url{https://arxiv.org/pdf/1412.7311.pdf}. Accessible at: 30 Oct. 2016.

\bibitem{kolmSmir}
	FABBRI, R. \textbf{A distance metric between histograms derived from the Kolmogorov-Smirnov test statistic}: specification, measures reference and example uses. Available from: \url{https://github.com/ttm/kolmogorov-smirnov/raw/master/paper.pdf}. Accessible at: 10 Mar. 2017.

\bibitem{GMANEwikipedia}
	GMANE. Available from \url{http://en.wikipedia.org/wiki/Gmane}. Accessible at: 06 Jan. 2015.

\bibitem{netvizz}
	RIEDER, B. Studying facebook via data extraction: the netvizz application. In: ANNUAL ACM WEB SCIENCE CONFERENCE, 5., 2013. Paris. \textbf{Proceedings...} New York: ACM, 2013. p. 346–355.

\bibitem{Gmane}
	INGEBRIGTSEN, L. M. \textbf{Gmane}. 2008. Available from: \url{http://gmane.org/}. Accessible at: 10 Mar. 2017.

\bibitem{aaPaper}
	FABBRI, R. et al. The algorithmic autoregulation software development methodology/a metodologia de desenvolvimento de software autorregulação algorítmica. \textbf{Revista Eletrônica de Sistemas de Informação}, v. 13, n. 2, p. 1, 2014. doi: 10.5329/RESI.2014.1302002.

\bibitem{social}
	FABBRI, R. \textbf{The Social Python package to deliver social linked data from Participa.br, Cidade Democrática and AA}. 2015. Available from: \url{https://github.com/ttm/social}. Accessible at: 23 Jan. 2016.

\bibitem{datahub}
	FABBRI, R. \textbf{Data from Participa.br, Cidade Democrática and AA, in XML/RDF and Turtle/RDF}. 2014. Available from: \url{http://datahub.io/organization/socialparticipation}. Accessible at: 10 Mar. 2017.

\bibitem{openSci}
	WOELFLE, M.; OLLIARO, P.; TODD, M. H. Open science is a research accelerator. \textbf{Nature Chemistry}, v. 3, n. 10, p. 745–748, 2011.

\bibitem{directionalStats}
	MARDIA, K. V.; JUPP, P. E. \textbf{Directional statistics}. Chichester: John Wiley \& Sons, 2009.

\bibitem{newmanCommunityDirected}
	LEICHT, E.  A.; NEWMAN, M. E. Community structure in directed networks. \textbf{Physical Review Letters}, v. 100, n. 11, p. 118703, 2008.

\bibitem{newmanCommunity2013}
	NEWMAN, M. \textbf{Community detection and graph partitioning}.  2013. Available from: \url{https://arxiv.org/pdf/1305.4974.pdf}. Accessible at: 23 Jan. 2016.

\bibitem{faster}
	BRANDES, U. A faster algorithm for betweenness centrality. \textbf{Journal of Mathematical Sociology}, v. 25, n. 2, p. 163–177, 2001.

\bibitem{3setores}
	JACKSON, M. O. \textbf{Social and economic networks: models and analysis}. 2013. Available from: \url{https://class.coursera.org/networksonline-001}. Accessible at:  23 Jan. 2016.

\bibitem{pca}
	JOLLIFFE, I. \textbf{Principal component analysis}. New York: Wiley Online Library, 2005.

\bibitem{animacoes}
	FABBRI, R. \textbf{Video visualizations of email interaction network evolution}. 2013–5. Available from: \url{https://www.youtube.com/playlist?list=PLf_EtaMqu3jVodaqDjN7yaSgsQx2Xna3d}. Accessible at: 30 Oct. 2016.

\bibitem{Viz1}
	ELZEN, S. et al. Reordering massive sequence views: enabling temporal and structural analysis of dynamic networks. In: IEEE VISUALIZATION SYMPOSIUM (PACIFICVIS), 2013, Sydney. \textbf{Proceedings...} Sydney: IEEE, 2013. p. 33–40.

\bibitem{Viz3}
	KOOP, D.; FREIRE, J.; SILVA, C. T. Visual summaries for graph collections. In: IEEE PACIFIC VISUALIZATION SYMPOSIUM (PacificVis), 2013, Sydney. \textbf{Proceedings...} Sydney. IEEE, 2013. p. 57–64.

\bibitem{petrov}
	PETROV, S.; DAS, D.; MCDONALD, R. \textbf{A universal part-of-speech tagset}. 2011. Available from: \url{https://arxiv.org/pdf/1104.2086.pdf}. Accessible at: 13 Jan. 2015.

\bibitem{trabNLTK}
	FABBRI, R. \textbf{Resumo introdutório ao natural language toolkit (NLTK)}. 2013. Available from: \url{https://sourceforge.net/p/labmacambira/rcpln/ci/master/tree/pln/trabNLTK/resumoNLTK.pdf?format=raw}. Accessible at: 15 Mar. 2017.

\bibitem{wordnet}
	MILLER, G. A. Wordnet: a lexical database for english. \textbf{Communications of the ACM}, v. 38, n. 11, p. 39–41, 1995.

\bibitem{machado}
	FABBRI, R. \textbf{Incidência de letras, palavras e sentenças na obra de Machado de Assis}. 2013. Available from: \url{http://sourceforge.net/p/labmacambira/rcpln/ci/master/tree/pln/trabLetras/resumoLetras.pdf?format=raw}. Accessible at: 06 Jan. 2016.

\bibitem{kolm}
	KOLMOGOROV–SMIRNOV test. 2015. Available from: \url{https://en.wikipedia.org/w/index.php?title=Kolmogorov%E2%80%93Smirnov_test&oldid=682456076}. Accessible at: 26-Sept. 2015. 

\bibitem{secFree}
	BOCCALETTI, S. et al. Complex networks: Structure and dynamics. \textbf{Physics Reports}, v. 424, n. 4, p. 175–308, 2006.

\bibitem{textTables}
	FABBRI, R.; OLIVEIRA JUNIOR, O. N. \textbf{Tables of measurements of texts produced by each of the Erdös sectors}. 2017. Available from: \url{https://github.com/ttm/artigoTextoNasRedes/raw/master/supportingInformation.pdf}. Accessible at: 13 Jan. 2017.

\bibitem{mad}
	FABBRI, R. \textbf{Música no áudio digital}: descrição psicofísica e caixa de ferramentas. 2013. 253 p. Dissertação (Mestrado em Ciências) – Instituto de Física de São Carlos, Universidade de São Paulo, São Carlos, 2013. Available from: \url{http://www.teses.usp.br/teses/disponiveis/76/76132/tde-19042013-095445/}. Accessible at: 15 Mar. 2017.

\bibitem{participation}
	FABBRI, R. \textbf{The participation Python toolbox for rendering linked data from ParticipaBR, Cidade Democrática and AA}. 2015. Available from: \url{https://github.com/ttm/participation}. Accessible at: 10 Mar. 2017.

\bibitem{music}
	FABBRI, R. \textbf{The music Python toolbox for rendering music}. 2015. Available from: \url{https://github.com/ttm/music}. Accessible at: 10 Mar. 2017.

\bibitem{visuals}
	FABBRI, R. \textbf{The visuals toolbox for rendering visualizations of networks}. 2015. Available from: \url{https://github.com/ttm/visuals}. Accessible at: 10 Mar. 2017.

\bibitem{gmaneLegacy}
	FABBRI, R. \textbf{The Gmane legacy repository}. 2015. Available from: \url{https://github.com/ttm/gmaneLegacy}. Accessible at: 10 Mar. 2017.

\bibitem{percolationLegacy}
	FABBRI, R. \textbf{The percolation legacy repository}. 2015. Available from: \url{https://github.com/ttm/percolationLegacy}. Accessible at: 15 Mar. 2017.

\bibitem{ensaio}
	FABBRI, R. \textbf{Ensaio sobre o auto aproveitamento}: um relato de investidas naturais na participação social. 2014. Available from: \url{https://arxiv.org/pdf/1412.6868.pdf}. Accessible at: 10 Mar. 2017.

\bibitem{ops}
	FABBRI, R. et al. \textbf{Social participation ontology}: community documentation, enhancements and use examples. 2015. Available from: \url{https://arxiv.org/pdf/1501.02662.pdf}. Accessible at: 10 Mar. 2017.

\bibitem{rcText}
	FABBRI, R. \textbf{A connective differentiation of textual production in interaction networks}. 2013. Available from: \url{https://arxiv.org/pdf/1412.7309.pdf}. Accessible at: 10 Mar. 2017.

\bibitem{easly}
	EASLY, D.; KLEINBERG, J. \textbf{Networks, crowds, and markets}: reasoning about a highly connected world, Cambridge: Cambridge University Press, 2010.

\bibitem{coh}
	GRAESSER, A. C.; MCNAMARA, D. S.; KULIKOWICH, J. M. Coh-metrix: providing multilevel analyses of text characteristics, \textbf{Publications Sage}, v. 40, n. 5, p. 223–234, 2011.

\bibitem{dialoga}
	DIALOGA Brasil: a federal social participation platform. Available from: \url{http://dialoga.gov.br}. Accessible at: 15 Mar. 2017.

\bibitem{autoRede}
	FABBRI, R. \textbf{Software for online making email interaction network images, gml files and measurements}. 2013. Available from: \url{https://sourceforge.net/p/labmacambira/fimDoMundo/ci/master/tree/python/autoRede/}. Accessible at: 15 Mar. 2017.

\bibitem{gephi}
	BASTIAN, M. et al. \textbf{Gephi}: an open source software for exploring and manipulating networks. 2009. Available from: \url{https://gephi.org/publications/gephi-bastian-feb09.pdf}. Accessible at 23 Jan. 2016.

\bibitem{fa2}
	JACOMY, M. et al. Force atlas2, a continuous graph layout algorithm for handy network visualization designed for the gephi software. \textbf{PloS One}, v. 9, n. 6, p. e98679, 2014.

\bibitem{pprGal}
	FABBRI, R.; ROCHA, P. P. \textbf{Artistic elaborations over network visualizations}. 2013–4. Available from: \url{https://www.facebook.com/renato.fabbri/media_set?set=a.10152260246819430.781909429&type=3}. Accessible at: 15 Mar. 2017.

\bibitem{teloes}                                                                             
	FABBRI, R. \textbf{Source code for the social structures live streaming screens}. (networks and language features). Available from: \url{https://github.com/ttm/indicadores-participativos/tree/master/meteorfront/ex5}. Accessible at: 15 Mar. 2017.

\bibitem{ubiIne}
	FABBRI, R.; OLIVEIRA JUNIOR, O. N. \textbf{A simple model that explains why inequality is ubiquitous}. 2017. Available from: \url{https://github.com/ttm/ubiquitousInequality/raw/master/essay.pdf}. Accessible at: 15 Mar. 2017.

\bibitem{eqFree}
	FABBRI, R. \textbf{Three equanimous aspects of scale-free networks}. 2015. Available from: \url{https://github.com/ttm/equanimousScaleFree}. Accessible at: 15 Mar. 2017.

\bibitem{ensaaio}
	FABBRI, R. \textbf{The algorithmic-autoregulation essay}: a collective and natural focus on self-transparency. 2015. Available from: \url{https://github.com/ttm/ensaaio}. Accessible at: 15 Mar. 2017.

\bibitem{opa0}
	FABBRI, R. \textbf{United Nations Development Programme}: primeira ontologia do portal federal de participação social. Descrição e código OWL: ontologia do ParticipaBR. 2014. Available from: \url{https://sourceforge.net/p/labmacambira/fimDoMundo/ci/master/tree/textos/ontologia/ontologiaParticipa_.pdf?format=raw}. Accessible at: 15 Mar. 2017.

\bibitem{ocd}
	FABBRI, R. \textbf{Cidade democrática ontology and triplification routines}. 2014. Available from: \url{https://github.com/OpenLinkedSocialData/ocd}. Accessible at: 15 Mar. 2017.

\bibitem{vocabP}
	FABBRI, R. \textbf{Ontologia e vocabulário da biblioteca de participação social}. 2014. Available from: \url{https://github.com/ttm/vocabulario-participacao}. Accessible at: 15 Mar. 2017.

\bibitem{caixamagica}
	FABBRI, R. \textbf{Magic box social participation ontology and visualization scripts}. 2015. Available from: \url{https://github.com/ttm/caixamagica}. Accessible at: 15 Mar. 2017.

\bibitem{OT}
	FABBRI, R. \textbf{Ontologia do trabalho}. 2015. Available from: \url{https://github.com/ttm/OT}. Accessible at: 15 Mar. 2017.

\bibitem{ore}
	FABBRI, R. \textbf{ORe}: Ontology of the research. 2015. Available from: \url{https://github.com/ttm/ORe}. Accessible at: 15 Mar. 2017.

\bibitem{comp1}
	FABBRI, R. et al. Analise de redes sociais complexas por correio eletronico. 2013. In: ENCONTRO DE MODELAGEM COMPUTACIONAL, 16, 2013, Ilhéus.  \textbf{Anais eletrônicos...} Ilhèus: UESC, 2013. Available from: \url{https://sourceforge.net/p/labmacambira/fimDoMundo/ci/master/tree/artigo/metaMails-submitted-emc2013.pdf?format=raw}. Accessible at: 15 Mar. 2017.

\bibitem{pnud2}
	FABBRI, R. \textbf{United Nations Development Programme}: ParticipaBR context, triplification of data and example of usage. Available from: \url{http://sourceforge.net/p/labmacambira/fimDoMundo/ci/master/tree/textos/SparQL/triplificaDisponibiliza.pdf?format=raw}. Accessible at: 15 Mar. 2017.

\bibitem{pnudExtra}
	FABBRI, R. \textbf{United Nations Development Programme}: notes on reading products from other consultants. Available from: \url{https://github.com/ttm/pnudExtra/raw/master/produto.pdf}. Accessible at: 15 Mar. 2017.

\bibitem{sm}
	FABBRI, R. \textbf{Sistematização dos usos de TI para monitoramento das redes sociais}. 2014. Brazilian presidential memorandum. Available from: \url{https://dl.dropboxusercontent.com/u/22209842/presidencia/20141707_anexo_Memorando_maquina_sistema_monitoramento.pdf}. Accessible at: 15 Mar. 2017.

\bibitem{participaPNPS}
	CONSTRUÇÃO da Política e do Compromisso Nacional para a Participação Social. 2013. ParticipaBR. Available from: \url{http://www.participa.br/participacaosocial/construcao-da-politica-e-do-compromisso-nacional-para-a-participacao-social}. Accessible at: 15 Mar. 2017.

\bibitem{analisePNPS}
	FABBRI, R. \textbf{Scripts for analyzing the social networks considering the National Plan and Commitment for Social Participation}. 2014. Available from: \url{https://github.com/ttm/analisePNPS}. Accessible at: 15 Mar. 2017.

\bibitem{pcPS}
	FABBRI, R. \textbf{Scripts for analyzing the Decree 8.243 of the Plan for Social Participation and a related commitment declaration}. 2013. Available from: \url{https://github.com/ttm/politica_compromisso_ps}. Accessible at: 15 Mar. 2017.

\bibitem{lmPS}
	FABBRI, R. \textbf{2 instâncias federais de participação social}: caracterização, observações e possibilidades (PNPS e CNPS). 2013. Available from: \url{https://sourceforge.net/p/labmacambira/rcpln/ci/master/tree/participacaoSocial/instanciasFederaisVirtuais.pdf?format=raw}. Accessible at: 15 Mar. 2017.

\bibitem{oscEmRede}
	FABBRI, R; ROCHA, R. \textbf{Temporal}: an online gadget for network visualization data from civil organizations (Siconv data). Available from: \url{https://github.com/ttm/oscEmRede}. Accessible at: 15 Mar. 2017.

\bibitem{votoTwitter}
	FABBRI, R. \textbf{A small gadget for public voting through Twitter}. 2014. Available from: \url{https://github.com/ttm/votacaoTwitter}. Accessible at: 15 Mar. 2017.

\bibitem{hackmar}
	FABBRI, R. \textbf{Linked data from the Rio de Janeiro Art Museum (MAR) collection}. 2015.  Available from: \url{https://github.com/ttm/hackmar}. Accessible at: 15 Mar. 2017.

\bibitem{myLattes}
	RENATO Fabbri Lattes curriculum page. 2016. Available from: \url{http://lattes.cnpq.br/1840472218825589}. Accessible at: 15 Mar. 2017.

\bibitem{50uni}
	FABBRI, R.; PENALVA, D.; PISANI, M. M. Art, technology and politics: perspectives from complex networks. In: SIMPÓSIO HOMENAGEM AOS 50 ANOS DE O HOMEM UNIDIMENSIONAL, DE HERBERT MARCUSE, 2014, Santo André. \textbf{Resumos...} Santo André: UFABC, 2014.

\bibitem{IICri}
	FABBRI, R. Workshop in complex networks. 2016. In: INTERNATIONAL SYMPOSIUM OF CRITICAL THEORY, 2., 2016, Sobral. \textbf{Proceedings...} Sobral: UFC, 2016. Available from: \url{http://www.nexos.ufc.br/encontro2016/index.php/evento/transmissao-ao-vivo?id=142}. Accessible at: 15 Mar. 2017.

\bibitem{aars}
	FABBRI, R. \textbf{Análise em ação para redes sociais (Govern Art)}. Available from: \url{https://github.com/ttm/aars}. Accessible at: 15 Mar. 2017.

\bibitem{rilke}
	RILKE, R. M. \textbf{Primal sound \& other prose pieces}. Madison: The Cummington press, 1943.

\bibitem{idealIdeas}
	FABBRI, R. \textbf{O pensamento ideal}: uma descrição física do pensamento. 2013. Available from: \url{https://sourceforge.net/p/labmacambira/rcpln/ci/master/tree/pensamento/pensamento.pdf?format=raw}. Accessible at: 15 Mar. 2017.

\bibitem{sfARS}
	FABBRI, R. \textbf{Crowdfunding and discussions about the analysis of Brazilian networks related to free culture}. 2013. Available from: \url{http://labmacambira.sourceforge.net/redes/}. Accessible at: 15 Mar. 2017.

\bibitem{rfARS}
	FABBRI, R. \textbf{Complex networks and natural language processing collection and diffusion of information and goods}. 2013. Available from: \url{http://wiki.nosdigitais.teia.org.br/ARS}. Accessible at: 15 Mar. 2017.

\bibitem{nuvens}
	FABBRI, R. \textbf{Nuvens cognitivas e a unificação da espécie humana}. 2014. Available from: \url{http://cyberiun.tumblr.com/post/64607669759/nuvens-cognitivas}.  Available from: \url{http://wiki.nosdigitais.teia.org.br/Cyberiun}. Accessible at: 15 Mar. 2017.

\bibitem{aaclient}
	FABBRI, R. \textbf{Minimal algorithmic autoregulation interface for displaying shouts}. 2012. Available from: \url{http://aaserver.herokuapp.com/minimumClient/}. Accessible at: 15 Mar. 2017.

\bibitem{ttmio}
	FABBRI, R. \textbf{Homepage}. 2015.  Available from: \url{http://ttm.github.io/}. Accessible at: 15 Mar. 2017.

\bibitem{docDif}
	FABBRI, R. \textbf{A document reporting the progressive diffusion of information from peripherals to hubs}. 2013. Available from: \url{https://dl.dropboxusercontent.com/u/22209842/doc/mit/progressiveDiffusion.pdf}. Accessible at: 15 Mar. 2017.

\bibitem{anExp}
	FABBRI, R. \textbf{Notes, data and a preliminary analysis script of a massive Facebook tagging social experiment}. 2015. Available from: \url{https://github.com/ttm/anthropologicalExperiments}. Accessible at: 15 Mar. 2017.

\bibitem{servidor}
	FABBRI, R. \textbf{SERVDDCR}: sociedade e estado em reunião virtual para a democracia direta ou conectiva ou rolezinho. 2014. Available from: \url{https://www.facebook.com/groups/SERVDDCR/}. Accessible at: 15 Mar. 2017.

\bibitem{anPhy2}
	FABBRI, R. \textbf{Anthropological physics repository with the essay and notes}. 2015. Available from: \url{https://github.com/ttm/anthropologicalPhysics}. Accessible at: 15 Mar. 2017.

\bibitem{trabBots}
	FABBRI, R. \textbf{Descrição dos bots no canal IRC \#labmacambira \@ Freenode}. 2013. Available from: \url{https://sourceforge.net/p/labmacambira/rcpln/ci/master/tree/pln/trabBots/trabBots.pdf?format=raw}. Accessible at: 15 Mar. 2017.

\bibitem{ovo}
	FABBRI R.; VIEIRA, V. \textbf{Ouvir para olhar (OvO) meetings scripts and documentation}. 2015. Available from: \url{https://github.com/ttm/ovo}. Accessible at: 15 Mar. 2017.

\bibitem{gradus}
	FABBRI, R. \textbf{Complex networks gradus ad Parnassum}. 2015. Available from: \url{https://github.com/ttm/gradus}. Accessible at: 15 Mar. 2017.

\bibitem{ssl}
	FABBRI, R. \textbf{Mean-cut and label propagation graph-based semi-supervised learning implementations and documentation}. 2010. Available from: \url{https://github.com/ttm/aprendizadoSemiSupervisionado}. Accessible at: 15 Mar. 2017.

\bibitem{microcontos}
	FABBRI, R. \textbf{Microcontos}. 2010–5. Available from: \url{https://github.com/ttm/microcontos}. Accessible at: 15 Mar. 2017.

\bibitem{pingo}
	FABBRI, R. \textbf{A sound effects bank for the Pingo game}. 2015. Available from: \url{https://github.com/ttm/pingosom}. Accessible at: 15 Mar. 2017.

\bibitem{sonhos}
	FABBRI R.; BORGES. F. \textbf{Analysis of dreams by text mining for schizoanalysis}. 2016. Available from: \url{https://github.com/ttm/sonhos}. Accessible at: 15 Mar. 2017.

\bibitem{tokiio}
	FABBRI, R. \textbf{Tokisona}: an account on learning Toki Pona and making some translations. 2016. Available from: \url{http://tokisona.github.io/}. Accessible at: 15 Mar. 2017.

\bibitem{tokir}
	FABBRI, R. \textbf{Scripts for exploring and syntax highlighting Toki Pona}. 2016. Available from: \url{https://github.com/ttm/tokipona}. Accessible at: 15 Mar. 2017.

\bibitem{vivacecmj}
	VIEIRA, V. et al. \textbf{Vivace}: a collaborative live coding language. 2015. Available from: \url{https://arxiv.org/abs/1502.01312}. Accessible at: 15 Mar. 2017.

\bibitem{vivacecode}
	VIEIRA, V.; FABBRI, R. \textbf{Code and documentation for the Vivace live coding platform}. 2013. Available from: \url{https://github.com/automata/vivace}. Accessible at: 15 Mar. 2017.

\bibitem{freakManifesto}
	VIEIRA, V. et al. \textbf{Freakcoding manifesto}. 2013. Quiosque editora. Available from: \url{http://void.cc/freakcoding}. Accessible at: 15 Mar. 2017.

\bibitem{chuLivro}
	SCHUSTER, E.; LEVKOWITZ, H.; OLIVEIRA JUNIOR, O. N. \textbf{Writing scientific papers in english successfully}: your complete roadmap. São Carlos: Compacta, 2014.

\bibitem{sciStyle}
	FABBRI, R. \textbf{The scientific style}. 2015. Available from: \url{https://github.com/ttm/sciStyle}. Accessible at: 15 Mar. 2017.

\bibitem{cartaML}
	FABBRI, R. \textbf{Carta mídias livres}. 2010. Available from: \url{https://github.com/ttm/cartaML/raw/master/carta-de-indicacoes-Midias-Livres3b.pdf}. Accessible at: 15 Mar. 2017.

\bibitem{ccd}
	DIGITAL counter culture. independent publishing, 2012. Available from: \url{http://culturadigital.br/contraculturadigital/files/2012/02/contraculturadigital_publcia_v3.pdf}. Accessible at: 15 Mar. 2017.

\bibitem{subMid}
	SUBMIDIALOGY. independent publishing, 2010. Available from: \url{https://catahistorias.files.wordpress.com/2012/07/peixemorto.pdf}. Available from: \url{https://catahistorias.files.wordpress.com/2012/07/peixemorto.pdf}. Accessible at: 15 Mar. 2017.

\bibitem{insp}
	FABBRI, R.; CHRIST, J.; TARSUS, P. of. \textbf{Instrumental Spirituality}. 2015. Available from: \url{https://www.academia.edu/11784001/Espiritualidade_Instrumental}. Accessible at: 19 Mar. 2017.

\end{thebibliography}

% ----------------------------------------------------------
% Glossário
% ----------------------------------------------------------
%
% Consulte o manual da classe abntex2 para orientações sobre o glossário.
%
%\glossary

% ----------------------------------------------------------
% Apêndices
% ----------------------------------------------------------
%% USPSC-Apendice.tex
% ---
% Inicia os apêndices
% ---

\begin{apendicesenv}
	% Imprime uma página indicando o início dos apêndices
	\partapendices
	\chapter{Additional tables of the textual differences found in all networks}\label{ap:textd}

\begin{table}[h!]
\begin{center}
\caption{Counts of evidence of difference in the Erd\"os sectors in each of the analyzed networks.}
	\def\arraystretch{1.5}
\begin{tabular}{| l || c | c | c || c | c | c |}\hline
{\bf synset} & {\bf p.} & {\bf i.} & {\bf h} & {\bf peaks} & {\bf total} & {\bf depth} \\\hline\hline
$\frac{punct}{chars-spaces}$ & 11  & 4  & 1  & 5  & 17  & 1 \\\hline
$\frac{vowels}{letters}$ & 0  & 1  & 1  & 1  & 18  & 1 \\\hline
$\frac{spaces}{chars}$ & 2  & 0  & 8  & 2  & 18  & 1 \\\hline
$\frac{letters}{chars-spaces}$ & 0  & 0  & 3  & 0  & 18  & 1 \\\hline
$\frac{digits}{chars-spaces}$ & 4  & 1  & 0  & 2  & 5  & 1 \\\hline
$\frac{uppercase}{letters}$ & 10  & 3  & 1  & 6  & 15  & 1 \\\hline
\end{tabular}
\begin{flushleft}
		Source: Prepared by the authors.\
\end{flushleft}
\end{center}
\end{table}

\begin{table}[h!]
\begin{center}
\caption{Counts of evidence of token-related differences in the Erd\"os sectors in each of the analyzed networks.}
	\def\arraystretch{1.5}
\begin{tabular}{| l || c | c | c || c |}\hline
{\bf synset} & {\bf p.} & {\bf i.} & {\bf h} & {\bf peaks} \\\hline\hline
$\frac{knownw}{tokens}$ & 1  & 0  & 5  & 1 \\
$\frac{knownw \neq}{knownw}$ & 13  & 1  & 4  & 9 \\
$\frac{stopw}{knownw}$ & 0  & 0  & 14  & 2 \\
$\frac{punct}{tokens}$ & 10  & 3  & 1  & 3 \\
$\frac{contrac}{tokens}$ & 0  & 2  & 15  & 4 \\\hline
$\mu(\overline{tokens})$ & 0  & 1  & 2  & 1 \\
$\sigma(\overline{tokens})$ & 7  & 1  & 0  & 2 \\\hline
$\mu(\overline{knownw})$ & 0  & 0  & 2  & 0 \\
$\sigma(\overline{knownw})$ & 0  & 0  & 1  & 1 \\\hline
$\mu(\overline{knownw \neq})$ & 0  & 0  & 0  & 0 \\
$\sigma(\overline{knownw \neq})$ & 0  & 0  & 0  & 0 \\\hline
$\mu(\overline{stopw})$ & 0  & 0  & 0  & 0 \\
$\sigma(\overline{stopw})$ & 0  & 0  & 1  & 0 \\\hline
\end{tabular}
\begin{flushleft}
		Source: By the author.\
\end{flushleft}
\end{center}
\end{table}

\begin{table}[h!]
\begin{center}
\caption{Counts of evidence of sentence-related differences in the Erd\"os sectors in each of the analyzed networks.}
	\def\arraystretch{1.5}
\begin{tabular}{l || c | c | c || c}\hline
{\bf synset} & {\bf p.} & {\bf i.} & {\bf h} & {\bf peaks} \\\hline\hline
$\mu_S(chars)$ & 9  & 3  & 1  & 6 \\
$\sigma_S(chars)$ & 11  & 6  & 1  & 9 \\\hline
$\mu_S(tokens)$ & 10  & 2  & 1  & 5 \\
$\sigma_S(tokens)$ & 9  & 7  & 1  & 9 \\\hline
$\mu_S(knownw)$ & 9  & 3  & 2  & 6 \\
$\sigma_S(knownw)$ & 11  & 5  & 2  & 8 \\\hline
$\mu_S(stopw)$ & 2  & 3  & 7  & 7 \\
$\sigma_S(stopw)$ & 6  & 7  & 4  & 10 \\\hline
$\mu_S(puncts)$ & 13  & 2  & 1  & 2 \\
$\sigma_S(puncts)$ & 7  & 8  & 1  & 8 \\\hline
\end{tabular}
\begin{flushleft}\footnotesize
		Source: By the author.\
\end{flushleft}
\end{center}
\end{table}

\begin{table}[h!]
\begin{center}
\caption{Counts of evidence of message-related differences in the Erd\"os sectors in each of the analyzed networks.}
	\def\arraystretch{1.5}
\begin{tabular}{| l || c | c | c || c | c |}\hline
{\bf synset} & {\bf p.} & {\bf i.} & {\bf h} & {\bf peaks} & {\bf total} \\\hline\hline
$\mu_M(sents)$ & 4  & 7  & 1  & 9  & 16 \\
$\sigma_M(sents)$ & 5  & 7  & 2  & 11  & 15 \\\hline
$\mu_M(tokens)$ & 10  & 5  & 2  & 6  & 18 \\
$\sigma_M(tokens)$ & 8  & 8  & 2  & 9  & 18 \\\hline
$\mu_M(knownw)$ & 8  & 5  & 3  & 7  & 18 \\
$\sigma_M(knownw)$ & 10  & 5  & 3  & 9  & 18 \\\hline
$\mu_M(stopw)$ & 5  & 6  & 6  & 8  & 18 \\
$\sigma_M(stopw)$ & 7  & 6  & 3  & 11  & 18 \\\hline
$\mu_M(puncts)$ & 12  & 4  & 2  & 5  & 18 \\
$\sigma_M(puncts)$ & 8  & 9  & 1  & 10  & 18 \\\hline
$\mu_M(chars)$ & 10  & 5  & 2  & 6  & 18 \\
$\sigma_M(chars)$ & 9  & 7  & 2  & 8  & 18 \\\hline
\end{tabular}
\begin{flushleft}
		Source: Prepared by the authors.\
\end{flushleft}
\end{center}
\end{table}

\begin{table}[h!]
\begin{center}
\caption{Counts of evidence of differences related to POS tags in the Erd\"os sectors in each of the analyzed networks.}
\begin{tabular}{| l || c | c | c || c |}\hline
{\bf synset} & {\bf p.} & {\bf i.} & {\bf h} & {\bf peaks} \\\hline\hline
NOUN & 13  & 1  & 0  & 1 \\
X & 4  & 9  & 5  & 14 \\\hline
ADP & 0  & 1  & 4  & 1 \\
DET & 1  & 0  & 9  & 2 \\\hline
VERB & 0  & 0  & 6  & 1 \\\hline
ADJ & 1  & 2  & 6  & 2 \\
ADV & 0  & 0  & 17  & 1 \\\hline
PRT & 1  & 1  & 9  & 4 \\
PRON & 0  & 1  & 11  & 3 \\
NUM & 8  & 5  & 3  & 7 \\
CONJ & 2  & 6  & 4  & 8 \\\hline
\end{tabular}
\begin{flushleft}\footnotesize
		Source: By the author.\
\end{flushleft}
\end{center}
\end{table}

\begin{table}[h!]
\begin{center}
\caption{Counts of evidence of differences related to Wordnet POS tags in the Erd\"os sectors in each of the analyzed networks.}
\begin{tabular}{l || c | c | c || c}\hline
{\bf synset} & {\bf p.} & {\bf i.} & {\bf h} & {\bf peaks} \\\hline\hline
N & 8  & 1  & 0  & 1 \\
ADJ & 0  & 2  & 12  & 6 \\
VERB & 0  & 1  & 16  & 2 \\
ADV & 0  & 0  & 9  & 1 \\\hline\hline
POS & 0  & 0  & 3  & 1 \\
POS! & 0  & 1  & 0  & 1 \\\hline
\end{tabular}
\begin{flushleft}\footnotesize
		Source: By the author.\
\end{flushleft}
\end{center}
\end{table}

\begin{table}[h!]
\begin{center}
\begin{tabular}{| l || c | c | c || c |}\hline
{\bf synset} & {\bf p.} & {\bf i.} & {\bf h} & {\bf peaks} \\\hline\hline
$\mu(min\,depth)$ & 0  & 0  & 0  & 0 \\
$\sigma(min\,depth)$ & 1  & 1  & 2  & 1 \\\hline
$\mu(max\,depth)$ & 0  & 0  & 0  & 0 \\
$\sigma(max\,depth)$ & 0  & 1  & 3  & 1 \\\hline
$\mu(holonyms)$ & 7  & 4  & 4  & 6 \\
$\sigma(holonyms)$ & 3  & 4  & 7  & 6 \\\hline
$\mu(meronyms)$ & 8  & 5  & 3  & 7 \\
$\sigma(meronyms)$ & 12  & 4  & 2  & 9 \\\hline
$\mu(domains)$ & 6  & 4  & 5  & 8 \\
$\sigma(domains)$ & 3  & 1  & 4  & 3 \\\hline
$\mu(lemmas)$ & 6  & 0  & 1  & 2 \\
$\sigma(lemmas)$ & 6  & 2  & 2  & 4 \\\hline
$\mu(hyponyms)$ & 1  & 6  & 6  & 9 \\
$\sigma(hyponyms)$ & 4  & 6  & 6  & 11 \\\hline
$\mu(hypernyms)$ & 0  & 0  & 0  & 0 \\
$\sigma(hypernyms)$ & 4  & 4  & 4  & 6 \\\hline
\end{tabular}
\caption{Counts of evidence of differences related to Wordnet noun synset characteristics in the Erd\"os sectors in each of the analyzed networks.}
\end{center}
\end{table}
\end{apendicesenv}


% ----------------------------------------------------------
% Anexos
% ----------------------------------------------------------
% \include{USPSC-Anexos}

%---------------------------------------------------------------------
% INDICE REMISSIVO
%--------------------------------------------------------------------
\phantompart
\printindex

%---------------------------------------------------------------------

\end{document}
