%% USPSC-Introducao.tex

% ----------------------------------------------------------
% Introdução (exemplo de capítulo sem numeração, mas presente no Sumário)
% ----------------------------------------------------------
% \chapter[Introdução]{Introdução}
\chapter{Introduction and Related Work}\label{ch:int}

The first studies dealing explicitly with human interaction networks
date from the nineteenth century while the foundation of
social network analysis is generally attributed to the psychiatrist Jacob Moreno in mid twentieth century~\cite{moreno,newmanBook}.
With the increasing availability of data related to human interactions, research about these networks has grown continuously.
Contributions can now be found in a variety of fields, from social sciences and humanities~\cite{latour2013} to computer science~\cite{bird} and physics~\cite{barabasiHumanDyn,newmanFriendship}, given the multidisciplinary nature of the topic.
One of the approaches from an exact science perspective is to represent interaction networks as complex networks~\cite{barabasiHumanDyn,newmanFriendship}, with which 
several features of human interaction have been revealed.
For example, the topology of human interaction networks exhibits a scale-free outline,
which points to the existence of a small number of highly connected hubs and a large number of poorly connected nodes.
The dynamics of complex networks representing human interaction has also been addressed~\cite{barabasiEvo,newmanEvolving}, but only to a limited extent, since research is normally focused on a particular metric or task, such as accessibility or community detection~\cite{access,newmanModularity}. 

There are numerous articles, books, websites and software tools about complex and social networks and about text mining in social media.
There are fewer endeavours to characterize these networks beyond general features such as the scale-free 
aspect or to deal with text produced by social networks from the complex networks background.
Research on network evolution is often restricted to network growth, in which there is a monotonic increase in the number of events~\cite{barabasiEvo}.
Network types have been discussed with regard to the number of participants, intermittence of their activity and network longevity~\cite{barabasiEvo}. Two topologically different networks emerged from human interaction networks, depending on whether the frequency of interactions follows a generalized power law or an exponential connectivity distribution~\cite{barabasiTopologicalEv}. In email list networks, scale-free properties were reported with $\alpha \approx 1.8$~\cite{bird} (as in web browsing and library loans~\cite{barabasiHumanDyn}), and different linguistic traces were related to weak and strong ties~\cite{Gmane2}.

The fact that unreciprocated edges often exceed 50\% in human interaction networks~\cite{newmanEvolving} motivated the inclusion of symmetry metrics in our analysis.
No correlation of topological characteristics and geographical coordinates was found~\cite{barabasiGeo},
therefore geographical positions were not considered in our study.
Gender related behavior in mobile phone datasets was indeed reported~\cite{barabasiSex}
but it is not relevant for the present work because email messages and addresses have no gender related metadata~\cite{gmanePack}.

\section{Related knowledge}
\subsection{Complex networks}
\subsection{Text mining of social data}

\subsection{Visualization of graphs}
\subsection{Linked data}
The fields of social network analysis and complex networks
are widely researched.
However, there is a lack of open datasets for benchmarking results,
especially associated with the complex networks field,
yielding diverse results from poorly related sources.
Recently, a myriad of results have been reported which are based in
diverse datasets most often not accessible to researchers other than the publishing authors.
In this thesis we present resources for having open databases to provide the scientific community with a friendly and common repertoire.
% This work makes availalable an open database with diverse provenance and
% to furnish the scientific community with a friendly and common repertoire.

Integration and uniformity of access is obtained through linked data
representation, as explained in Section~\ref{queries}.

\subsection{Social participation}
\subsection{Other}
% statistics, circular statistics, PCA, programming, free culture, anthropological physics
% tipologies of human persolanity (myer-briggs and adorno and ours)

\section{Polisemy and synonymes}
% graphs and networks
% nodes, vertices, agents, participants
% links, edges, connections
% NLP, text mining and computational linguistics
% linked data and semantic web

\section{Considerations about the presented work}
% use of complex and social networks scientific knowledge by the participant (open software and texts, UNDP endaviors)
% many fields are related to the subject, which reflected in the fields tackled in the thesis
% many documents produced by this research

\section{Structure of the thesis}
% canonical chapters; stability through circular statistics and topological statistics (PCA, sectioning)
% text mining through measures of sizes of tokens, sentences, messages
% through kolmogorov-smirnov, POS tagging, PCA
% visualization of network evolution through animations (and other gadgets)
% linked data representation of data and ontological organization
% Appendices: supporting tables and diagrams, related results:
% kolmogorov-smirnov, ubiquity of inequality through power laws, list of UNDP products, list of software
