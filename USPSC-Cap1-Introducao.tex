%% USPSC-Introducao.tex

% ----------------------------------------------------------
% Introdução (exemplo de capítulo sem numeração, mas presente no Sumário)
% ----------------------------------------------------------
% \chapter[Introdução]{Introdução}
\chapter{Introduction}\label{ch:int}

The first studies dealing explicitly with human interaction networks
date from the nineteenth century while the foundation of
social network analysis is generally attributed to the psychiatrist Jacob Moreno in mid twentieth century~\cite{moreno,newmanBook}.
With the increasing availability of data related to human interactions, research about these networks has grown continuously.
Contributions can now be found in a variety of fields, from social sciences and humanities~\cite{latour2013} to computer science~\cite{bird} and physics~\cite{barabasiHumanDyn,newmanFriendship}, given the multidisciplinary nature of the topic.
One of the approaches from an exact science perspective is to represent interaction networks as complex networks~\cite{barabasiHumanDyn,newmanFriendship}, with which 
several features of human interaction have been revealed.
For example, the topology of human interaction networks exhibits a scale-free outline,
which points to the existence of a small number of highly connected hubs and a large number of poorly connected nodes.
The dynamics of complex networks representing human interaction has also been addressed~\cite{barabasiEvo,newmanEvolving}, but only to a limited extent, since research is normally focused on a particular metric or task, such as accessibility or community detection~\cite{access,newmanModularity}. 

There are numerous articles, books, websites and software tools about complex and social networks and about text mining in social media.
There are fewer endeavours to characterize these networks beyond general features such as the scale-free 
aspect or to deal with text produced by social networks from the complex networks background.
Research on network evolution is often restricted to network growth, in which there is a monotonic increase in the number of events~\cite{barabasiEvo}.
Network types have been discussed with regard to the number of participants, intermittence of their activity and network longevity~\cite{barabasiEvo}. Two topologically different networks emerged from human interaction networks, depending on whether the frequency of interactions follows a generalized power law or an exponential connectivity distribution~\cite{barabasiTopologicalEv}. In email list networks, scale-free properties were reported with $\alpha \approx 1.8$~\cite{bird} (as in web browsing and library loans~\cite{barabasiHumanDyn}), and different linguistic traces were related to weak and strong ties~\cite{Gmane2}.

The fact that unreciprocated edges often exceed 50\% in human interaction networks~\cite{newmanEvolving} motivated the inclusion of symmetry metrics in our analysis.
No correlation of topological characteristics and geographical coordinates was found~\cite{barabasiGeo},
therefore geographical positions were not considered in our study.
Gender related behavior in mobile phone datasets was indeed reported~\cite{barabasiSex}
but it is not relevant for the present work because email messages and addresses have no gender related metadata~\cite{gmanePack}.

\section{Related knowledge}
\subsection{Complex networks}
Although not universally accepted, it is commonplace to define a complex network to be
a ``graph with non-trivial topological features''.
We might add to this definition that a complex network is also a large graph (even while there
seems not to be a consensus to what \emph{large} means in such context)
and that it is a graph representation of a system found in nature or in real or empirical systems.
Another way to approach the definition of ``complex networks'' is to define it as
complex systems modeled as networks.
This second definition is also useful but is even more problematic as
there is no consensus of what a \emph{complex system} is.
Even so, one should keep in mind that authors often define a complex system
to be a system composed with many parts in which ``the whole is more than
the sum of its parts''.
Authors also often consider complex systems to have capabilities
to ``process information'', to adapt, to reproduce. 

A graph is a structure that consists of a set of objects (called vertices)
and a set of binary/dual relations of the objects (called edges).
Such graph might be unweighted and undirected (the simplest possibility),
weighted and undirected, unweighted and directed, or weighted and directed.

The most usual representations of graphs (and networks) are the matrix and node-edge representations.
In the matrix representation, each entry $a_{ij})$ is non-zero if $i$ is linked to $j$;
entries might be other than 0 and 1 in weighted graphs; undirected graphs yield symmetric matrices.
In the node-edge representation, each node i represented as a point while each edge is represented
by a line between correspondent nodes.
The matrix representation is essential for algebraic reasoning and for deriving measures
while the node-edge representation is important for illustration and intuitive guidance
of the characterization of the systems.

\subsubsection{A good justification for the complex networks theory}
The estimated number of atoms in the universe is often used as a reference of largeness
and is $10^80$.
Let us find the number of vertices needed to reach such number of possible networks.
Let also us consider the simplest case of the unweighted and undirected networks.
Each edge can exist or not (i.e. it is a Bernoulli variable) and with $n$ vertices there are
at most ${n \choose 2}$ edges.
Therefore:
\begin{align}
	2^{n \choose 2} > 10^{80} \Rightarrow 
	log_2[2^{n \choose 2}] > log_2(10^{80}) \Rightarrow
	{n \choose 2} > \frac{log_{10}(10^{80})}{log_{10}2} \Rightarrow \nonumber\\
	\Rightarrow \frac{n.(n-1)}{2} > \frac{80}{log_{10}2} \Rightarrow
		N > 23,5988 \;\;\;\;\;\;\;\;\;\;\;\;\;\;\;\;\;\;\;\;\;
			\nonumber
\end{align}
That is, with only 24 vertices we have more possible networks than
the estimated number of atoms in the universe.
We should also add that the number of possible networks grows
very fast with the number of vertices.
This is a good reason for characterizing such systems by means
of paradigmatic networks and generic measures for nodes and the network (and less often for the edges).

\subsubsection{Basic measures}
Section~\ref{measures} gives a mathematical account of the following measures,
which are here for pointing the characteristics of basic types
of networks presented in the next section.
Such measures are:
\begin{itemize}
	\item Degree     $k_i$: number of edges linked to vertex $i$.
	\item In-degree  $k_i^{in}$: number of edges ending at vertex $i$.
	\item Out-degree $k_i^{out}$: number of edges departing from vertex $i$.
	\item Strength $s_i$: sum of weights of all edges linked to vertex $i$.
	\item In-strength $s_i^{in}$: sum of weights of all edges ending at vertex $i$.
	\item Out-strength $s_i^{out}$: sum of weights of all edges departing from vertex $i$.
	\item Betweenness centrality $bt_i$: fraction of geodesics that contain vertex $i$.
	\item Clustering coefficient $cc_i$: fraction of pairs of neighbors of $i$ that are linked, i.e. the standard clustering coefficient metric for undirected graphs.
\end{itemize}
% distance between vertices

\subsubsection{Basic types of networks}
Complex networks are often characterized in terms of paradigmatic models.
There are diverse models, but we can glimpse the background theory
with the following ones:
\begin{itemize}
	\item The Erdös-Rényi model\footnote{This name is also used for the model in which, for a fixed number of vertices and a fixed number of edges, all graphs are equally likely. This is the model originally introduces by Paul Erdös and Alfréd Rényi~\cite{erdosOrig}.
		We choose the definition given inline, which is closely related to the one given in this footnote, because it is more commonly used nowadays.}: each pair of nodes is connected with a fixed random probability $p$.
		This model presents a characteristic degree ($n.p$ where $n$ is the number of vertices), low clustering and low average distance between nodes.
	\item Spatial network, also called geographic network or geometric graph: nodes are located in a metric space and edges are the probability that two vertices are connected is greater as the distance between them gets smaller. These networks present characteristic degrees, high clustering and large average distance between nodes.
	\item Small-word network: defined as a network where the typical distance between vertices grows with the logarithm of the number of nodes while the average clustering coefficient is not small (larger than e.g. in the Erdös-Rényi model).
		To construct a small-world network, start with a regular lattice in which each vertex is connected to $k$ nearest neighbors.
		Each edge is then rewired with probability $p$.
		With intermediate values of $p$ such as $0.01<p<0.1$, we obtain a network with both short average distance between vertices (as in the Erdös-Rényi model) and a high average clustering coefficient (as in the spatial network).
		This model presents also a characteristic degree.
	\item Scale-free networks: in which the degree distribution $p(k)$ follows a power law ($p(k)=C.k^{-\gamma}$ where C and k are constants).
		These networks are qualitatively characterized by the presence of a large number of poorly connected and of few highly connected vertices.
		Important is the absence of a characteristic degree, thus the name 'scale-free network'.
	\item Other networks: among important models of networks are exponential networks, networks with community structure and hybrid models.
\end{itemize}
Real networks most often exhibit scale-free and small-world properties.
This is the case of most of e.g. social, gene and food networks.
However, one should be cautious about such statement because
the networks derived from the real systems depend heavily
in what is considered a vertex and an edge,
i.e. on how the system is modeled as a graph.
Another noteworthy remark is that
the Erdös-Rényi networks, i.e. graphs of the Erdös-Rényi model, are frequently pin-pointed as the networks with trivial
topological properties.
Even though, it is posed as a paradigmatic ``complex network'', concept often defined as graphs with non-trivial topological properties,
which is a contradiction and exposes that complex networks is not a very well defined notion,
as is the case with the complexity field in general.

\subsection{Text mining of social data}
Text mining is an extension of data mining to - often unstructured - textual data
to discover structure and meaning.
A general outline of a text mining endeavor involves structuring input text,
deriving patterns and the evaluation of the output.
Text mining tasks include document summarization, sentiment analysis
and natural language processing techniques such as part of speech tagging.
Among application are social media monitoring, automated ad placement,
publishing and making tools for semantics, sentiment and general natural language.
It is believed that applying text mining to social media
can yield interesting findings in human behavior.
Although there is no clear cut, text mining is sometimes divided into linguistic and non-linguistic.
In the first case, linguistic techniques are present, such as
the analysis of discourse and part of speech tagging,
and it is often mingled with the natural language processing or computational linguistic fields.
In the non-linguistic text mining, text is analyzed by means of statistical features
such as the size of tokens and sentences, and might be more easily related to the intuitive concept of data mining of text.
On this thesis we use both perspectives.

\subsection{Visualization of graphs}
\subsection{Linked data}
The fields of social network analysis and complex networks
are widely researched.
However, there is a lack of open datasets for benchmarking results,
especially associated with the complex networks field,
yielding diverse results from poorly related sources.
Recently, a myriad of results have been reported which are based in
diverse datasets most often not accessible to researchers other than the publishing authors.
In this thesis we present resources for having open databases to provide the scientific community with a friendly and common repertoire.
% This work makes availalable an open database with diverse provenance and
% to furnish the scientific community with a friendly and common repertoire.

Integration and uniformity of access is obtained through linked data
representation, as explained in Section~\ref{queries}.

\subsection{Social participation}
\subsection{Other}
% statistics, circular statistics, PCA, programming, free culture, anthropological physics
% tipologies of human persolanity (myer-briggs and adorno and ours)

\section{Polisemy and synonyms}
In the context of complex networks, the words \emph{network} and \emph{graph}
are often used interchangeably, although the word graph might refer to the
mathematical structure of vertices and edges and the network might refer to the
real system being represented.
The nomenclature of vertices and edges vary widely among interested fields (mathematics, physics, biology, sociology, etc).
A vertex might be called e.g. a node, a point, an agent, a actor, a participant.
An edge might be called e.g. a link, a bond, a relation, a tie, a connection.

% graphs and networks
% nodes, vertices, agents, participants
% links, edges, connections
% clustering, degree of a vertex and of a network, which is usually the average of the measure in all nodes.
% NLP, text mining and computational linguistics
% linked data and semantic web

\section{Considerations about the presented work}
% use of complex and social networks scientific knowledge by the participant (open software and texts, UNDP endaviors)
% many fields are related to the subject, which reflected in the fields tackled in the thesis
% many documents produced by this research

\section{Structure of the thesis}
% canonical chapters; stability through circular statistics and topological statistics (PCA, sectioning)
% text mining through measures of sizes of tokens, sentences, messages
% through kolmogorov-smirnov, POS tagging, PCA
% visualization of network evolution through animations (and other gadgets)
% linked data representation of data and ontological organization
% Appendices: supporting tables and diagrams, related results:
% kolmogorov-smirnov, ubiquity of inequality through power laws, list of UNDP products, list of software
