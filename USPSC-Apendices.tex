%% USPSC-Apendice.tex
% ---
% Inicia os apêndices
% ---

\begin{apendicesenv}
	% Imprime uma página indicando o início dos apêndices
	\partapendices
	\chapter{Additional tables of the textual differences found in all networks}\label{ap:textd}
In the following tables the counting of differences of textual features among the analyzed networks
are shown.
	These results are auxiliary for the discussion on Section~\ref{sec:tresults}.
\FloatBarrier
\begin{table}[h!]
\begin{center}
\caption{Counts of evidence of difference in the Erd\"os sectors in each of the analyzed networks.}
	\def\arraystretch{1.5}
\begin{tabular}{| l || c | c | c || c | c | c |}\hline
{\bf synset} & {\bf p.} & {\bf i.} & {\bf h} & {\bf peaks} & {\bf total} & {\bf depth} \\\hline\hline
$\frac{punct}{chars-spaces}$ & 11  & 4  & 1  & 5  & 17  & 1 \\\hline
$\frac{vowels}{letters}$ & 0  & 1  & 1  & 1  & 18  & 1 \\\hline
$\frac{spaces}{chars}$ & 2  & 0  & 8  & 2  & 18  & 1 \\\hline
$\frac{letters}{chars-spaces}$ & 0  & 0  & 3  & 0  & 18  & 1 \\\hline
$\frac{digits}{chars-spaces}$ & 4  & 1  & 0  & 2  & 5  & 1 \\\hline
$\frac{uppercase}{letters}$ & 10  & 3  & 1  & 6  & 15  & 1 \\\hline
\end{tabular}
\begin{flushleft}
		Source: Prepared by the authors.\
\end{flushleft}
\end{center}
\end{table}

\begin{table}[h!]
\begin{center}
\caption{Counts of evidence of token-related differences in the Erd\"os sectors in each of the analyzed networks.}
	\def\arraystretch{1.5}
\begin{tabular}{| l || c | c | c || c |}\hline
{\bf synset} & {\bf p.} & {\bf i.} & {\bf h} & {\bf peaks} \\\hline\hline
$\frac{knownw}{tokens}$ & 1  & 0  & 5  & 1 \\
$\frac{knownw \neq}{knownw}$ & 13  & 1  & 4  & 9 \\
$\frac{stopw}{knownw}$ & 0  & 0  & 14  & 2 \\
$\frac{punct}{tokens}$ & 10  & 3  & 1  & 3 \\
$\frac{contrac}{tokens}$ & 0  & 2  & 15  & 4 \\\hline
$\mu(\overline{tokens})$ & 0  & 1  & 2  & 1 \\
$\sigma(\overline{tokens})$ & 7  & 1  & 0  & 2 \\\hline
$\mu(\overline{knownw})$ & 0  & 0  & 2  & 0 \\
$\sigma(\overline{knownw})$ & 0  & 0  & 1  & 1 \\\hline
$\mu(\overline{knownw \neq})$ & 0  & 0  & 0  & 0 \\
$\sigma(\overline{knownw \neq})$ & 0  & 0  & 0  & 0 \\\hline
$\mu(\overline{stopw})$ & 0  & 0  & 0  & 0 \\
$\sigma(\overline{stopw})$ & 0  & 0  & 1  & 0 \\\hline
\end{tabular}
\begin{flushleft}
		Source: By the author.\
\end{flushleft}
\end{center}
\end{table}

\begin{table}[h!]
\begin{center}
\caption{Counts of evidence of sentence-related differences in the Erd\"os sectors in each of the analyzed networks.}
	\def\arraystretch{1.5}
\begin{tabular}{l || c | c | c || c}\hline
{\bf synset} & {\bf p.} & {\bf i.} & {\bf h} & {\bf peaks} \\\hline\hline
$\mu_S(chars)$ & 9  & 3  & 1  & 6 \\
$\sigma_S(chars)$ & 11  & 6  & 1  & 9 \\\hline
$\mu_S(tokens)$ & 10  & 2  & 1  & 5 \\
$\sigma_S(tokens)$ & 9  & 7  & 1  & 9 \\\hline
$\mu_S(knownw)$ & 9  & 3  & 2  & 6 \\
$\sigma_S(knownw)$ & 11  & 5  & 2  & 8 \\\hline
$\mu_S(stopw)$ & 2  & 3  & 7  & 7 \\
$\sigma_S(stopw)$ & 6  & 7  & 4  & 10 \\\hline
$\mu_S(puncts)$ & 13  & 2  & 1  & 2 \\
$\sigma_S(puncts)$ & 7  & 8  & 1  & 8 \\\hline
\end{tabular}
\begin{flushleft}\footnotesize
		Source: By the author.\
\end{flushleft}
\end{center}
\end{table}

\begin{table}[h!]
\begin{center}
\caption{Counts of evidence of message-related differences in the Erd\"os sectors in each of the analyzed networks.}
	\def\arraystretch{1.5}
\begin{tabular}{| l || c | c | c || c | c |}\hline
{\bf synset} & {\bf p.} & {\bf i.} & {\bf h} & {\bf peaks} & {\bf total} \\\hline\hline
$\mu_M(sents)$ & 4  & 7  & 1  & 9  & 16 \\
$\sigma_M(sents)$ & 5  & 7  & 2  & 11  & 15 \\\hline
$\mu_M(tokens)$ & 10  & 5  & 2  & 6  & 18 \\
$\sigma_M(tokens)$ & 8  & 8  & 2  & 9  & 18 \\\hline
$\mu_M(knownw)$ & 8  & 5  & 3  & 7  & 18 \\
$\sigma_M(knownw)$ & 10  & 5  & 3  & 9  & 18 \\\hline
$\mu_M(stopw)$ & 5  & 6  & 6  & 8  & 18 \\
$\sigma_M(stopw)$ & 7  & 6  & 3  & 11  & 18 \\\hline
$\mu_M(puncts)$ & 12  & 4  & 2  & 5  & 18 \\
$\sigma_M(puncts)$ & 8  & 9  & 1  & 10  & 18 \\\hline
$\mu_M(chars)$ & 10  & 5  & 2  & 6  & 18 \\
$\sigma_M(chars)$ & 9  & 7  & 2  & 8  & 18 \\\hline
\end{tabular}
\begin{flushleft}
		Source: Prepared by the authors.\
\end{flushleft}
\end{center}
\end{table}

\begin{table}[h!]
\begin{center}
\caption{Counts of evidence of differences related to POS tags in the Erd\"os sectors in each of the analyzed networks.}
\begin{tabular}{| l || c | c | c || c |}\hline
{\bf synset} & {\bf p.} & {\bf i.} & {\bf h} & {\bf peaks} \\\hline\hline
NOUN & 13  & 1  & 0  & 1 \\
X & 4  & 9  & 5  & 14 \\\hline
ADP & 0  & 1  & 4  & 1 \\
DET & 1  & 0  & 9  & 2 \\\hline
VERB & 0  & 0  & 6  & 1 \\\hline
ADJ & 1  & 2  & 6  & 2 \\
ADV & 0  & 0  & 17  & 1 \\\hline
PRT & 1  & 1  & 9  & 4 \\
PRON & 0  & 1  & 11  & 3 \\
NUM & 8  & 5  & 3  & 7 \\
CONJ & 2  & 6  & 4  & 8 \\\hline
\end{tabular}
\begin{flushleft}\footnotesize
		Source: By the author.\
\end{flushleft}
\end{center}
\end{table}

\begin{table}[h!]
\begin{center}
\caption{Counts of evidence of differences related to Wordnet POS tags in the Erd\"os sectors in each of the analyzed networks.}
\begin{tabular}{l || c | c | c || c}\hline
{\bf synset} & {\bf p.} & {\bf i.} & {\bf h} & {\bf peaks} \\\hline\hline
N & 8  & 1  & 0  & 1 \\
ADJ & 0  & 2  & 12  & 6 \\
VERB & 0  & 1  & 16  & 2 \\
ADV & 0  & 0  & 9  & 1 \\\hline\hline
POS & 0  & 0  & 3  & 1 \\
POS! & 0  & 1  & 0  & 1 \\\hline
\end{tabular}
\begin{flushleft}\footnotesize
		Source: By the author.\
\end{flushleft}
\end{center}
\end{table}

\begin{table}[h!]
\begin{center}
\begin{tabular}{| l || c | c | c || c |}\hline
{\bf synset} & {\bf p.} & {\bf i.} & {\bf h} & {\bf peaks} \\\hline\hline
$\mu(min\,depth)$ & 0  & 0  & 0  & 0 \\
$\sigma(min\,depth)$ & 1  & 1  & 2  & 1 \\\hline
$\mu(max\,depth)$ & 0  & 0  & 0  & 0 \\
$\sigma(max\,depth)$ & 0  & 1  & 3  & 1 \\\hline
$\mu(holonyms)$ & 7  & 4  & 4  & 6 \\
$\sigma(holonyms)$ & 3  & 4  & 7  & 6 \\\hline
$\mu(meronyms)$ & 8  & 5  & 3  & 7 \\
$\sigma(meronyms)$ & 12  & 4  & 2  & 9 \\\hline
$\mu(domains)$ & 6  & 4  & 5  & 8 \\
$\sigma(domains)$ & 3  & 1  & 4  & 3 \\\hline
$\mu(lemmas)$ & 6  & 0  & 1  & 2 \\
$\sigma(lemmas)$ & 6  & 2  & 2  & 4 \\\hline
$\mu(hyponyms)$ & 1  & 6  & 6  & 9 \\
$\sigma(hyponyms)$ & 4  & 6  & 6  & 11 \\\hline
$\mu(hypernyms)$ & 0  & 0  & 0  & 0 \\
$\sigma(hypernyms)$ & 4  & 4  & 4  & 6 \\\hline
\end{tabular}
\caption{Counts of evidence of differences related to Wordnet noun synset characteristics in the Erd\"os sectors in each of the analyzed networks.}
\end{center}
\end{table}
\begin{table}[h!]
\begin{center}
\caption{Counts of evidence of differences related to Wordnet adjective synset characteristics in the Erd\"os sectors in each of the analyzed networks.}
\begin{tabular}{| l || c | c | c || c |}\hline
{\bf synset} & {\bf p.} & {\bf i.} & {\bf h} & {\bf peaks} \\\hline\hline
$\mu(domains)$ & 2  & 6  & 8  & 10 \\
$\sigma(domains)$ & 2  & 4  & 5  & 7 \\\hline
$\mu(similar)$ & 1  & 0  & 7  & 4 \\
$\sigma(similar)$ & 4  & 0  & 5  & 3 \\\hline
$\mu(lemmas)$ & 1  & 2  & 1  & 2 \\
$\sigma(lemmas)$ & 6  & 3  & 4  & 6 \\\hline
\end{tabular}
\begin{flushleft}
		Source: Prepared by the author.\
\end{flushleft}
\end{center}
\end{table}

\begin{table}[h!]
\begin{center}
\caption{Counts of evidence of differences related to Wordnet verb synset characteristics in the Erd\"os sectors in each of the analyzed networks.}
\begin{tabular}{| l || c | c | c || c |}\hline
{\bf synset} & {\bf p.} & {\bf i.} & {\bf h} & {\bf peaks} \\\hline\hline
$\mu(min\,depth)$ & 2  & 1  & 1  & 3 \\
$\sigma(min\,depth)$ & 2  & 1  & 1  & 2 \\\hline
$\mu(max\,depth)$ & 2  & 1  & 0  & 1 \\
$\sigma(max\,depth)$ & 3  & 1  & 1  & 2 \\\hline
$\mu(domains)$ & 7  & 3  & 4  & 4 \\
$\sigma(domains)$ & 8  & 3  & 3  & 5 \\\hline
$\mu(verb\,groups)$ & 0  & 2  & 3  & 2 \\
$\sigma(verb\,groups)$ & 0  & 0  & 0  & 0 \\\hline
$\mu(lemmas)$ & 0  & 0  & 2  & 0 \\
$\sigma(lemmas)$ & 1  & 0  & 3  & 0 \\\hline
$\mu(entailments)$ & 7  & 1  & 7  & 3 \\
$\sigma(entailments)$ & 4  & 1  & 5  & 3 \\\hline
$\mu(hyponyms)$ & 1  & 2  & 6  & 3 \\
$\sigma(hyponyms)$ & 2  & 3  & 8  & 6 \\\hline
$\mu(hypernyms)$ & 2  & 2  & 0  & 2 \\
$\sigma(hypernyms)$ & 1  & 0  & 1  & 1 \\\hline
\end{tabular}
\begin{flushleft}
		Source: Prepared by the author.\
\end{flushleft}
\end{center}
\end{table}

\begin{table}[h!]
\begin{center}
\caption{Counts of evidence of differences related to Wordnet adverb synset characteristics in the Erd\"os sectors in each of the analyzed networks.}
\begin{tabular}{l || c | c | c || c}\hline
{\bf synset} & {\bf p.} & {\bf i.} & {\bf h} & {\bf peaks} \\\hline\hline
$\mu(domains)$ & 3  & 3  & 10  & 10 \\
$\sigma(domains)$ & 1  & 4  & 7  & 7 \\\hline
$\mu(lemmas)$ & 0  & 0  & 1  & 1 \\
$\sigma(lemmas)$ & 3  & 1  & 2  & 4 \\\hline
\end{tabular}
\begin{flushleft}\footnotesize
		Source: By the author.\
\end{flushleft}
\end{center}
\end{table}


\chapter{Developments made in this research but not included elsewhere in this thesis}\label{ap:vot}
In this appendix are gathered developments relevant for the initial proposal of this research:
enabling the use of complex networks scientific knowledge by the participant of the social networks.
These developments are not included elsewhere in the thesis because we chose to present results
more closely related to physics in a simple fashion.
Furthermore, most of the following contributions have received dedicated documentation,
reason why we next only summarize and cite the documents when they are available.

\section{Continuous voting by approval and participation}
In finding the adequate way to prioritize proposals, the Brazilian social participation community agreed about the measurement of two indexes,
one of approval and one of participation. Both practice and literature
was constantly handled by the experts involved, and the formalization
of such model and metrics and is very simple and seems novel.
Also, the relevance of this development is strengthened by the use of these indexes by the
Brazilian General Secretariat of the Republic to raise and prioritize
proposals about public health care in open processes.
This was achieved by means of the Dialoga Brasil platform~\cite{dialoga}.
A short report on these indexes and their use is on~\cite{dialogaAlg}
	(co-authorship by Ricardo Poppi).

\section{Visualization of static networks}
% email pelo app online
% fb pelo gephi
% arte pelo Pedro Paulo Rocha
\section{Social structures live streaming}
% telões de redes e de medidas de texto
\section{OPS: the Social Participation (OWL) Ontology}
\section{OPa: the ParticipaBR (OWL) Ontology}
\section{OntologiAA: the AA (OWL) Ontology}
\section{Social Library Ontology and Social Library Vocabulary: OBS in OWL and VBS in SKOS}
% Conselhos, Foruns, Mesas Redondas, etc. Casos feitos com especialistas, casos feitos a partir de documentações
\section{United Nations Development Program consulting}
\subsection{Partnership with the Brazilian Presidency}
\subsection{Description of each product}
\section{Collection and diffusion of information in social networks}
% processo dos periféricos aos hubs
% instantâneos pelos de maior betweenness e maior closeness
\section{Anthropological physics}
The study of complex systems can be undertaken as a physics endeavor,
specially if complex networks and statistics are into play. When the complex
system is constituted by people, intriguing questions arise from diverse field such as math, ethics, and sociology. The “anthropological physics”is an approach to these scenarios that enables scientific research while resolving ethical and moral issues by an open study of the self.
	It yields a transdisciplinary practice whose relevance emanate from anthropological
and physical matters, from human constituted systems and natural laws.
A sweet spot was found in recent civil, government and academic efforts~\cite{opa,ensaio}, and has been called anthropological physics. General characteristics are:
\begin{itemize}
	\item Exposure of the researcher to the environment of interest, such as virtual social networks.
	\item Use of the annotations from the exposure, be them activity logs, friendship or interaction networks, textual contents, etc.
	\item Upon need, expansion of observations to encompass open datasets or data donated by partners.
	\item Observance of natural laws as they appear in network structures and natural language.
	\item All resources are kept as open and publicized as possible, including software, data, and writings.
\end{itemize}                                                                                                                                     
A short report the first insights regarding anthropological physics is on~\cite{anPhy}.
 

% conceitualização básica, artigo e conferências
\section{Listing of documents written, conferences attended and artistic presentations}
% rcpln
% artigos
% conferências: duas nexos+ccdc+linguística+sifiscs+ufpa+
% apresentações na SGPR, imersões, arenaNETmundial
% vivace

\end{apendicesenv}
