% ---
%% USPSC-Cap3-Citacoes.tex
% --
% Este capítulo traz os exemplos de citações das "Diretrizes para apresentação de dissertações e teses da USP: documento eletrônico e impresso - Parte I (ABNT)" disponílvel em: http://biblioteca.puspsc.usp.br/pdfFiles_Caderno_Estudos_9_PT_1.pdf


% --- 
\chapter{Results and Discussion}
\label{ch:disc}
% --- 
\subsection{Activity along time}\label{constDisc}
Regular patterns of activity were observed along time
in the scales of seconds, minutes, hours, days and months.
Histograms in each of the time scales were computed as were circular average and dispersion values, and the results are given in Tables~\ref{tab:circ}-\ref{tab:min2}. For example, uniform activity is found with respect to seconds, minutes and days of the months. Weekend days exhibit about half the activity of regular weekdays, and there is a peak of activity between 11am and noon.


\begin{table}
\caption{The rescaled circular mean $\theta_\mu'$ and the circular dispersion $\delta(z)$, described in Section~\ref{sec:mtime}, for different timescales. This example table was constructed using all LAD messages, and the results are the same for other lists, as shown in Section~\ref*{si:circ} of the Supporting Information document. The most uniform distribution of activity was found in seconds and minutes. 	Hours of the day exhibited the most concentrated activity (lowest $\delta(z)$), with mean between 2 p.m. and 3 p.m. ($\theta'=-9.61$). Weekdays, days of the month and months have mean near zero (i.e. near the beginning of the week, month and year) and high dispersion. Note that $\theta_u'$ has the dimensional unit of the corresponding time period while $\delta(z)$ is dimensionless.}
\begin{center}
\begin{tabular}{ |l|| c|c| }
\hline
%scale & $\theta_\mu'$ & $S(z)$ & $Var(z)$ & $\delta(z)$ & $\frac{max(incidence)}{min(incidence)}$ & $ \mu_{\frac{max(incidence')}{min(incidence')}} $ & $ \sigma_{\frac{max(incidence')}{min(incidence')} } $ \\ \hline\hline
%& $\theta_\mu'$ & $S(z)$ & $Var(z)$ & $\delta(z)$  \\ \hline\hline
scale & mean $\theta_\mu'$ & dispersion $\delta(z)$  \\ \hline
\input{tables/tab2TimeLAD___}
\end{tabular}
\end{center}
\label{tab:circ}
\end{table}

\begin{table}
\caption{Activity percentages along the hours of the day. Nearly identical distributions were observed on other social systems as shown in Section~\ref*{si:hours} of the Supporting Information document.
Highest activity was observed between noon and 6pm (with 1/3 of total day activity), followed by the time period between 6pm and midnight.
Around 2/3 of the activity takes place from noon to midnight
but the activity peak occurs between 11 a.m. and 12 p.m.
This table shows results for the activity in CPP.}
\footnotesize
\input{tables/tabHoursCPP_}
\label{tab:hin}
\end{table}


\begin{table}
\caption{Activity percentages along weekdays.
Higher activity was observed during workweek days, with a decrease of activity on weekend days of at least one third and at most two thirds.}
\begin{center}
\begin{tabular}{ | l ||  c | c | c | c | c |   c | c |}
\hline
& Mon & Tue & Wed & Thu & Fri & Sat & Sun  \\ \hline
\input{tables/tabWeekdays}
\end{tabular}
\end{center}
\label{tab:win}
\end{table}

In the scales of seconds and minutes, activity is uniform,
with the messages being slightly more evenly distributed in all lists than in simulations with the uniform distribution\footnote{Numpy version 1.8.2, ``random.randint'' function, was used for simulations, algorithms in \url{https://github.com/ttm/percolation}.}.
In the networks, $\frac{min(incidence)}{max(incidence)} \in (0.784,.794)$ while simulations reach these values but have on average more discrepant higher and lower peaks, i.e. if $\xi=\frac{min(incidence')}{max(incidence')}$ than $\mu_\xi=0.7741 \text{ and } \sigma_\xi=0.02619$.
Therefore, the incidence of messages at each second of a minute and at each minute of an hour was considered uniform.
In these cases, the circular dispersion is maximized and the mean has little meaning as indicated in Table~\ref{tab:circ}.
As for the hours of the day, an abrupt peak is found between 11am and 12pm with the most active period being the afternoon, with one third of total daily activity, and two thirds of activity are allocated in the second 12h of each day. Days of the week revealed a decrease between one third and two thirds of activity on weekends.
Days of the month were regarded as homogeneous with an inconclusive slight tendency of the first week to be more active.
Months of the year revealed patterns matching usual work and academic calendars. The time period examined here was not sufficient for the analysis of activity along the years. These patterns are exemplified in Tables~\ref{tab:hin}-\ref{tab:min2}.


\FloatBarrier

\begin{table}
\caption{Activity along the days of the month cycle.
Nearly identical distributions are found in all systems
as indicated in Section~\ref*{si:monthdays} of the Supporting Information. Although slightly higher activity rates are found in the beginning of the month, the most important feature seems to be the homogeneity made explicit by the high circular dispersion in Table~\ref{tab:circ}.
This specific example and empirical table correspond to the activity of the MET email list.}
\footnotesize
\input{tables/tabMonthdaysMET}
\label{tab:min}
\end{table}

\begin{table}
\caption{Activity percentages on months along the year. 	Activity is usually concentrated in Jun-Aug and/or in Dec-Mar, potentially due to academic calendars, vacations and end-of-year holidays. This table corresponds to activity in LAU. Similar results are shown for other lists in Section~\ref*{si:months} of the Supporting Information document.}
\footnotesize
\input{tables/tabMonthsLAU}
\label{tab:min2}
\end{table}


\subsection{Stable sizes of Erd\"os sectors}\label{subsec:pih}

The distribution of vertices in the hub, intermediary and periphery Erd\"os sectors is remarkably stable along time if the snapshots hold 200 or more messages,
as it is clear in Figure~\ref{fig:sectIL} and in Section~\ref{si:frac} of the Appendix. 
%Moreover, all email lists analyzed exhibit the same distribution profile.
Activity is highly concentrated on the hubs, while a very large number of peripheral vertices contribute to only a fraction of the activity.
This is expected for a system with a scale-free profile, as confirmed with the distribution of activity among participants in Table~\ref{autores}.

Typically, $[3\%-12\%]$ of the vertices are hubs,
$[15\%-45\%]$ are intermediary and $[44\%-81\%]$ are peripheral,
which is consistent with other studies~\cite{secFree}.
These results hold for the total, in and out degrees and strengths.
Stable sizes are also observed for 100 or less messages if the classification 
of the three sectors is performed with one of the compound criteria established in Section~\ref{sectioning}. The networks often hold this basic structure with as few as 10-50 messages, i.e. concentration of activity and the abundance of low-activity participants take place even with very few messages, which is highlighted in Section~\ref{si:frac} of the Appendix.
A minimum window size for the observation of more general properties might be inferred by monitoring 
both the giant component and the degeneration of the Erd\"os sectors.

In order to support the generality of these findings,
we list the Erd\"os sector sizes of 12 networks from Facebook, Twitter and Participabr in Table~\ref{tab:secE} of the Appendix.
The fractions of hubs, intermediary and peripheral nodes are
essentially the same as for the email list networks but with exceptions and a greater variability.

\begin{figure*} 
\centering
\includegraphics[width=\textwidth]{figs/InText-WLAU-S1000__}
\caption{Stability of Erd\"os sector sizes.
Fractions of participants derived from degree and strength criteria, $E_1$ and $E_4$ described in Section~\ref{sectioning}, are both on the left.
Fractions derived from the exclusivist $C_1$ and the inclusivist $C_2$ compound criteria are shown in the plots to the right.
The ordinates $\overline{e_{\gamma,\phi}}=\frac{|e_{\gamma,\phi}|}{N}$ denote the fraction of participants in sector $\phi$ through criterion $E_\gamma$
and, similarly, $\overline{c_{\delta,\phi}}=\frac{|c_{\delta,\phi}|}{N}$ denotes the fraction of participants in sector $\phi$ through criterion $C_\delta$.
Sections~\ref*{si:frac} and~\ref*{si:ext} of the Supporting Information bring a systematic collection of such timeline figures with all simple and compound criteria specified in Section~\ref{sectioning}, with results for networks from Facebook, Twitter and Participabr.}
\label{fig:sectIL}
\end{figure*}


\begin{table}[h]
\caption{Distribution of activity among participants.
The first column shows the percentage of messages sent by the most active participant. The column for the first quartile ($Q_1$) gives the minimum percentage of participants responsible for at least 25\% of total messages with the actual percentage in parentheses. Similarly, the column for the first three quartiles $Q_3$ gives the minimum percentage of participants responsible for 75\% of total messages.
The last decile $D_{-1}$ column shows the maximum percentage of participants responsible for 10\% of messages.}
\begin{center}
\begin{tabular}{ | l ||  c | c | c | c | }
\hline
list & hub & $ Q_1 $ & $ Q_3 $ & $D_{-1}$ \\ \hline
\input{tables/userTab}
\end{tabular}
\end{center}
\label{autores}
\end{table}


\subsection{Stability of principal components}\label{prevalence}
%The topology was analyzed using standard, well-established metrics of centrality and clustering.
%We also introduced symmetry metrics given the evidence of their importance in social contexts~\cite{newmanEvolving}.
%The contribution of each metric to the variance is very similar for all the networks and along time.

The principal components of the participants are very stable in the topological space,
i.e. in the space of network measures.
Table~\ref{tab:pcain} exemplifies the formation of principal components by providing the averages over non-overlapped activity snapshots of a network. The most important result of this application of PCA, the stability of principal components, is underpinned by the very small dispersion of the contribution of each metric to each principal component.
%The contribution of each metric to the
%principal components presents
%very small standard deviation.

\begin{table}[!h]
\caption{Loadings for the 14 metrics into the principal components for the MET list, $1000$ messages in 20 disjoint positions. The clustering coefficient (cc) appears as the first metric in the table, followed by 7 centrality metrics and 6 symmetry-related metrics. Note that the centrality measurements, including degrees, strength and betweenness centrality, are the most important contributors for the first principal component, while the second component is dominated by symmetry metrics. The clustering coefficient is only relevant for the third principal component. The three components have in average more than 85\% of the variance.
The low standard deviation $\sigma$ implies that the principal components are considerably stable.}
\footnotesize
\input{tables/tabPCA3CPP}
\label{tab:pcain}
\end{table}

The first principal component is an average of centrality metrics:
degrees, strengths and betweenness centrality.
On one hand, the similar relevance of all centrality metrics is not surprising since they are highly correlated,
e.g. degree and strength have Spearman correlation coefficient $\in [0.95,1]$ 
and Pearson coefficient $\in [0.85,1)$ for window sizes greater than a thousand messages.
On the other hand, each of these metrics is related to a different participation characteristic,
and their equal relevance for variability,
as measured by the principal component, is noticeable.
Also, this suggests that these centrality metrics 
are equally adequate for characterizing the networks
and the participants.

\begin{figure} 
\centering
%\includegraphics[width=.6\textwidth,height=10cm]{figs/im13PCAPLOT__}
\includegraphics[width=.45\textwidth]{figs/im13PCAPLOT__}
\caption{The first plot highlights the well-known pattern of degree versus clustering coefficient, characterized by the higher clustering coefficient of lower degree vertices.
    The second plot shows the greater dispersion of the symmetry-related ordinates dominant in the second principal component (PC2).
This larger dispersion suggests that symmetry-related metrics are more powerful,
for characterizing interaction networks than the clustering coefficient,
especially for hubs and intermediary vertices.
This figure reflects a snapshot of the LAU list with 1000 contiguous messages.}

%		Similar structures were observed in all window sizes $ws\;\in\;[500,10000]$, in networks derived from email lists,
%		and in networks from Facebook, Twitter and Participabr,
%		which suggests a common relationship between the metrics of degrees, strengths and betweenness centrality,
%		the symmetry-related metrics and clustering coefficient.}
\label{fig:sym}
\end{figure}

According to Table~\ref{tab:pcain} and Figure~\ref{fig:sym},
dispersion is larger in symmetry-related metrics than in clustering coefficient.
% As expected from basic complex network theory, peripheral vertices have low values of centrality metrics and larger dispersion with regard to the clustering coefficient.
% %The scatter plot in the third system of Figure~\ref{fig:sym},
% %where all metrics are considered and there is a greater dispersion
% %with respect to the ordinates,
% This reflects in the relevance of the symmetry-related metrics.
We conclude that the symmetry metrics are more powerful, in terms of dispersion in the topological metrics space, in characterizing interaction networks and their participants, than the clustering coefficient, especially for hubs and intermediary vertices (peripheral vertices have larger dispersion with regard to the clustering coefficient).
Interestingly, the clustering coefficient is always combined
with the standard deviation of the asymmetry and disequilibrium
of edges $\sigma^{asy}$ and $\sigma^{dis}$ in the third principal component.

%These results are also reported for 12 networks from Facebook, Twitter and Participabr
%in Section~\ref{si:ext} of the Supporting Information document.
Similar results are presented in Sections~\ref{si:pcat} and~\ref{si:ext}
of the Supporting Information for other email lists and interaction networks. A larger variability was found for the latter networks,
which motivated the use of interaction networks derived from email lists for benchmarking.

%the overall behavior was maintained in that centrality measurements 
%were found prevalent in the first principal component,
%followed by symmetry-related metrics on the second principal
%component and then clustering coefficient on the third principal component.
%Similar results are presented in Sections~\ref{si:pcat} and~\ref{si:ext}
%of the Supporting Information document for other email lists and other interaction networks,
%with the consideration of strategic combinations of metrics.

\subsection{Types from Erd\"os sectors}\label{sec:pty}


Assigning a type to a participant raises important issues about the scientific cannon for human types and the potential for stigmatization and prejudice. The Erd\"os sector to which a participant belongs can be regarded as implying a social type for this participant.
In this case, the type of a participant changes both along time and as different networks are considered, despite the stability of the network. Therefore, the potential for prejudice of such participant typology is attenuated~\cite{adorno}. In other words, an individual is a hub in a number of networks and peripheral in other networks, and even within the same network he/she most probably changes type along time~\cite{animacoes}.

The importance of this issue can be grasped by the consideration of static types derived from quantitative criteria. For example, in email lists with a small number of participants, the number of threads has a negative correlation with the number of participants.
When the number of participants exceeds a threshold, the number of threads has a positive correlation with the number of participants.
This finding is illustrated in Figure~\ref{fig:nmgamma3d}
and can also be observed in Table~\ref{tab:genLists}.
The assignment of types to individuals, in this latter case,
has more potential for prejudice because
the derived participant type is static and
one fails to acknowledge that
human individuals are not immutable entities.

Further observations regarding the Erd\"os sectors
and the implicit participant types were made, which are consistent with the literature~\cite{barabasiEvo}: 1) hubs and intermediary participants usually have intermittent activity, and stable activity was found only in smaller communities. For instance, the MET list had stable hubs while LAU, LAD and CPP exhibited intermittent hubs.
2) Network structure seems to be most influenced by the
activity of intermediary participants as they have less extreme
roles than hubs and peripheral participants and
can therefore connect to the sectors and other participants 
in a more selective and explicit manner.




%Moreover, such typology of participants bridges exact and human sciences and may 
%be enriched with concepts from other typologies,
%such as Meyer-Briggs, Pavlov or the authoritarian types of the F-Scale~\cite{adorno}.

%We analyzed the temporal evolution of the networks
%using visualization
%tools developed for this research~\cite{rcText,versinus}
%and inspected raw data.

%dictated (or revealed) by the
%(e.g. stable or intermittent patterns of activity and preferential communication
%with hubs or periphery)

%of both hubs and peripheral vertices
%have the trivial facets of interacting 
%
%
%
%\begin{itemize}
%	\item Typically, the activity of hubs is trivial: they interact as much as possible, in every occasion with everyone.
%The activity of peripheral vertices also follows a simple pattern: they interact very rarely, in very few occasions.
%Therefore, intermediary vertices seem responsible for the network structure.
%Intermediary vertices may exhibit preferential communication to peripheral, intermediary, or hub vertices; can be marked by stable communication partners; can involve stable or intermittent patterns of activity, to point just a few examples of this greater variety of roles.
%%	\item Some of the most active participants receive many responses with relative few messages sent, and rarely are top hubs.
%%These seem as authorities and contrast with participants that respond much more than receive responses.
%%	\item The most obvious community structure, as observed by a high clustering coefficient, i.e. members know each other often, is found mostly in peripheral and intermediary sectors.
%\end{itemize}

%Within networks as the whole objects of analysis,
%we were able to observe a peculiar correlation pattern 
%between the number of threads and the number of participants.
\begin{figure}
\centering
\includegraphics[trim={0 0 0 1cm},clip,width=.7\columnwidth]{figs/mpgamma2_}
\caption{A scatter plot of number of messages $M$ versus number of participants $N$ versus number of threads $\Gamma$ for 140 email lists.
Highest $\Gamma$ is associated with low $N$.
The correlation between $N$ and $\Gamma$ is negative for low values of $N$ but positive otherwise.
This negative correlation between $N$ and $\Gamma$ can also be observed in Table~\ref{tab:genLists}.
Accordingly, for $M=20000$ messages, this inflection
of correlation was found around $N=1500$, while CPP, LAU, LAD, MET lists 
present smaller networks.}
\label{fig:nmgamma3d}
\end{figure}


%\section{Discussion}
% given the results, and before reaching the conclusions
% what to say?
% --> what is the overall knowledge derived from the results
% --> what are the limitations of this knowledge and of individual results
% --> how should this results carry on is on the next sections.
%\subsection{Consecutive scientific research}
% --> research
% textual diferences
% audiovisualization of data
% typologies, sociological critical theory, social psychology

%\subsection{Technological applications}
% --> technological 
% resources categorization and recommendation
% document creation
% ontologies for the semantic web

%\subsection{Experimental and theoretical aspects of the research}
% --> methods
% Exploratory?
% Hypothesis testing?
% --> contributions
% verifiable
% knowledge
% contextualization in the academic knowledge

\subsection{Implications of the main findings}\label{sec:impl}
The findings reported in this article arose from an exploratory procedure to visually inspect the networks and to analyze considerable amounts of interaction networks data.
% deriving from email lists and also from other networks.
While this procedure has certainly an ad hoc nature, the statistics in the data are sufficiently robust for important features from these interaction networks to be extracted.
Temporal stability, in the sense that interaction networks could be considered as stationary time series, is the most important feature. Also relevant is the significant stability found on the principal components, on the fraction of participants in each Erd\"os Sector and on the activity along different timescales. In fact, these findings confirm our initial hypothesis - based on the literature~\cite{newmanBook} - that interaction networks should exhibit some stability traces. The potential generality of these findings is suggested by the analysis of networks derived from diverse systems, with interaction networks from public email lists serving as proper benchmarks. Indeed, with such benchmarks one can compare any social network system. Furthermore, this analysis enables us to establish an outline of human interaction networks. It takes the hub, intermediary and periphery sectors out of the scientific folklore and into classes drawn from quantitative criteria. It enables the conception of non-static human types derived from natural properties.

 
We envisage that the knowledge generated in the analysis may be exploited in applications where the type of each participant and the relative proportion of participants in each sector can be useful metadata. Just by way of illustration, this could be applied in semantic web initiatives, given that the Erd\"os sectorialization is static in a given snapshot. These results are also useful for classifying resources, e.g. in social media, and for resources recommendation to users~\cite{opa}. 
Finally, the knowledge acquired with a quantitative treatment of the whole data may help guide the creation through collective processes of documents to assist in participatory democracy.

 
Perhaps the most outreaching implications are related to sociological consequences. The results expose a classification of human individuals which is directly related to the concentration of wealth and based on natural laws. The derived human typology changes over different systems and over time in the same system, which implies a negation of the absolute concentration of wealth. Such concentration exists but changes across different wealth criteria and with time. Also, the hubs stand out as dedicated, sometimes enslaved,
components of the social system. The peripheral participants have very limited interaction with the network. This suggests that intermediary participants tend to dictate structure, legitimate the hubs and stand out as authorities.

 
With regard to the limitations of our study, one should emphasize that not all types of human interaction networks were analyzed. Therefore, the plausible generalization of properties has to be treated with caution, as a natural tendency of such systems and not as a rule. Also, the stable properties in the networks were not explored to the limit, which leaves many open questions. For example, what are the maximum and minimum sizes of the networks for which they hold? What is the outcome of PCA analysis when more metrics are considered? What is the granularity in which the activity along the timescales is preserved? Do the findings reported also apply to other systems, beyond human networks?
 
 % TTM find
\section{Text results and discussion}\label{sec:tresults}

The most important result of including textual metrics in our analysis is the
extreme differentiation of each Erd\"os sector with respect to the texts produced.
For example: hubs use more contractions, more adjectives,
more common words, and less punctuation if compared to the rest of the network,
specially the peripheral sector.
In general, the rise or fall of a metric is monotonic along connectivity,
but some of them reached extreme values in the intermediary sector.

Next sections summarize results of immediate interest
and further insights can be obtained by skipping through
the tables and figures os the Appendix~\ref{ap:texttop}.

\subsection{General characteristics of activity distribution among participants}\label{sec:gen}
\begin{table}[h!]
\begin{center}
\caption{Distribution of participants, messages and threads among each Erd\"os sector ({\bf p.} for periphery, {\bf i.} for intermediary, 
    {\bf h.} for hubs) in a total timespan of 0.72 years (from 2003-11-30T20:21:32 to 2004-08-19T18:11:24). $N$ is the number of participants, $M$ is the number of messages, $\Gamma$ is the number of threads, and $\gamma$ is the number of messages in a thread.
    The \% denotes the usual `per cent' with respecto to the total quantity ($100\%$ for {\bf g.})
    while $\mu$ and $\sigma$ denote mean and standard deviation. TAG of list in the Appendix~\ref{ap:texttop}: 10}\label{geralListas}    
\begin{tabular}{| l || c | c | c | c |}\hline
 & {\bf g.} & {\bf p.} & {\bf i.} & {\bf h.} \\\hline\hline
$N$ & 131  & 80  & 46  & 5 \\
$N_{\%}$ & 100.00  & 61.07  & 35.11  & 3.82 \\\hline
$M$ & 1000.00  & 136.00  & 361.00  & 503.00 \\
$M_{\%}$ & 100.00  & 13.60  & 36.10  & 50.30 \\\hline
$\Gamma$ & 292.00  & 76.00  & 147.00  & 69.00 \\
$\Gamma_{\%}$ & 100.00  & 26.03  & 50.34  & 23.63 \\\hline
$\frac{\Gamma}{M}\%$ & 29.20  & 55.88  & 40.72  & 13.72 \\
$\mu(\gamma)$ & 2.74  & 2.76  & 2.81  & 2.58 \\
$\sigma(\gamma)$ & 0.44  & 0.43  & 0.39  & 0.49 \\\hline
\end{tabular}
\end{center}
\begin{flushleft}
		Source: Prepared by the authors.\
\end{flushleft}
\end{table}

Hubs and periphery swap fractions of participants and activity:
while peripheral sector has $\approx 75\%$ of participants, it produces $\approx 10\%$ of all messages.
Conversely, hubs sector present $\approx 10\%$ of participants and produces $\approx 75\%$ of all messages.
% verificar TTM
Fewer threads are created by the hubs in proportion to total messages sent,
while threads created by the periphery are twice as frequent as general messages.
% medir
This suggests a complementarity between peripheral diversity and hub specialization
which, on its turn, deepens the understanding of the interaction network as a meaningful system, 
notably if yield by online activity.
These assertions are condensed in Table~\ref{geralListas}.

%Also, comparing lists with a fixed number of messages,
%the number of threads created seem to increase as the number of participants decrease.

\subsection{Characters}\label{sec:cha}
\begin{table}[h!]
\begin{center}
\begin{tabular}{| l || c | c | c | c |}\hline
 & {\bf g.} & {\bf p.} & {\bf i.} & {\bf h.} \\\hline\hline
$chars$ & 82933  & 7162  & 28170  & 47601 \\
$chars_{\%}$ & 100.00  & 8.64  & 33.97  & 57.40 \\\hline
$\frac{spaces}{chars}$ & 14.96  & 13.59  & 15.21  & 15.01 \\
$\frac{punct}{chars-spaces}$ & 8.17  & 6.98  & 8.03  & 8.44 \\
$\frac{digits}{chars-spaces}$ & 0.90  & 1.97  & 0.77  & 0.80 \\\hline
$\frac{letters}{chars-spaces}$ & 88.72  & 88.88  & 88.98  & 88.54 \\
$\frac{vogals}{letters}$ & 40.47  & 39.17  & 40.72  & 40.53 \\
$\frac{uppercase}{letters}$ & 5.27  & 6.22  & 5.39  & 5.05 \\\hline
\end{tabular}
\caption{Characters in each Erd\"os sector ({{\bf p.}} for periphery, {{\bf i.}} for intermediary, 
    {{\bf h.}} for hubs).}
\end{center}
\end{table}
Texts from peripheral vertices use more punctuation characters, digits and uppercase letters.
Hubs use more letters and vowels among letters.
The use of white spaces does not seem to have any relation to connectivity,
with the exception that the intermediary often presented a slightly higher or lower
incidence of spaces than both peripheral and hub sectors. 
% The total number of characters in ELE list,
% in the 20 thousand messages,
% is more than three times what other lists exhibited.
% This suggests peculiarities related to communication conventions and style (see Appendix~\ref{sec:materials}) and were found not related to topological features.
Further information is given in Table~\ref{tab:cha}.

\subsection{Tokens and words}\label{subsec:tw}
%\begin{table}[h!]
\begin{center}
\begin{tabular}{| l || c | c | c | c |}\hline
 & {\bf g.} & {\bf p.} & {\bf i.} & {\bf h.} \\\hline\hline
$tokens$ & 17964  & 1539  & 6064  & 10361 \\
$tokens_{\%}$ & 100.00  & 8.57  & 33.76  & 57.68 \\
$tokens \neq$ & 15.21  & 32.16  & 25.89  & 18.02 \\\hline
$\frac{knownw}{tokens}$ & 36.48  & 35.74  & 38.03  & 35.69 \\
$\frac{knownw \neq}{knownw}$ & 8.62  & 24.73  & 15.31  & 10.84 \\\hline
$\frac{stopw}{knownw}$ & 11.43  & 10.00  & 11.71  & 11.47 \\
$\frac{punct}{tokens}$ & 23.73  & 22.35  & 22.82  & 24.47 \\
$\frac{contrac}{tokens}$ & 0.01  & 0.00  & 0.00  & 0.02 \\\hline
\end{tabular}
\caption{tokens in each Erd\"os sector ({{\bf p.}} for periphery, {{\bf i.}} for intermediary, 
    {{\bf h.}} for hubs).}
\end{center}
\end{table}
\begin{table}[h!]
\begin{center}
\begin{tabular}{| l || c | c | c | c |}\hline
 & {\bf g.} & {\bf p.} & {\bf i.} & {\bf h.} \\\hline\hline
$tokens$ & 17964  & 1539  & 6064  & 10361 \\
$tokens_{\%}$ & 100.00  & 8.57  & 33.76  & 57.68 \\
$tokens \neq$ & 15.21  & 32.16  & 25.89  & 18.02 \\\hline
$\frac{knownw}{tokens}$ & 36.48  & 35.74  & 38.03  & 35.69 \\
$\frac{knownw \neq}{knownw}$ & 8.62  & 24.73  & 15.31  & 10.84 \\\hline
$\frac{stopw}{knownw}$ & 11.43  & 10.00  & 11.71  & 11.47 \\
$\frac{punct}{tokens}$ & 23.73  & 22.35  & 22.82  & 24.47 \\
$\frac{contrac}{tokens}$ & 0.01  & 0.00  & 0.00  & 0.02 \\\hline\hline
$\mu(\overline{tokens})$ & 3.84  & 3.94  & 3.86  & 3.82 \\
$\sigma(\overline{tokens})$ & 2.99  & 3.10  & 2.96  & 2.99 \\\hline
$\mu(\overline{knownw})$ & 3.28  & 3.27  & 3.32  & 3.26 \\
$\sigma(\overline{knownw})$ & 1.81  & 1.86  & 1.80  & 1.81 \\\hline
$\mu(\overline{knownw \neq})$ & 5.12  & 4.35  & 4.92  & 4.98 \\
$\sigma(\overline{knownw \neq})$ & 2.20  & 2.22  & 2.25  & 2.15 \\\hline
$\mu(\overline{stopw})$ & 1.77  & 1.84  & 1.77  & 1.76 \\
$\sigma(\overline{stopw})$ & 0.74  & 0.76  & 0.73  & 0.75 \\\hline
\end{tabular}
\caption{Token sizes in each Erd\"os sector ({{\bf p.}} for periphery, {{\bf i.}} for intermediary, {{\bf h.}} for hubs).}
\end{center}
\end{table}

%The largest average size of tokens is from the most wordy list (ELE).
%This implies that is has more characters, tokens, and characters per token in comparison to the other lists. 
The longer words used by hubs might be related to the use of a specialized vocabulary.
% VERITICCAR TTM
% palavras mais longas podem vir dos perifericos pois nao usam jargao
Although the token diversity ($\frac{|tokens \neq|}{|tokens|}$) found in peripheral sector is far greater, this result has the masking artifact that the peripheral sector corpus is smaller, yielding a larger token diversity.
% medir para tamanhos equivalentes de texto TTM
This can be noticed by the token diversity of the whole network, which is lower than in any of the sectors.
This same results apply to the lexical diversity ($\frac{|kw\neq|}{kw}$).

Punctuations among tokens are less abundant in hubs, and discrepancies here are larger than with character comparisons (subsection~\ref{sec:cha}). Known words are used more frequently by hubs.

ELE and CPP both exhibit intermediaries 
with the more frequent production of punctuation,
less frequent production of known words, and
the highest incidence of words with wordnet synsets among known words.
This suggests some peculiarity in network structure,
such as authorities in the intermediary sector of such networks,
using smaller sentences and a more intensive use of jargons,
as made explicit in the following sections.
% verificar isso e ver o que fazer

Words with synsets,
among known English words, are less frequent in hubs sector,
evidencing the jargon and specialization of hubs.
% verificar, desenvolvar

Further information is given in Table~\ref{tab:tokens}.

\subsection{Sizes of tokens and words}\label{subsec:tw2}
%\begin{table}[h!]
\begin{center}
\begin{tabular}{| l || c | c | c | c |}\hline
 & {\bf g.} & {\bf p.} & {\bf i.} & {\bf h.} \\\hline\hline
$\mu(\overline{tokens})$ & 3.84  & 3.94  & 3.86  & 3.82 \\
$\sigma(\overline{tokens})$ & 2.99  & 3.10  & 2.96  & 2.99 \\\hline
$\mu(\overline{knownw})$ & 3.28  & 3.27  & 3.32  & 3.26 \\
$\sigma(\overline{knownw})$ & 1.81  & 1.86  & 1.80  & 1.81 \\\hline
$\mu(\overline{knownw \neq})$ & 5.12  & 4.35  & 4.92  & 4.98 \\
$\sigma(\overline{knownw \neq})$ & 2.20  & 2.22  & 2.25  & 2.15 \\\hline
$\mu(\overline{stopw})$ & 1.77  & 1.84  & 1.77  & 1.76 \\
$\sigma(\overline{stopw})$ & 0.74  & 0.76  & 0.73  & 0.75 \\\hline
\end{tabular}
\caption{Token sizes in each Erd\"os sector ({{\bf p.}} for periphery, {{\bf i.}} for intermediary, {{\bf h.}} for hubs).}
\end{center}
\end{table}
Sizes of known words are smaller for hubs, which suggests its use of more common words, although some of the previous results suggests that hubs have a very differentiated and specialized vocabulary. 
Larger words seems to be related to intermediary sector,
which might be related to the use of elaborated vocabulary.
Further details are given in Table~\ref{tab:sizesTokens}.

\subsection{Sizes of sentences}\label{subsec:ss}
\begin{table}[h!]
\begin{center}
\begin{tabular}{| l || c | c | c | c |}\hline
 & {\bf g.} & {\bf p.} & {\bf i.} & {\bf h.} \\\hline\hline
$sents$ & 558  & 45  & 211  & 304 \\
$sents_{\%}$ & 99.64  & 8.04  & 37.68  & 54.29 \\\hline
$\mu_S(chars)$ & 147.51  & 158.07  & 132.44  & 155.44 \\
$\sigma_S(chars)$ & 147.95  & 154.11  & 135.56  & 154.01 \\\hline
$\mu_S(tokens)$ & 32.21  & 34.22  & 28.75  & 34.09 \\
$\sigma_S(tokens)$ & 32.08  & 32.64  & 29.17  & 33.60 \\\hline
$\mu_S(knownw)$ & 9.90  & 9.82  & 9.10  & 10.39 \\
$\sigma_S(knownw)$ & 10.37  & 11.34  & 9.14  & 10.92 \\\hline
$\mu_S(stopw)$ & 1.18  & 1.11  & 1.15  & 1.20 \\
$\sigma_S(stopw)$ & 1.57  & 1.22  & 1.52  & 1.64 \\\hline
$\mu_S(puncts)$ & 7.65  & 7.67  & 6.57  & 8.35 \\
$\sigma_S(puncts)$ & 11.30  & 8.19  & 10.69  & 11.98 \\\hline
\end{tabular}
\caption{Sentences sizes in each Erd\"os sector ({{\bf p.}} for periphery, {{\bf i.}} for intermediary, {{\bf h.}} for hubs).}
\end{center}
\end{table}
Hubs present the lowest average sentence size,
both in characters and in tokens.
We hypothesize that this smaller sentence use
is related to the efficiency of hub specialization.
Also, the incidence of usual known words seems to decay with connectivity, as does the number of known words with wordnet synsets.
This reflects our view that connectivity is inversely proportional
to diversity.

Further information is given in Table~\ref{tab:sizesSents}.

\subsection{Messages}\label{subsec:mm}
\begin{table}[h!]
\begin{center}
	\caption{Messages sizes in each Erd\"os sector ({{\bf p.}} for periphery, {{\bf i.}} for intermediary, {{\bf h.}} for hubs). TAG: 0}\label{tab:sizesMsgs}
\begin{tabular}{| l || c | c | c | c |}\hline
 & {\bf g.} & {\bf p.} & {\bf i.} & {\bf h.} \\\hline\hline
$msgs$ & 1992  & 286  & 841  & 865 \\
$msgs_{\%}$ & 100.00  & 14.36  & 42.22  & 43.42 \\\hline
$\mu_M(sents)$ & 5.21  & 6.08  & 6.43  & 3.74 \\
$\sigma_M(sents)$ & 6.78  & 4.03  & 9.40  & 3.26 \\\hline
$\mu_M(tokens)$ & 145.82  & 230.45  & 186.07  & 78.71 \\
$\sigma_M(tokens)$ & 260.61  & 291.17  & 326.68  & 127.13 \\\hline
$\mu_M(knownw)$ & 38.83  & 56.29  & 48.87  & 23.29 \\
$\sigma_M(knownw)$ & 50.54  & 58.28  & 58.67  & 31.16 \\\hline
$\mu_M(stopw)$ & 34.29  & 41.96  & 42.42  & 23.84 \\
$\sigma_M(stopw)$ & 41.11  & 32.32  & 52.81  & 25.35 \\\hline
$\mu_M(puncts)$ & 36.34  & 66.11  & 47.66  & 15.49 \\
$\sigma_M(puncts)$ & 103.42  & 114.84  & 135.49  & 39.61 \\\hline
$\mu_M(chars)$ & 637.40  & 977.77  & 811.14  & 355.94 \\
$\sigma_M(chars)$ & 1054.36  & 1195.70  & 1290.46  & 566.92 \\\hline
\end{tabular}
\begin{flushleft}
		Source: By the author.\
\end{flushleft}
\end{center}
\end{table}

Connectivity was related to smaller messages in terms of characters and tokens.
% tirar medidas de correlacao: perason e spearmanA TTM
ELE list displayed an inverse situation:
the more connected the sector,
the longer the messages are.
This was considered a peculiarity of the culture bonded
with the political subject of ELE list, to be further verified.
% make some sort of verification TTM
% process the lists just for this feature
Regarding sentences,
the size of messages seem to hold steady throughout connectivity.
% ver a correlação das medidas de grau e força com TTM
% texto
% ver em especial se algo eh mais para in ou out
% e se indifere greu e força como no particionamento de erdos
Further information is given in Table~\ref{tab:sizesMsgs}.

\subsection{POS tags}\label{subsec:pos}
\begin{table}[h!]
\begin{center}
\caption{POS tags in each Erd\"os sector ({{\bf p.}} for periphery, {{\bf i.}} for intermediary, {{\bf h.}} for hubs).
    Universal POS tags~\cite{{petrov}}:
    VERB - verbs (all tenses and modes);
    NOUN - nouns (common and proper);
    PRON - pronouns;
    ADJ - adjectives;
    ADV - adverbs;
    ADP - adpositions (prepositions and postpositions);
    CONJ - conjunctions;
    DET - determiners;
    NUM - cardinal numbers;
    PRT - particles or other function words;
    X - other: foreign words, typos, abbreviations;
    PUNCT - punctuation.
	TAG: 13}\label{tab:pos}
\begin{tabular}{| l || c | c | c | c |}\hline
 & {\bf g.} & {\bf p.} & {\bf i.} & {\bf h.} \\\hline\hline
NOUN & 51.86  & 63.77  & 48.31  & 37.37 \\
X & 0.08  & 0.14  & 0.02  & 0.07 \\\hline
ADP & 7.25  & 5.23  & 7.86  & 9.69 \\
DET & 7.48  & 6.47  & 7.43  & 9.28 \\\hline
VERB & 16.93  & 11.93  & 20.01  & 20.54 \\\hline
ADJ & 3.97  & 3.37  & 3.83  & 5.18 \\
ADV & 4.02  & 2.41  & 4.45  & 6.05 \\\hline
PRT & 3.17  & 3.98  & 2.37  & 3.05 \\
PRON & 3.16  & 1.29  & 3.64  & 5.55 \\
NUM & 0.43  & 0.30  & 0.38  & 0.74 \\
CONJ & 1.65  & 1.11  & 1.70  & 2.49 \\\hline
\end{tabular}
\begin{flushleft}
		Source: Prepared by the author.\
\end{flushleft}
\end{center}
\end{table}

% rever taggeamento e tentar melhor accuracy com o Bril Tagger TTM
Lower connectivity yields more nouns and less adjectives,
adverbs and verbs.
This suggests that the networks collect meaningful issues
through the peripheral sector. 
These issues are qualified, elaborated about,
by the more connected participants.
This is a further indicative that peripheral sectors
are related to diversity while hubs relate to specialization.
Further information is given in Table~\ref{tab:pos}.

\subsection{Wordnet synsets}\label{subsec:pos}
\begin{table}[h!]
\begin{center}
\begin{tabular}{| l || c | c | c | c |}\hline
 & {\bf g.} & {\bf p.} & {\bf i.} & {\bf h.} \\\hline\hline
N & 88.79  & 91.15  & 87.41  & 89.26 \\\hline
ADJ & 6.11  & 7.08  & 5.34  & 6.42 \\\hline
VERB & 0.24  & 0.00  & 0.38  & 0.19 \\\hline
ADV & 4.86  & 1.77  & 6.86  & 4.13 \\\hline\hline
POS & 21.05  & 22.01  & 21.61  & 20.58 \\\hline
POS! & 72.54  & 74.83  & 71.48  & 72.85 \\\hline
\end{tabular}
\caption{Percentage of synsets with each of the POS tags used by Wordnet. The last lines give the percentage of words considered from all of the tokens (POS) and from the words with synset (POS!). The tokens not considered are punctuations, unrecognized words, words without synsets, stopwords and words for which Wordnet has no synset  tagged with POS tags . Values for each Erd\"os sectors are in the columns {{\bf p.}} for periphery, {{\bf i.}} for intermediary, {{\bf h.}} for hubs.}
\end{center}
\end{table} % uma soh
%\input{tables/wnInline_} % uma soh


% POS -> n
% fname -> /home/r/repos/artigoTextoNasRedes/tables/wnPOSInline2-n_.tex
\begin{table}[h!]
\begin{center}
\begin{tabular}{| l || c | c | c | c |}\hline
 & {\bf g.} & {\bf p.} & {\bf i.} & {\bf h.} \\\hline\hline
$\mu(min\,depth)$ & 6.52  & 6.41  & 6.52  & 6.53 \\
$\sigma(min\,depth)$ & 1.94  & 1.77  & 2.01  & 1.93 \\\hline
$\mu(max\,depth)$ & 7.02  & 6.99  & 7.00  & 7.03 \\
$\sigma(max\,depth)$ & 2.13  & 1.99  & 2.20  & 2.11 \\\hline
$\mu(holonyms)$ & 0.65  & 0.73  & 0.65  & 0.63 \\
$\sigma(holonyms)$ & 1.41  & 1.48  & 1.39  & 1.40 \\\hline
$\mu(meronyms)$ & 1.08  & 0.83  & 0.86  & 1.25 \\
$\sigma(meronyms)$ & 4.66  & 2.72  & 4.02  & 5.22 \\\hline
$\mu(domains)$ & 0.07  & 0.07  & 0.10  & 0.06 \\
$\sigma(domains)$ & 0.27  & 0.26  & 0.30  & 0.25 \\\hline
$\mu(similar)$ & 0.00  & 0.00  & 0.00  & 0.00 \\
$\sigma(similar)$ & 0.00  & 0.00  & 0.00  & 0.00 \\\hline
$\mu(verb\,groups)$ & 0.00  & 0.00  & 0.00  & 0.00 \\
$\sigma(verb\,groups)$ & 0.00  & 0.00  & 0.00  & 0.00 \\\hline
$\mu(lemmas)$ & 3.07  & 2.80  & 3.19  & 3.04 \\
$\sigma(lemmas)$ & 2.14  & 1.74  & 2.32  & 2.08 \\\hline
$\mu(entailments)$ & 0.00  & 0.00  & 0.00  & 0.00 \\
$\sigma(entailments)$ & 0.00  & 0.00  & 0.00  & 0.00 \\\hline
$\mu(hyponyms)$ & 3.42  & 3.30  & 3.02  & 3.68 \\
$\sigma(hyponyms)$ & 13.17  & 21.19  & 7.42  & 14.14 \\\hline
$\mu(hypernyms)$ & 1.09  & 1.09  & 1.08  & 1.09 \\
$\sigma(hypernyms)$ & 0.30  & 0.34  & 0.28  & 0.31 \\\hline
\end{tabular}
\caption{Measures of wordnet features in each Erd\"os sector ({{\bf p.}} for periphery, {{\bf i.}} for intermediary, {{\bf h.}} for hubs).}
\end{center}
\end{table}
% fname -> /home/r/repos/artigoTextoNasRedes/tables/wnPOSInline2a-n_.tex
\begin{table}[h!]
\begin{center}
\begin{tabular}{| l || c | c | c | c |}\hline
 & {\bf g.} & {\bf p.} & {\bf i.} & {\bf h.} \\\hline\hline
entity.n.01 & 100.00  & 100.00  & 100.00  & 100.00 \\\hline\hline
{{\bf total}} & 100.00  & 100.00  & 100.00  & 100.00 \\\hline
\end{tabular}
\caption{Counts for the most incident synsets at the semantic roots in each Erd\"os sector ({\bf p.} for periphery, {\bf i.} for intermediary, {\bf h.} for hubs). Yes.}
\end{center}
\end{table}
% fname -> /home/r/repos/artigoTextoNasRedes/tables/wnPOSInline2b-n_.tex
\begin{table}[h!]
\begin{center}
\begin{tabular}{| l || c | c | c | c |}\hline
 & {\bf g.} & {\bf p.} & {\bf i.} & {\bf h.} \\\hline\hline
abstraction.n.06 & 58.86  & 62.14  & 55.58  & 60.29 \\\hline
physical\_entity.n.01 & 41.14  & 37.86  & 44.42  & 39.71 \\\hline\hline
{{\bf total}} & 100.00  & 100.00  & 100.00  & 100.00 \\\hline
\end{tabular}
\caption{Counts for the most incident synsets one step from the semantic roots in each Erd\"os sector ({\bf p.} for periphery, {\bf i.} for intermediary, {\bf h.} for hubs).}
\end{center}
\end{table}
% fname -> /home/r/repos/artigoTextoNasRedes/tables/wnPOSInline2c-n_.tex
\begin{table}[h!]
\begin{center}
\begin{tabular}{| l || c | c | c | c |}\hline
 & {\bf g.} & {\bf p.} & {\bf i.} & {\bf h.} \\\hline\hline
matter.n.03 & 24.03  & 22.33  & 24.96  & 23.74 \\\hline
communication.n.02 & 11.97  & 15.86  & 11.26  & 11.76 \\\hline
group.n.01 & 11.70  & 13.59  & 9.42  & 12.76 \\\hline
relation.n.01 & 10.12  & 9.39  & 9.51  & 10.61 \\\hline
object.n.01 & 8.99  & 4.85  & 11.08  & 8.40 \\\hline
psychological\_feature.n.01 & 8.99  & 6.80  & 8.55  & 9.61 \\\hline
measure.n.02 & 8.43  & 7.77  & 9.42  & 7.93 \\\hline
attribute.n.02 & 7.65  & 8.74  & 7.42  & 7.62 \\\hline
causal\_agent.n.01 & 5.06  & 7.12  & 5.15  & 4.67 \\\hline
thing.n.12 & 3.04  & 3.24  & 3.23  & 2.89 \\\hline
process.n.06 & 0.03  & 0.32  & 0.00  & 0.00 \\\hline\hline
{{\bf total}} & 100.00  & 100.00  & 100.00  & 100.00 \\\hline
\end{tabular}
\caption{Counts for the most incident synsets two step from the semantic roots in each Erd\"os sector ({\bf p.} for periphery, {\bf i.} for intermediary, {\bf h.} for hubs).}
\end{center}
\end{table}
% fname -> /home/r/repos/artigoTextoNasRedes/tables/wnPOSInline2d-n_.tex
\begin{table}[h!]
\begin{center}
\begin{tabular}{| l || c | c | c | c |}\hline
 & {\bf g.} & {\bf p.} & {\bf i.} & {\bf h.} \\\hline\hline
substance.n.01 & 26.49  & 26.15  & 27.39  & 26.01 \\\hline
position.n.07 & 10.58  & 9.62  & 10.37  & 10.86 \\\hline
definite\_quantity.n.01 & 8.40  & 5.77  & 9.23  & 8.32 \\\hline
state.n.02 & 8.25  & 10.00  & 8.30  & 7.95 \\\hline
event.n.01 & 7.62  & 5.77  & 7.16  & 8.19 \\\hline
whole.n.02 & 7.37  & 2.69  & 9.23  & 7.01 \\\hline
social\_group.n.01 & 7.05  & 9.23  & 5.71  & 7.51 \\\hline
written\_communication.n.01 & 6.98  & 8.85  & 5.91  & 7.32 \\\hline
person.n.01 & 5.71  & 8.08  & 5.91  & 5.21 \\\hline
message.n.02 & 5.29  & 8.46  & 4.36  & 5.34 \\\hline
body\_of\_water.n.01 & 3.21  & 3.08  & 3.42  & 3.10 \\\hline
cognition.n.01 & 3.03  & 2.31  & 3.01  & 3.17 \\\hline\hline
{{\bf total}} & 100.00  & 100.00  & 100.00  & 100.00 \\\hline
\end{tabular}
\caption{Counts for the most incident synsets three step from the semantic roots in each Erd\"os sector ({\bf p.} for periphery, {\bf i.} for intermediary, {\bf h.} for hubs).}
\end{center}
\end{table}

% POS -> as
% fname -> /home/r/repos/artigoTextoNasRedes/tables/wnPOSInline2-as_.tex
\begin{table}[h!]
\begin{center}
\begin{tabular}{| l || c | c | c | c |}\hline
 & {\bf g.} & {\bf p.} & {\bf i.} & {\bf h.} \\\hline\hline
$\mu(min\,depth)$ & 0.00  & 0.00  & 0.00  & 0.00 \\
$\sigma(min\,depth)$ & 0.00  & 0.00  & 0.00  & 0.00 \\\hline
$\mu(max\,depth)$ & 0.00  & 0.00  & 0.00  & 0.00 \\
$\sigma(max\,depth)$ & 0.00  & 0.00  & 0.00  & 0.00 \\\hline
$\mu(holonyms)$ & 0.00  & 0.00  & 0.00  & 0.00 \\
$\sigma(holonyms)$ & 0.00  & 0.00  & 0.00  & 0.00 \\\hline
$\mu(meronyms)$ & 0.00  & 0.00  & 0.00  & 0.00 \\
$\sigma(meronyms)$ & 0.00  & 0.00  & 0.00  & 0.00 \\\hline
$\mu(domains)$ & 0.02  & 0.00  & 0.03  & 0.02 \\
$\sigma(domains)$ & 0.15  & 0.00  & 0.17  & 0.15 \\\hline
$\mu(similar)$ & 4.29  & 4.00  & 3.63  & 4.68 \\
$\sigma(similar)$ & 4.41  & 3.50  & 3.25  & 4.99 \\\hline
$\mu(verb\,groups)$ & 0.00  & 0.00  & 0.00  & 0.00 \\
$\sigma(verb\,groups)$ & 0.00  & 0.00  & 0.00  & 0.00 \\\hline
$\mu(lemmas)$ & 2.07  & 2.08  & 2.49  & 1.85 \\
$\sigma(lemmas)$ & 1.79  & 1.50  & 2.06  & 1.64 \\\hline
$\mu(entailments)$ & 0.00  & 0.00  & 0.00  & 0.00 \\
$\sigma(entailments)$ & 0.00  & 0.00  & 0.00  & 0.00 \\\hline
$\mu(hyponyms)$ & 0.00  & 0.00  & 0.00  & 0.00 \\
$\sigma(hyponyms)$ & 0.00  & 0.00  & 0.00  & 0.00 \\\hline
$\mu(hypernyms)$ & 0.00  & 0.00  & 0.00  & 0.00 \\
$\sigma(hypernyms)$ & 0.00  & 0.00  & 0.00  & 0.00 \\\hline
\end{tabular}
\caption{Measures of wordnet features in each Erd\"os sector ({{\bf p.}} for periphery, {{\bf i.}} for intermediary, {{\bf h.}} for hubs).}
\end{center}
\end{table}
% fname -> /home/r/repos/artigoTextoNasRedes/tables/wnPOSInline2a-as_.tex
\begin{table}[h!]
\begin{center}
\begin{tabular}{| l || c | c | c | c |}\hline
 & {\bf g.} & {\bf p.} & {\bf i.} & {\bf h.} \\\hline\hline
public.a.01 & 40.34  & 57.14  & 33.33  & 40.59 \\\hline
temporal.s.01 & 16.48  & 0.00  & 11.11  & 22.77 \\\hline
chief.s.01 & 13.64  & 0.00  & 24.07  & 10.89 \\\hline
new.a.01 & 6.25  & 4.76  & 1.85  & 8.91 \\\hline
global.s.01 & 4.55  & 9.52  & 5.56  & 2.97 \\\hline
ocular.a.02 & 3.98  & 19.05  & 1.85  & 1.98 \\\hline
variable.a.01 & 3.41  & 0.00  & 1.85  & 4.95 \\\hline
impermanent.a.01 & 2.84  & 0.00  & 5.56  & 1.98 \\\hline
simple.a.01 & 2.27  & 4.76  & 3.70  & 0.99 \\\hline
alive.a.01 & 2.27  & 0.00  & 7.41  & 0.00 \\\hline
standard.a.01 & 2.27  & 0.00  & 0.00  & 3.96 \\\hline
virtual.s.01 & 1.70  & 4.76  & 3.70  & 0.00 \\\hline\hline
{{\bf total}} & 100.00  & 100.00  & 100.00  & 100.00 \\\hline
\end{tabular}
\caption{Counts for the most incident synsets at the semantic roots in each Erd\"os sector ({\bf p.} for periphery, {\bf i.} for intermediary, {\bf h.} for hubs). Yes.}
\end{center}
\end{table}

% POS -> v
% fname -> /home/r/repos/artigoTextoNasRedes/tables/wnPOSInline2-v_.tex
\begin{table}[h!]
\begin{center}
\begin{tabular}{| l || c | c | c | c |}\hline
 & {\bf g.} & {\bf p.} & {\bf i.} & {\bf h.} \\\hline\hline
$\mu(min\,depth)$ & 1.72  & 2.83  & 1.70  & 1.66 \\
$\sigma(min\,depth)$ & 1.38  & 1.77  & 1.17  & 1.51 \\\hline
$\mu(max\,depth)$ & 1.72  & 2.83  & 1.70  & 1.66 \\
$\sigma(max\,depth)$ & 1.38  & 1.77  & 1.17  & 1.51 \\\hline
$\mu(holonyms)$ & 0.00  & 0.00  & 0.00  & 0.00 \\
$\sigma(holonyms)$ & 0.00  & 0.00  & 0.00  & 0.00 \\\hline
$\mu(meronyms)$ & 0.00  & 0.00  & 0.00  & 0.00 \\
$\sigma(meronyms)$ & 0.00  & 0.00  & 0.00  & 0.00 \\\hline
$\mu(domains)$ & 0.21  & 0.00  & 0.26  & 0.18 \\
$\sigma(domains)$ & 0.41  & 0.00  & 0.44  & 0.39 \\\hline
$\mu(similar)$ & 0.00  & 0.00  & 0.00  & 0.00 \\
$\sigma(similar)$ & 0.00  & 0.00  & 0.00  & 0.00 \\\hline
$\mu(verb\,groups)$ & 0.23  & 0.50  & 0.22  & 0.23 \\
$\sigma(verb\,groups)$ & 0.48  & 0.50  & 0.49  & 0.47 \\\hline
$\mu(lemmas)$ & 3.42  & 1.67  & 3.49  & 3.48 \\
$\sigma(lemmas)$ & 2.09  & 0.75  & 1.97  & 2.23 \\\hline
$\mu(entailments)$ & 0.12  & 0.00  & 0.16  & 0.09 \\
$\sigma(entailments)$ & 0.32  & 0.00  & 0.36  & 0.29 \\\hline
$\mu(hyponyms)$ & 6.52  & 3.00  & 7.97  & 5.27 \\
$\sigma(hyponyms)$ & 13.64  & 2.71  & 18.33  & 6.38 \\\hline
$\mu(hypernyms)$ & 0.78  & 0.83  & 0.82  & 0.73 \\
$\sigma(hypernyms)$ & 0.42  & 0.37  & 0.38  & 0.45 \\\hline
\end{tabular}
\caption{Measures of wordnet features in each Erd\"os sector ({{\bf p.}} for periphery, {{\bf i.}} for intermediary, {{\bf h.}} for hubs).}
\end{center}
\end{table}
% fname -> /home/r/repos/artigoTextoNasRedes/tables/wnPOSInline2a-v_.tex
\begin{table}[h!]
\begin{center}
\begin{tabular}{| l || c | c | c | c |}\hline
 & {\bf g.} & {\bf p.} & {\bf i.} & {\bf h.} \\\hline\hline
think.v.03 & 24.52  & 0.00  & 25.00  & 25.33 \\\hline
act.v.01 & 16.13  & 75.00  & 13.16  & 16.00 \\\hline
change.v.01 & 12.26  & 0.00  & 11.84  & 13.33 \\\hline
change.v.02 & 11.61  & 25.00  & 11.84  & 10.67 \\\hline
include.v.01 & 9.68  & 0.00  & 10.53  & 9.33 \\\hline
make.v.03 & 7.74  & 0.00  & 11.84  & 4.00 \\\hline
affect.v.05 & 3.23  & 0.00  & 2.63  & 4.00 \\\hline
work.v.01 & 3.23  & 0.00  & 0.00  & 6.67 \\\hline
be.v.01 & 3.23  & 0.00  & 1.32  & 5.33 \\\hline
travel.v.01 & 3.23  & 0.00  & 5.26  & 1.33 \\\hline
use.v.01 & 2.58  & 0.00  & 3.95  & 1.33 \\\hline
insist.v.01 & 2.58  & 0.00  & 2.63  & 2.67 \\\hline\hline
{{\bf total}} & 100.00  & 100.00  & 100.00  & 100.00 \\\hline
\end{tabular}
\caption{Counts for the most incident synsets at the semantic roots in each Erd\"os sector ({\bf p.} for periphery, {\bf i.} for intermediary, {\bf h.} for hubs). Yes.}
\end{center}
\end{table}
% fname -> /home/r/repos/artigoTextoNasRedes/tables/wnPOSInline2b-v_.tex
\begin{table}[h!]
\begin{center}
\begin{tabular}{| l || c | c | c | c |}\hline
 & {\bf g.} & {\bf p.} & {\bf i.} & {\bf h.} \\\hline\hline
reason.v.01 & 27.73  & 0.00  & 28.33  & 28.57 \\\hline
change\_shape.v.01 & 13.45  & 0.00  & 13.33  & 14.29 \\\hline
end.v.02 & 13.45  & 0.00  & 13.33  & 14.29 \\\hline
interact.v.01 & 13.45  & 100.00  & 10.00  & 12.50 \\\hline
try.v.01 & 7.56  & 0.00  & 6.67  & 8.93 \\\hline
create\_verbally.v.01 & 5.04  & 0.00  & 10.00  & 0.00 \\\hline
surprise.v.01 & 4.20  & 0.00  & 3.33  & 5.36 \\\hline
assert.v.01 & 3.36  & 0.00  & 3.33  & 3.57 \\\hline
evaluate.v.02 & 3.36  & 0.00  & 3.33  & 3.57 \\\hline
specify.v.03 & 3.36  & 0.00  & 1.67  & 5.36 \\\hline
return.v.01 & 2.52  & 0.00  & 3.33  & 1.79 \\\hline
discard.v.01 & 2.52  & 0.00  & 3.33  & 1.79 \\\hline\hline
{{\bf total}} & 100.00  & 100.00  & 100.00  & 100.00 \\\hline
\end{tabular}
\caption{Counts for the most incident synsets one step from the semantic roots in each Erd\"os sector ({\bf p.} for periphery, {\bf i.} for intermediary, {\bf h.} for hubs).}
\end{center}
\end{table}
% fname -> /home/r/repos/artigoTextoNasRedes/tables/wnPOSInline2c-v_.tex
\begin{table}[h!]
\begin{center}
\begin{tabular}{| l || c | c | c | c |}\hline
 & {\bf g.} & {\bf p.} & {\bf i.} & {\bf h.} \\\hline\hline
deduce.v.01 & 32.67  & 0.00  & 32.69  & 35.56 \\\hline
start.v.14 & 15.84  & 0.00  & 15.38  & 17.78 \\\hline
interrupt.v.04 & 15.84  & 0.00  & 15.38  & 17.78 \\\hline
relate.v.05 & 9.90  & 25.00  & 9.62  & 8.89 \\\hline
write.v.01 & 5.94  & 0.00  & 11.54  & 0.00 \\\hline
catch.v.01 & 4.95  & 0.00  & 3.85  & 6.67 \\\hline
communicate.v.02 & 3.96  & 50.00  & 0.00  & 4.44 \\\hline
dump.v.01 & 2.97  & 0.00  & 3.85  & 2.22 \\\hline
treat.v.01 & 1.98  & 0.00  & 1.92  & 2.22 \\\hline
name.v.01 & 1.98  & 0.00  & 3.85  & 0.00 \\\hline
correct.v.01 & 1.98  & 25.00  & 1.92  & 0.00 \\\hline
represent.v.09 & 1.98  & 0.00  & 0.00  & 4.44 \\\hline\hline
{{\bf total}} & 100.00  & 100.00  & 100.00  & 100.00 \\\hline
\end{tabular}
\caption{Counts for the most incident synsets two step from the semantic roots in each Erd\"os sector ({\bf p.} for periphery, {\bf i.} for intermediary, {\bf h.} for hubs).}
\end{center}
\end{table}
% fname -> /home/r/repos/artigoTextoNasRedes/tables/wnPOSInline2d-v_.tex
\begin{table}[h!]
\begin{center}
\begin{tabular}{| l || c | c | c | c |}\hline
 & {\bf g.} & {\bf p.} & {\bf i.} & {\bf h.} \\\hline\hline
disrespect.v.01 & 35.71  & 25.00  & 45.45  & 30.77 \\\hline
inform.v.01 & 14.29  & 50.00  & 0.00  & 15.38 \\\hline
map.v.01 & 7.14  & 0.00  & 0.00  & 15.38 \\\hline
object.v.01 & 7.14  & 0.00  & 9.09  & 7.69 \\\hline
ignore.v.01 & 7.14  & 0.00  & 9.09  & 7.69 \\\hline
prefer.v.03 & 7.14  & 0.00  & 18.18  & 0.00 \\\hline
debug.v.01 & 7.14  & 25.00  & 9.09  & 0.00 \\\hline
double.v.01 & 3.57  & 0.00  & 0.00  & 7.69 \\\hline
adhere.v.06 & 3.57  & 0.00  & 0.00  & 7.69 \\\hline
program.v.01 & 3.57  & 0.00  & 0.00  & 7.69 \\\hline
roll\_up.v.02 & 3.57  & 0.00  & 9.09  & 0.00 \\\hline\hline
{{\bf total}} & 100.00  & 100.00  & 100.00  & 100.00 \\\hline
\end{tabular}
\caption{Counts for the most incident synsets three step from the semantic roots in each Erd\"os sector ({\bf p.} for periphery, {\bf i.} for intermediary, {\bf h.} for hubs).}
\end{center}
\end{table}

% POS -> r
% fname -> /home/r/repos/artigoTextoNasRedes/tables/wnPOSInline2-r_.tex
\begin{table}[h!]
\begin{center}
\begin{tabular}{| l || c | c | c | c |}\hline
 & {\bf g.} & {\bf p.} & {\bf i.} & {\bf h.} \\\hline\hline
$\mu(min\,depth)$ & 0.00  & nan  & 0.00  & 0.00 \\
$\sigma(min\,depth)$ & 0.00  & nan  & 0.00  & 0.00 \\\hline
$\mu(max\,depth)$ & 0.00  & nan  & 0.00  & 0.00 \\
$\sigma(max\,depth)$ & 0.00  & nan  & 0.00  & 0.00 \\\hline
$\mu(holonyms)$ & 0.00  & nan  & 0.00  & 0.00 \\
$\sigma(holonyms)$ & 0.00  & nan  & 0.00  & 0.00 \\\hline
$\mu(meronyms)$ & 0.00  & nan  & 0.00  & 0.00 \\
$\sigma(meronyms)$ & 0.00  & nan  & 0.00  & 0.00 \\\hline
$\mu(domains)$ & 0.11  & nan  & 0.20  & 0.00 \\
$\sigma(domains)$ & 0.31  & nan  & 0.40  & 0.00 \\\hline
$\mu(similar)$ & 0.00  & nan  & 0.00  & 0.00 \\
$\sigma(similar)$ & 0.00  & nan  & 0.00  & 0.00 \\\hline
$\mu(verb\,groups)$ & 0.00  & nan  & 0.00  & 0.00 \\
$\sigma(verb\,groups)$ & 0.00  & nan  & 0.00  & 0.00 \\\hline
$\mu(lemmas)$ & 2.00  & nan  & 1.20  & 3.00 \\
$\sigma(lemmas)$ & 1.33  & nan  & 0.40  & 1.41 \\\hline
$\mu(entailments)$ & 0.00  & nan  & 0.00  & 0.00 \\
$\sigma(entailments)$ & 0.00  & nan  & 0.00  & 0.00 \\\hline
$\mu(hyponyms)$ & 0.00  & nan  & 0.00  & 0.00 \\
$\sigma(hyponyms)$ & 0.00  & nan  & 0.00  & 0.00 \\\hline
$\mu(hypernyms)$ & 0.00  & nan  & 0.00  & 0.00 \\
$\sigma(hypernyms)$ & 0.00  & nan  & 0.00  & 0.00 \\\hline
\end{tabular}
\caption{Measures of wordnet features in each Erd\"os sector ({{\bf p.}} for periphery, {{\bf i.}} for intermediary, {{\bf h.}} for hubs).}
\end{center}
\end{table}
% fname -> /home/r/repos/artigoTextoNasRedes/tables/wnPOSInline2a-r_.tex
\begin{table}[h!]
\begin{center}
\begin{tabular}{| l || c | c | c | c |}\hline
 & {\bf g.} & {\bf p.} & {\bf i.} & {\bf h.} \\\hline\hline
still.r.01 & 44.44  & nan  & 80.00  & 0.00 \\\hline
promptly.r.02 & 22.22  & nan  & 0.00  & 50.00 \\\hline
overhead.r.01 & 11.11  & nan  & 0.00  & 25.00 \\\hline
well.r.01 & 11.11  & nan  & 20.00  & 0.00 \\\hline
entirely.r.02 & 11.11  & nan  & 0.00  & 25.00 \\\hline\hline
{{\bf total}} & 100.00  & nan  & 100.00  & 100.00 \\\hline
\end{tabular}
\caption{Counts for the most incident synsets at the semantic roots in each Erd\"os sector ({\bf p.} for periphery, {\bf i.} for intermediary, {\bf h.} for hubs). Yes.}
\end{center}
\end{table} % uma soh

%\input{tables/wnInline2-n_} % uma para cada nivel do synset
%\input{tables/wnInline2a-n_}
%\input{tables/wnInline2b-n_}
%\input{tables/wnInline2c-n_}
%\input{tables/wnInline2d-n_}
%
%\input{tables/wnInline2-as_} % uma para cada nivel do synset
%\input{tables/wnInline2a-as_}
%%\input{tables/wnInline2b-as_}
%%\input{tables/wnInline2c-as_}
%%\input{tables/wnInline2d-as_}
%
%\input{tables/wnInline2-v_} % uma para cada nivel do synset
%\input{tables/wnInline2a-v_}
%\input{tables/wnInline2b-v_}
%\input{tables/wnInline2c-v_}
%\input{tables/wnInline2d-v_}
%
%\input{tables/wnInline2-r_} % uma para cada nivel do synset
%\input{tables/wnInline2a-r_}
%%\input{tables/wnInline2b-r_}
%%\input{tables/wnInline2c-r_}
%%\input{tables/wnInline2d-r_}





\subsection{Differentiation of the texts from Erd\"os sectors}\label{subsec:di}
\begin{table}[h!]
\begin{center}
\caption{KS distances on size of tokens. TAG: 6}
	\label{tab:kolTok}
\begin{tabular}{| l || c | c | c | c |}\hline
 & {\bf g.} & {\bf p.} & {\bf i.} & {\bf h.} \\\hline\hline
{\bf g.} & 0.000 & 4.327 & 17.168 & 7.851 \\
{\bf a } & 0.000 & 0.014 & 0.115 & 0.044 \\\hline
{\bf p.} & 4.327 & 0.000 & 18.907 & 7.833 \\
{\bf } & 0.014 & 0.000 & 0.129 & 0.045 \\\hline
{\bf i.} & 17.168 & 18.907 & 0.000 & 15.540 \\
{\bf } & 0.115 & 0.129 & 0.000 & 0.129 \\\hline
{\bf h.} & 7.851 & 7.833 & 15.540 & 0.000 \\
{\bf } & 0.044 & 0.045 & 0.129 & 0.000 \\\hline
\end{tabular}
\begin{flushleft}
		Source: Prepared by the author.\
\end{flushleft}
\end{center}
\end{table}

\begin{table}[h!]
\begin{center}
\caption{KS distances on size of known words. TAG: 1}
\begin{tabular}{| l || c | c | c | c |}\hline
 & {\bf g.} & {\bf p.} & {\bf i.} & {\bf h.} \\\hline\hline
{\bf g.} & 0.000 & 5.904 & 5.264 & 5.549 \\
{\bf } & 0.000 & 0.043 & 0.040 & 0.150 \\\hline
{\bf p.} & 5.904 & 0.000 & 9.547 & 7.073 \\
{\bf } & 0.043 & 0.000 & 0.083 & 0.193 \\\hline
{\bf i.} & 5.264 & 9.547 & 0.000 & 4.058 \\
{\bf } & 0.040 & 0.083 & 0.000 & 0.111 \\\hline
{\bf h.} & 5.549 & 7.073 & 4.058 & 0.000 \\
{\bf } & 0.150 & 0.193 & 0.111 & 0.000 \\\hline
\end{tabular}
\begin{flushleft}
		Source: Prepared by the author.\
\end{flushleft}
\end{center}
\end{table} 

\begin{table}[h!]
\begin{center}
\begin{tabular}{| l || c | c | c | c |}\hline
 & {\bf g.} & {\bf p.} & {\bf i.} & {\bf h.} \\\hline\hline
{\bf g.} & 0.000 & 1.192 & 1.491 & 1.551 \\
{\bf } & 0.000 & 0.026 & 0.073 & 0.047 \\\hline
{\bf p.} & 1.192 & 0.000 & 1.977 & 2.194 \\
{\bf } & 0.026 & 0.000 & 0.098 & 0.070 \\\hline
{\bf i.} & 1.491 & 1.977 & 0.000 & 2.078 \\
{\bf } & 0.073 & 0.098 & 0.000 & 0.113 \\\hline
{\bf h.} & 1.551 & 2.194 & 2.078 & 0.000 \\
{\bf } & 0.047 & 0.070 & 0.113 & 0.000 \\\hline
\end{tabular}
\caption{KS distances on size of sentences. TAG: 6}
\end{center}
\end{table}

\begin{table}[h!]
\begin{center}
\caption{KS distances on use of adjectives on sentences. TAG: 3}
\begin{tabular}{| l || c | c | c | c |}\hline
 & {\bf g.} & {\bf p.} & {\bf i.} & {\bf h.} \\\hline\hline
{\bf g.} & 0.000 & 0.461 & 0.564 & 0.617 \\
{\bf } & 0.000 & 0.011 & 0.010 & 0.010 \\\hline
{\bf p.} & 0.461 & 0.000 & 0.385 & 0.800 \\
{\bf } & 0.011 & 0.000 & 0.011 & 0.021 \\\hline
{\bf i.} & 0.564 & 0.385 & 0.000 & 0.986 \\
{\bf } & 0.010 & 0.011 & 0.000 & 0.020 \\\hline
{\bf h.} & 0.617 & 0.800 & 0.986 & 0.000 \\
{\bf } & 0.010 & 0.021 & 0.020 & 0.000 \\\hline
\end{tabular}
\begin{flushleft}
		Source: Prepared by the author.\
\end{flushleft}
\end{center}
\end{table}

\begin{table}[h!]
\begin{center}
\caption{KS distances on use of substantives on sentences. TAG: 1}
	\label{tab:kolSub}
\begin{tabular}{| l || c | c | c | c |}\hline
 & {\bf g.} & {\bf p.} & {\bf i.} & {\bf h.} \\\hline\hline
{\bf g.} & 0.000 & 0.642 & 1.791 & 6.936 \\
{\bf } & 0.000 & 0.023 & 0.067 & 0.537 \\\hline
{\bf p.} & 0.642 & 0.000 & 1.007 & 6.970 \\
{\bf } & 0.023 & 0.000 & 0.044 & 0.560 \\\hline
{\bf i.} & 1.791 & 1.007 & 0.000 & 7.510 \\
{\bf } & 0.067 & 0.044 & 0.000 & 0.607 \\\hline
{\bf h.} & 6.936 & 6.970 & 7.510 & 0.000 \\
{\bf } & 0.537 & 0.560 & 0.607 & 0.000 \\\hline
\end{tabular}
\end{center}
\begin{flushleft}
		Source: By the author.\
\end{flushleft}
\end{table}

\begin{table}[h!]
\begin{center}
\begin{tabular}{| l || c | c | c | c |}\hline
 & {\bf g.} & {\bf p.} & {\bf i.} & {\bf h.} \\\hline\hline
{\bf g.} & 0.000 & 1.484 & 0.978 & 1.277 \\
{\bf } & 0.000 & 0.036 & 0.017 & 0.020 \\\hline
{\bf p.} & 1.484 & 0.000 & 0.349 & 2.157 \\
{\bf } & 0.036 & 0.000 & 0.010 & 0.056 \\\hline
{\bf i.} & 0.978 & 0.349 & 0.000 & 1.739 \\
{\bf } & 0.017 & 0.010 & 0.000 & 0.035 \\\hline
{\bf h.} & 1.277 & 2.157 & 1.739 & 0.000 \\
{\bf } & 0.020 & 0.056 & 0.035 & 0.000 \\\hline
\end{tabular}
\caption{KS distances on use of punctuations on sentences. TAG: 3}
\end{center}
\end{table}

Results from our adaptation of the Kolmogorov-Smirnov test
suggest that the texts produced by each sector are extremely different.
Intermediary sectors sometimes exhibit greater differences 
from periphery and hubs than these extreme sectors themselves 
(Tables~\ref{tab:kolSub} and~\ref{tab:kolSw}).
This differentiation of the three sectors is a
strong indicative that the Erd\"os Sectioning
described in~\cite{evoSN} reveals meaningful
sectors of the networks.

Tables~\ref{tab:kolSub}-\ref{tab:kolPctInter}
illustrate two strong results:
\begin{itemize}
    \item Differences of textual production of the Erd\"os sectors are extreme.
	    This can be noticed from the high values on these tables,
	    beyond reference values used for the acceptance of the 
	    null hypothesis (see Section~\ref{sec:ks}).
    \item Differences between sectors on the same network 
	    (Tables~\ref{tab:kolSub},~\ref{tab:kolAdj},~\ref{tab:kolSw} and~\ref{tab:kolPct}) are greater than differences between same sector from distinct lists (Tables~\ref{tab:kolSubInter},~\ref{tab:kolAdjInter},~\ref{tab:kolSwInter} and~\ref{tab:kolPctInter}).
\end{itemize}

We can summarize these results stating that the extreme difference
found between the texts produced by the Erd\"os sectors
are greater than the difference found between texts from different
networks or from the same sector of different networks.

\subsection{Correlation of topological and textual metrics}\label{subsec:cor}

\begin{table*}[h!]
\begin{center}
\begin{tabular}{| l || c | c | c | c | c | c | c | c | c |}\hline
 & $cc$ & $d$ & $s$ & $\mu_S(p)$ & $\sigma_S(p)$ & $\mu_S(kw)$ & $\sigma_S(kw)$ & $\mu_S(sw)$ & $\sigma_S(sw)$ \\\hline\hline
$cc$ & {\bf 1.06} & 0.18 & 0.11 & -0.15 & 0.13 & 0.36 & 0.39 & -0.03 & 0.09 \\
(p.) & {\bf 1.17} & {\bf 0.85} & 0.20 & -0.12 & -0.04 & 0.47 & 0.41 & 0.06 & {\bf 0.62} \\
(i.) & {\bf 1.20} & -0.37 & -0.04 & -0.01 & 0.17 & 0.53 & {\bf 0.62} & -0.28 & -0.37 \\
(h.) & {\bf 1.33} & {\bf -1.21} & {\bf -0.83} & 0.33 & -0.41 & 0.17 & 0.52 & 0.51 & {\bf 0.66} \\\hline
$d$ & 0.18 & {\bf 1.06} & {\bf 0.99} & 0.18 & 0.53 & 0.38 & 0.27 & 0.26 & 0.57 \\
 & {\bf 0.85} & {\bf 1.17} & {\bf 0.93} & -0.11 & 0.02 & 0.57 & {\bf 0.60} & -0.17 & {\bf 0.75} \\
 & -0.37 & {\bf 1.20} & {\bf 0.87} & {\bf 0.61} & 0.44 & 0.32 & 0.21 & {\bf 1.02} & {\bf 1.04} \\
 & {\bf -1.21} & {\bf 1.33} & {\bf 0.84} & -0.43 & {\bf 0.77} & -0.30 & -0.55 & -0.55 & {\bf -0.71} \\\hline
$s$ & 0.11 & {\bf 0.99} & {\bf 1.06} & 0.10 & 0.41 & 0.18 & 0.11 & 0.08 & 0.34 \\
 & 0.20 & {\bf 0.93} & {\bf 1.17} & -0.06 & 0.07 & 0.40 & 0.51 & -0.30 & 0.54 \\
 & -0.04 & {\bf 0.87} & {\bf 1.20} & 0.27 & 0.21 & 0.08 & -0.15 & {\bf 0.85} & {\bf 0.91} \\
 & {\bf -0.83} & {\bf 0.84} & {\bf 1.33} & {\bf -1.22} & -0.23 & {\bf -1.14} & {\bf -1.28} & {\bf -1.28} & {\bf -1.32} \\\hline
$\mu_S(p)$ & -0.15 & 0.18 & 0.10 & {\bf 1.06} & {\bf 0.78} & {\bf 0.67} & 0.51 & {\bf 0.70} & 0.39 \\
 & -0.12 & -0.11 & -0.06 & {\bf 1.17} & {\bf 1.08} & {\bf 0.68} & 0.41 & {\bf 0.76} & 0.12 \\
 & -0.01 & {\bf 0.61} & 0.27 & {\bf 1.20} & {\bf 1.15} & {\bf 1.03} & {\bf 0.78} & {\bf 0.92} & {\bf 0.79} \\
 & 0.33 & -0.43 & {\bf -1.22} & {\bf 1.33} & 0.45 & {\bf 1.32} & {\bf 1.31} & {\bf 1.32} & {\bf 1.29} \\\hline
$\sigma_S(p)$ & 0.13 & 0.53 & 0.41 & {\bf 0.78} & {\bf 1.06} & {\bf 0.82} & {\bf 0.70} & 0.54 & 0.57 \\
 & -0.04 & 0.02 & 0.07 & {\bf 1.08} & {\bf 1.17} & {\bf 0.86} & {\bf 0.71} & 0.51 & 0.47 \\
 & 0.17 & 0.44 & 0.21 & {\bf 1.15} & {\bf 1.20} & {\bf 1.12} & {\bf 0.91} & {\bf 0.86} & {\bf 0.73} \\
 & -0.41 & {\bf 0.77} & -0.23 & 0.45 & {\bf 1.33} & 0.47 & 0.45 & 0.44 & 0.29 \\\hline
$\mu_S(kw)$ & 0.36 & 0.38 & 0.18 & {\bf 0.67} & {\bf 0.82} & {\bf 1.06} & {\bf 0.97} & {\bf 0.64} & {\bf 0.67} \\
 & 0.47 & 0.57 & 0.40 & {\bf 0.68} & {\bf 0.86} & {\bf 1.17} & {\bf 1.10} & 0.44 & {\bf 0.79} \\
 & 0.53 & 0.32 & 0.08 & {\bf 1.03} & {\bf 1.12} & {\bf 1.20} & {\bf 1.11} & {\bf 0.66} & 0.52 \\
 & 0.17 & -0.30 & {\bf -1.14} & {\bf 1.32} & 0.47 & {\bf 1.33} & {\bf 1.28} & {\bf 1.28} & {\bf 1.23} \\\hline
$\sigma_S(kw)$ & 0.39 & 0.27 & 0.11 & 0.51 & {\bf 0.70} & {\bf 0.97} & {\bf 1.06} & 0.36 & 0.48 \\
 & 0.41 & {\bf 0.60} & 0.51 & 0.41 & {\bf 0.71} & {\bf 1.10} & {\bf 1.17} & 0.10 & {\bf 0.92} \\
 & {\bf 0.62} & 0.21 & -0.15 & {\bf 0.78} & {\bf 0.91} & {\bf 1.11} & {\bf 1.20} & 0.43 & 0.31 \\
 & 0.52 & -0.55 & {\bf -1.28} & {\bf 1.31} & 0.45 & {\bf 1.28} & {\bf 1.33} & {\bf 1.33} & {\bf 1.32} \\\hline
$\mu_S(sw)$ & -0.03 & 0.26 & 0.08 & {\bf 0.70} & 0.54 & {\bf 0.64} & 0.36 & {\bf 1.06} & {\bf 0.76} \\
 & 0.06 & -0.17 & -0.30 & {\bf 0.76} & 0.51 & 0.44 & 0.10 & {\bf 1.17} & -0.19 \\
 & -0.28 & {\bf 1.02} & {\bf 0.85} & {\bf 0.92} & {\bf 0.86} & {\bf 0.66} & 0.43 & {\bf 1.20} & {\bf 1.18} \\
 & 0.51 & -0.55 & {\bf -1.28} & {\bf 1.32} & 0.44 & {\bf 1.28} & {\bf 1.33} & {\bf 1.33} & {\bf 1.32} \\\hline
$\sigma_S(sw)$ & 0.09 & 0.57 & 0.34 & 0.39 & 0.57 & {\bf 0.67} & 0.48 & {\bf 0.76} & {\bf 1.06} \\
 & {\bf 0.62} & {\bf 0.75} & 0.54 & 0.12 & 0.47 & {\bf 0.79} & {\bf 0.92} & -0.19 & {\bf 1.17} \\
 & -0.37 & {\bf 1.04} & {\bf 0.91} & {\bf 0.79} & {\bf 0.73} & 0.52 & 0.31 & {\bf 1.18} & {\bf 1.20} \\
 & {\bf 0.66} & {\bf -0.71} & {\bf -1.32} & {\bf 1.29} & 0.29 & {\bf 1.23} & {\bf 1.32} & {\bf 1.32} & {\bf 1.33} \\\hline
\end{tabular}
\caption{Pierson correlation coefficient for the topological and textual measures.}
\end{center}
\end{table*}
Correlation of degree 
and strength metrics is
substantially smaller for intermediary sector.
% verificar e quantificar TTM
Also noteworthy is the negative correlation of degree and message size (number of characters, tokens or sentences) that intermediaries presented.
This and other insights can be drawn from Tables~\ref{tab:corTop},~\ref{tab:corTex} and~\ref{tab:corTexTop}.
Overall, negligible correlation is found between textual and topological metrics.
% verificar e fazer tabelas corretamente:
% focadas para texto, extensivas para SI

\subsection{Formation of principal components}\label{subsec:pc}
\begin{table}[h!]
\begin{center}
\caption{PCA formation}
\begin{tabular}{| l || c | c | c | c | c |}\hline
 & {\bf PC1} & {\bf PC2} & {\bf PC3} & {\bf PC4} & {\bf PC5} \\\hline\hline
{\bf $cc$} & -3.84 & -5.10 & -31.80 & 9.76 & -22.06 \\
{\bf (p.)} & -9.03 & -10.13 & -30.55 & -4.77 & -11.17 \\
{\bf (i.)} & 0.26 & -15.63 & 37.08 & -3.17 & 6.95 \\
{\bf (h.)} & -7.00 & 22.74 & 28.27 & -5.22 & -5.22 \\\hline
{\bf $d$} & -10.20 & -24.84 & 3.64 & -0.15 & 1.34 \\
{\bf } & -11.87 & -14.57 & -6.71 & 14.22 & -5.89 \\
{\bf } & 11.61 & 12.06 & 5.21 & 27.91 & 20.47 \\
{\bf } & 7.44 & -25.19 & 0.51 & 7.60 & 7.60 \\\hline
{\bf $s$} & -7.41 & -27.67 & 4.95 & -7.51 & -3.76 \\
{\bf } & -9.18 & -12.11 & 17.12 & 24.67 & 1.31 \\
 & 8.41  & 13.63  & 29.19  & -13.73  & 0.58 \\
 & 13.89  & -3.62  & -5.85  & -1.29  & -1.29 \\\hline
$\mu_S(p)$ & -11.78  & 13.03  & 12.10  & -15.34  & -16.62 \\
 & -8.50  & 18.81  & 2.44  & 8.13  & -16.08 \\
 & 14.32  & -4.67  & -14.13  & -11.00  & 25.09 \\
 & -13.53  & -6.65  & -10.24  & -3.61  & -3.61 \\\hline
$\sigma_S(p)$ & -14.56  & 1.35  & 1.43  & -14.80  & -3.29 \\
 & -11.61  & 15.37  & 9.50  & -3.23  & -14.81 \\
 & 14.03  & -7.98  & -10.43  & -16.02  & -1.90 \\
 & -3.29  & -24.50  & 35.74  & -7.33  & -7.33 \\\hline
$\mu_S(kw)$ & -14.99  & 7.93  & -9.72  & -1.21  & 7.61 \\
 & -16.21  & 4.91  & -1.11  & -3.44  & 13.21 \\
 & 12.73  & -13.04  & 0.36  & -1.11  & -0.99 \\
 & -12.96  & -9.16  & -16.09  & -25.59  & -25.59 \\\hline
$\sigma_S(kw)$ & -12.65  & 7.94  & -16.56  & -7.93  & 18.37 \\
 & -15.68  & -0.14  & 7.38  & -10.56  & 17.41 \\
 & 10.06  & -15.45  & 2.09  & 23.68  & -14.15 \\
 & -13.92  & -4.00  & -0.65  & 6.10  & 6.10 \\\hline
$\mu_S(sw)$ & -11.81  & 9.78  & 13.26  & 19.14  & -14.16 \\
 & -3.80  & 16.64  & -21.56  & 15.34  & 14.73 \\
 & 14.78  & 7.50  & -0.47  & -2.87  & -9.72 \\
 & -13.92  & -3.93  & -1.67  & 27.54  & 27.54 \\\hline
$\sigma_S(sw)$ & -12.76  & -2.35  & 6.54  & 24.17  & 12.80 \\
 & -14.11  & -7.33  & 3.64  & -15.64  & -5.38 \\
 & 13.80  & 10.05  & 1.04  & -0.51  & -20.16 \\
 & -14.05  & 0.21  & -0.98  & 0.83  & 0.83 \\\hline\hline
$\lambda$ & 49.30  & 19.02  & 14.48  & 8.44  & 4.75 \\
 & 46.67  & 28.95  & 10.95  & 7.95  & 3.63 \\
 & 57.01  & 28.08  & 10.20  & 3.34  & 1.37 \\
 & 70.05  & 24.25  & 5.70  & 0.00  & 0.00 \\\hline
\end{tabular}
\begin{flushleft}\footnotesize
		Source: By the author.\
\end{flushleft}
\end{center}
\end{table}

\begin{figure}[!h]
	\centering
	\includegraphics[width=0.3\textwidth]{figs/plot_pca}
	\caption{First two principal components.}
	\label{fig:formation}
\end{figure}



Principal components formation seem to be the less stable of all results reported in this study.
First component, with $\approx 25\%$ of dispersion, relies heavily on POS tags,
and slightly on sizes of tokens, sentences and messages.
Second component, with almost $12\%$ of dispersion, blends topology, POS tags and size of textual units.
Third component, with about $8.5\%$ of dispersion is mostly nouns frequency and size of textual units.
Fourth and fifth components present less than $5\%$ of total dispersion,
but are included in the Supporting Information document for completeness of exposition.

Tables~\ref{tab:pca1}-\ref{tab:pca5} yield these results and further insights.

\subsection{Results still to be interpreted}\label{subsec:sii}
%These networks yield diverse characteristics,
%some of which were not of core importance for this step of the research.
%Even so, at least one of these characteristics was found interesting enough to be considered a result and an example of interesting artifacts found.
Histogram differences of incident word sizes with and without repetition of words are constant.
That is, in each email list, when a histogram of word sizes were made with all words written,
and another histogram made with sizes of all \emph{different} words,
the cumulative absolute difference of the two histograms throughout the bins were found constant for all lists analysed.
When all known English words were considered, 
the difference sums up to $\approx 1.0$.
When stopwords are discarded,
the difference found was different, but still constant, slightly above $0.5$.
When only stopwords were considered, the difference is $\approx 0.6$.
When only known English words that does not have wordnet synsets are used,
this difference is $\approx 1.2$.
Appendix ~\ref{sec:resE} and Figures~\ref{fig:kw}-\ref{fig:nssnsw} are dedicated to this histogram differences.

\section{Results from visualization}
Results from versinus are divided in two groups:
observations on features that made it useful for the task,
and the network properties it made possible to grasp.

\subsection{Useful visualization features for dynamic networks}

Among the numerous insights related to versinus, a few of them seem more fundamental, or plain useful.
Such insights were incorporated to Versinus as the result of tests which presented clear benefits within the context of our research.
The folowing list is an attempt to present them in an importance-first order:
\begin{enumerate}
	\item Vertices need to remain static.
		Even if they move smoothly, one notices solely transient artifacts.
	\item Very connected sectors (hubs and intermediary) need to be in a curve, otherwise the edges enclose each other and reasoning about the network becomes harsh.
	\item Height and width of a vertex are very informative, specially if measures mapped to them have a strong relation, such as out-degree (mapped to height in versinus) and in-degree (mapped to width).
	\item The color of nodes is also informative although less than height and weight, as differences is the latter are more noticeable.
	\item An ordering of nodes, related to their fixed position, is very useful. Among all tests, ordering of vertices by degree was considered the most informative, which led to the hub, intermediary and peripheral sectioning of the network delineated in Section~\ref{sectioning}.
		As node position in the layout is fixed throughout an animation which comprises consecutive but distinct network activity, such ordering is done with respect to the resulting network of all the activity.
	Numbering these positions with respect to the order of the vertices in the larger structure (e.g. all $M$ messages) is useful for understanding how much a vertex preserves the position in different scales of activity.
\end{enumerate}

Many other insights were given by Versinus, such as possible visualization tools, other kind of convenient layouts and glyph elaborations.
These receives dedicated attention in Section~\ref{sec:ref}.

\subsection{Understanding of network properties through Versinus}
A number of hypotheses were drawn about the networks for which Versinus was designed.
Another number of hypothesis were driven from versinus use itself.

As suggested by Palla, Barab\'asi and Vicsek~\cite{barabasiEvo}, stability of participant activity in social networks is more incident in smaller networks.
In accordance with this result, all hubs have intermittent activity in the settings analyzed, except for the email list with the smallest number of participants (the Metareciclagem email list).
The intermitence of hubs itself was one of the top hypotheses which motivated Versinus development.
The stability of the network structure, concomitant with the instability of the activity of each participant,
motivated a deeper analysis~\cite{evoSN}.
In doing so, we also found evidence for another hypothesis drawn from Versinus:
that in- and out-degree differences in each vertex are important for network characterization.
Furthermore, the visualization suggests that there are modes of operation of the network.
As an example, the intermediary sector often communicates mostly with the hubs or with the peripheral vertex.
Other hypotheses,
such as discrepancies in the authority and the degree of a vertex,
are numerous but need further research to be valuable.
 
\subsection{Refinement of Versinus}\label{sec:verref}
Versinus was convenient for obtaining insights about how to enhance its layout and use.
It was immediate to think of a tool for using Versinus
in real-time, but less obvious are some ideas about the layout and visual guides. 
To further enable visualization of hubs and intermediary vertex,
the sinusoid can have many periods 
with a decaying frequency.
The upper straight line can also have an oscillating outline.
The two halves of the sinusoidal period could be moved independently.
The waveform need not to be a sinusoid.
One can think of many ways to make more informative glyphs.
Also, visual and auditory signals for specific occurrences can be interesting
(e.g. when a new vertex appears, when one vanishes, when an ordering of vertices changes).
Measures of each vertex can be exposed with a vertical displacement,
to enable multiple measures, to avoid the need to blink the numbers and to keep network visualization free from occlusion.
Working with Versinus has also suggested other kinds of layout for vertices, 
specially geometric figures and iterative force-based methods for positioning vertex in a fixed layout.
The traditional matrix representation of the graphs has been gazed upon as support to Versinus 
as has been some recent approaches to network visualization~\cite{Viz1}.


\section{Linked data results}
\label{outline}
Current results include data selection and preparation for knowledge discovery.
In this respect, the main result is the data made available, which enables benchmarking of scientific results
and easy experimentations.
Secondary results include data outline through figures and tables,
software support and example SparQL queries.

\subsection{Standardization}
The data is embedded into standard URIs and triples, i.e. translated to RDF.
URIs are built in the namespace \url{http://purl.org/socialparticipation/participationontology/}
which are identified herein with the prefix \textttt{po:}.
Classes and properties are built by adding a suffix to the root,
as in the class \textttt{po:Participant} or in the property \textttt{po:text}.
Classes have ``UpperCamelCase'' suffixes while properties have ``lowerCamelCase'' suffixes.
All class instances, such as participants, messages, friendships and
interactions, are linked to
snapshots through the triple \textttt{<instance> po:snapshot <snapshot\_uri>}.
Message texts, including comments, are objects in the triple: \textttt{<message\_id> po:text <message\_text>}.
Preprocessed texts are objects of triples: \textttt{<message\_id> po:cleanText <message\_text>}.
More specialized predicates are used for delivering text when necessary,
such as \textttt{po:htmlBodyText} and \textttt{po:cleanBodyText} used
for ParticipaBR articles (instances of the class \textttt{po:Article}.
A participant URI is unique throughout the provenance (e.g. the same for
the same participant in all Twitter snapshots).
To enable annotations which differ when the snapshot changes,
\textttt{po:Observation} class instances are used in the triple
\textttt{<participant\_uri> po:observation <observation\_uri>}.
The observation instances are then linked to the snapshot and the
data.

Instances are built on top of the class they derive from plus a hashtag character,
a provenance string (e.g. \textttt{facebook-legacy} or
\textttt{participabr-legacy}) of the snapshot they refer to, and an identifier;
i.e. \textttt{po:Participant\#<provenance-legacy>-<id>}.
All snapshot URIs follow the formation rule: \textttt{po:<SnapshotProvenance>\#<snapshot\_id>}.
All snapshot ids follow the formation rule: \textttt{<platform>-legacy-<further\_identifier>}; e.g.
\textttt{irc-legacy-labmacambira} or
\textttt{email-legacy-linux.audio.devel1-20000}.

\subsection{Data outline}
The database consists of 34,120,026 triples, 3,172,927 edges yield by interactions or relations, 382,568 participants and 253,155,020 characters. Among all snapshots, 63 are ego snapshots, 54 are group snapshots; 49 have interaction edges, 89 have friendship edges; 43 have text content from messages.



\begin{table*}[h!]
\begin{center}
\caption{Number of snapshots from each provenance.}
\begin{tabular}{l | c}\hline
\textbf{social protocol} & \textbf{number of snapshots} \\\hline\hline
Algorithmic Autoregulation & 3 \\
Cidade Democrática & 1 \\
Email & 4 \\
Facebook & 88 \\
IRC & 4 \\
ParticipaBR & 1 \\
Twitter & 16 \\\hline
all & 117 \\\hline
\end{tabular}\end{center}
\end{table*}

% stats:
%% number of triples
%% number of edges
%% number of chars
%% number of users
% diagrams in the supporting information file
\subsection{Software tools}
The database is released with software for rendering itself, analyses and
multimedia artifacts.
\subsubsection{Triplification routines}
For each social platform there is a \emph{triplification} routine,
i.e. a script for translating data to RDF.
Original formats and further observations are presented in
Table~\ref{tab:provenance}.
\begin{table*}[h!]\scriptsize
	\begin{center}
		\caption{Social platforms, original formats and further observations for
		the database.}\label{tab:provenance}
		\begin{tabular}{| l || p{3cm} | p{3cm} | c |}\hline
			\textbf{social platform} & \textbf{original format} & \textbf{further observations} & \textbf{toolbox} \\\hline\hline
				AA & MySQL and MongoDB databases; IRC text logs & donated by AA users & Participation~\cite{participation} \\\hline
				    Cidade Democrática & MySQL database & donated by admins & Participation \\\hline
					Email & mbox & obtained through Gmane public database & Gmane~\cite{gmane} \\\hline
					    Facebook & GDF, GML and TAB & obtained through Netvizz~\cite{netvizz} & Social~\cite{social} \\\hline
						IRC & plain text log & obtained through Supybot logging & Social \\\hline
						    ParticipaBR & PostgreSQL database & donated by admins & Participation \\\hline
							Twitter & JSON & obtained through Twitter streaming API & Social \\\hline
		\end{tabular}\end{center}
	\end{table*}                    
\subsubsection{Topological and textual analysis}\label{ana}
Routines are available for taking topological and textual measures from
the database.
Auxiliary routines, such as performing principal component analysis
and taking Kolmogorov-Smirnov measures, are available
to ease pattern recognition.
Single, timeline and multi-scale analyzes are automated.

\subsubsection{Multimedia rendering}\label{media}
It is a core purpose of the framework to provide routines for rendering
audiovisualizations of the data.
Social structures are rendered into music, images and video animations
through the Percolation toolbox~\cite{percolation} in association with
the Music and Visuals toolboxes~\cite{music,visuals}.

\subsubsection{Migration from deprecated toolboxes}
Routines mentioned in Sections~\ref{ana} and~\ref{media} are being migrated from deprecated
toolboxes~\cite{gmaneLegacy,percolationLegacy} into newly designed
toolboxes~\cite{percolation,visuals}.

\subsection{Diagrams of the data and auxiliary tables}
The database exploration can be assisted through diagrams which expose
the structure from each provenance.
Such diagrams are exemplified in the Appendix~\ref{losd} and 
fully available in a dedicated article~\cite{losd}
with some tables to ease understanding of the provided data.
A simplified example is given in Figure~\ref{dia} where the friendship
structure of the Facebook snapshots are exposed.

	\begin{figure}[!ht]
		\centering
			\includegraphics[width=0.5\textwidth]{ontologies/facebook-legacy-AntonioAnzoategui18022013Friendship.ttl/draw}
			    \caption{A diagram of the structure involved in the friendship networks
				of the Facebook snapshots.
				A green edge denotes an OWL existential class restriction;
				an inverted nip denotes an OWL universal class restriction;
				a full (non-dashed) edge denotes an OWL functional property axiom.
				Further information and complete diagrams for each provenance are in the dedicated article~\cite{losd}.}\label{dia}
	\end{figure}


	\subsection{SPARQL queries}\label{queries}
	There are numerous useful and general purpose SPARQL queries to be performed against the database.
	Here we write some of such queries selected by their simplicity and potential to be varied.
	All queries assume the use the preamble \textttt{PREFIX po: <http://purl.org/socialparticipation/po/>}.
	\begin{enumerate}[leftmargin=0cm]
		\item Retrieve the number of participants:\\
			\textttt{SELECT (COUNT(DISTINCT ?author) as ?c) WHERE \{
				?author a po:Participant . \} }
			\item Retrieve the number of relations, be them interactions or
				friendships:\\
														\textttt{SELECT (COUNT(?interaction) as ?c) WHERE \{\\
																    \h \{ ?interaction a po:Friendship \} UNION \{ ?interaction
																			a po:Interaction \} UNION\\
																					    \h \{ ?interaction po:retweetOf
																								?message \} UNION \{ ?interaction po:replyTo ?message
																										    \}\\
																													\h UNION \{ ?interaction po:directedTo ?participant
																															    \}\\ \} }
																														    \item Retrieve all text produced by an specific user:\\
																															    \textttt{SELECT (CONCAT(?text) as ?texts) WHERE \{\\
																																					\h ?activity po:author <user\_uri> . ?activity po:text ?text .\\
																																					    \}}
																																				    \item List 1000 users (URIs and names) with the most friendships and the number of
																																					    friendships in descending order by the number of friendships:\\
																																										    \textttt{SELECT DISTINCT ?participant (COUNT(?friendship)
																																												as ?c) WHERE \{\\
																																														\h ?friendship a po:Friendship . ?friendship po:member ?participant . \\
																																															    \} ORDER BY DESC(?c) LIMIT 1000}
																																														    \item Retrieve text messages with the word ``pineapple'' (case insensitive):\\
																																															    \textttt{SELECT ?text WHERE \{ \\
																																																					    \h ?activity po:text ?text . FILTER regex(?text, 'pineapple', 'i')\\
																																																						\}}
																																																					\item List participants and respective full names whose name has the substring ``Amanda'':\\
																																																						\textttt{SELECT DISTINCT ?participant ?name WHERE \{\\
																																																							\h ?participant po:observation ?obs . ?obs po:name ?name .\\
																																																							    \h FILTER regex(?name, 'Amanda', 'i') \\
																																																								\}}
																																																							\item Return all pairs of friends of a participant which are friends themselves:\\
																																																								\textttt{SELECT DISTINCT ?friend1 ?friend2 WHERE \{\\
																																																									 \h       ?friendship1 po:member <participant\_uri> .  ?friendship1 po:member ?friend1 .\\
																																																									      \h       ?friendship2 po:member <participant\_uri> .  ?friendship2 po:member ?friend2 .\\
																																																										   \h       ?friendship3 po:member ?friend1 .  ?friendship3 po:member ?friend2 .\\
																																																										       \}}
																																																									       \item Return all interactions from replies in a snapshot:\\
																																																										       \textttt{SELECT ?from ?to WHERE \{\\
																																																												     \h  ?message1 po:snapshot <snapshot\_uri> .  ?message2 po:replyTo ?message1 .\\
																																																													       \h  ?message1 po:author ?from .  ?message2 po:author ?to .\\
																																																														   \}}
	\end{enumerate}

	\subsection{License issues}
	% Facebook, Twitter, IRC, Gmane, Participa, CD, AA.
	The database presented in this article is released under public domain.
	Computer scripts are in git repositories and PyPI Python packages, also under public domain.
	Although most data is already in open licenses (Twitter, Email, Participabr, Cidade Democrática, and AA data), IRC and Facebook data was collected
	and donated by the individuals which yield the data.
	This rises the the understanding of the right to study such data as the right to access the self,
	in parity with anthropological endaviors~\cite{antphy,antphy2}.


\subsection{Data-driven ontology synthesis}
OWL Ontologies are critical tools to describe taxonomies and the
structure of knowledge.
Most ontologies are created by domain experts even though there often is data they
organize that is given by a software system and which has a predefined
structure.  

We developed a simple ontology synthesis method that probes
the ontological structure in data with
SPARQL queries and post-processing.
The results are OWL code and diagrams which are 
exemplified in the Appendix~\ref{losd} and 
fully available in a dedicated article~\cite{losd}.
The method can be extended to comprise further OWL axioms and restrictions,
but is currently performed to fit present needs with maximum simplicity.
Present needs are limited to informative figures and
the steps implemented are as follows:
\begin{enumerate}[leftmargin=0cm]
	\item Obtain all distinct classes with the query:\\
		\textttt{SELECT DISTINCT ?class\_uri WHERE \{ ?s a ?class\_uri \}}
	\item For each class, obtain the properties that occur as predicates in triples where the subject is an instance of the class:\\
		\textttt{SELECT DISTINCT ?property\_uri WHERE \{ ?s a <class\_uri> . ?s ?property\_uri ?o . \}}\\
					Such properties are used to assert existential and universal restrictions for the class.
				\item Compare the total number of individuals (\textttt{?cs1}) of the class (\textttt{class\_uri}) with
					the number of such individuals (\textttt{?cs2}) that are subjects of at least one triple where 
							the predicate is the property (\textttt{property\_uri}).
								If the numbers match, there is an existential restriction for the class. The queries are:\\
									\textttt{SELECT (COUNT(DISTINCT ?s) as ?cs1) WHERE \{ ?s a <class\_uri> \}}\\
										\textttt{SELECT (COUNT(DISTINCT ?s) as ?cs) WHERE \{\\
											\h ?s a <class\_uri>. ?s <property\_uri> ?o .\\ \}}
										\item Find the number of instances which are subjects of triples where the predicate is the property but are not instances of the class.
											If there is zero of such instances, there is an universal restriction:\\
													\textttt{SELECT (COUNT(DISTINCT ?s)=0 as ?cs) WHERE \{\\
														\h ?s <property\_uri> ?o . ?s a ?ca . FILTER(str(?ca) != 'class\_uri')\\ \}}
													\item To keep a record of the restrictions (and occurring triples), get all object classes or datatypes where the subject is an instance of the class and the predicate is the property:\\
														\textttt{SELECT DISTINCT ?co (datatype(?o) as ?do) WHERE \{\\
																	\h ?s a <class\_uri>. ?s <property\_uri> ?o . OPTIONAL \{ ?o a ?co . \}\\
																	\}}
																\item Obtain all distinct properties:\\
																	\textttt{SELECT DISTINCT ?p WHERE \{ ?s ?p ?o \}}
																\item Check if each property is functional, i.e. if it
																	occurs at most once with each subject.
																							This is performed by counting the objects and further verifying
																								that they are at most one. The query is:\\
																									\textttt{SELECT DISTINCT (COUNT(?o) as ?co) WHERE \{ ?s
																										    <property\_uri> ?o \} GROUP BY ?s}
																									    \item For each property, find the incident range and domain with the
																										    queries:\\
																														\textttt{SELECT DISTINCT ?co (datatype(?o) as ?do) WHERE \{\\
																																	\h ?s <property\_uri> ?o . OPTIONAL \{ ?o a ?co . \}\\\}} \\
																																		and \\
																																			\textttt{SELECT DISTINCT ?cs WHERE \{ ?s <property\_uri> ?o . ?s a ?cs . \}}
																																		\item Render diagrams as exposed in the next section and in the Supporting Information file.
\end{enumerate}


