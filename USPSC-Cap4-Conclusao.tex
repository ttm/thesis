%% USPSC-Cap3-Conclusao.tex
% Capítulo 3 - Conclusão
% ---
% Conclusão
% ---
\chapter{Conclusion and Future Work}
% ---
% O comando abaixo insere parágrafos aleatórios só para exemplificar
The very small standard deviations of principal components formation
(see Sections~\ref{sec:pca} and~\ref{prevalence}),
the presence of the Erd\"os sectors even in networks with
few participants (see Sections~\ref{sectioning} and~\ref{subsec:pih}),
and the recurrent activity patterns along different timescales (see Sections~\ref{sec:mtime} and~\ref{constDisc}),
go a step further in characterizing scale-free networks in the context
of the interaction of human individuals.
Furthermore, the importance of symmetry-related metrics,
which surpassed that of clustering coefficient,
with respect to dispersion of the system in the topological measures space,
might add to the current understanding of key-differences between digraphs and
undirected graphs in complex networks.
Noteworthy is also the very stable fraction participants in each Erd\"os sector when the network reaches more than 200 participants.
Benchmarks were derived from email list networks
and the supplied analysis of
networks from Facebook,
Twitter and Participabr in the Supporting Information might ease hypothesizing
about the generality of these characteristics.

Further work should expand the analysis to include
more types of networks and more metrics.
The data and software needed to attain these results
should also receive dedicated and in-depth
documentation as they enable a greater level of transparency
and work share,
which is adequate for both benchmarking
and specifically for the study of systems constituted
by human individuals (see Section~\ref{sec:data}).
The derived typology of hub, intermediary and peripheral participants
has been applied for semantic web and participatory democracy efforts,
and these developments might be enhanced to yield scientific knowledge~\cite{opa}.
Also, we plan to further explore and publish the audiovisualizations
used for this research~\cite{versinus,animacoes} and
the linguistic differences found in each of the Erd\"os sectors~\cite{rcText}.

 
