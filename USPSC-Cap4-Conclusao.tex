%% USPSC-Cap3-Conclusao.tex
% Capítulo 3 - Conclusão
% ---
% Conclusão
% ---
\chapter{Conclusion and Future Work}
\label{ch:concl}
% ---
% O comando abaixo insere parágrafos aleatórios só para exemplificar
The very small standard deviations of principal components formation
(see Sections~\ref{sec:pca} and~\ref{prevalence}),
the presence of the Erd\"os sectors even in networks with
few participants (see Sections~\ref{sectioning} and~\ref{subsec:pih}),
and the recurrent activity patterns along different timescales (see Sections~\ref{sec:mtime} and~\ref{constDisc}),
go a step further in characterizing scale-free networks in the context
of the interaction of human individuals.
Furthermore, the importance of symmetry-related metrics,
which surpassed that of clustering coefficient,
with respect to dispersion of the system in the topological measures space,
might add to the current understanding of key-differences between digraphs and
undirected graphs in complex networks.
Noteworthy is also the very stable fraction participants in each Erd\"os sector when the network reaches more than 200 participants.
Benchmarks were derived from email list networks
and the supplied analysis of
networks from Facebook,
Twitter and Participabr in the Supporting Information might ease hypothesizing
about the generality of these characteristics.

Further work should expand the analysis to include
more types of networks and more metrics.
The data and software needed to attain these results
received dedicated and in-depth
documentation as they enable a greater level of transparency
and work share,
which is adequate for both benchmarking
and specifically for the study of systems constituted
by human individuals (see Section~\ref{sec:data}).
The derived typology of hub, intermediary and peripheral participants
has been applied for semantic web and participatory democracy efforts,
and these developments might be enhanced to yield scientific knowledge~\cite{opa}.
Also, we plan to further explore and publish the audiovisualizations
used for this research~\cite{versinus,animacoes} and
the linguistic differences found in each of the Erd\"os sectors~\cite{rcText}.

\section{Text final remarks}\label{sec:remarks}
%Human interaction networks yield diverse linguistic peculiarities reported by its members.
This is a first systematic exploration of the relation between topological and textual
metrics in human interaction networks, as far the author knows.
Different textual features were scrutinized and were found to present
evident patterns, specially in relation to topological measures and the Erd\"os sectors.
Furthermore, results suggest that less connected participants bring external content and concepts,
while hubs qualify the content.
For example, periphery sectors present more nouns while hubs use more adjectives and usual words.
Such findings have potential applications in the collection and diffusion and information,
resources recommendation in linked data contexts,
and open processes of document elaboration and refinement~\cite{ensaio,OPS,pnud5,evoSN,pbr}.

\section{Linked data final remarks}
The database presented in this article
constitutes a large database with diverse provenance.
Even so, the database should be expanded in upon need or requests from feedback.
All data should be available online in the \url{http://linkedopensocialdata.org}
address in near future to fulfill the purpose of being a common
repertoire in current research.
One should reach the diagrams and tables of the 
Appendices and of the articles produced in this research~\cite{stab,text,vers,losd}
for further directions
on the available structures and for an overview complement.


\subsection{Further work}\label{subsec:fw}

Similarity measures of texts in message-response threads has been thought about by the author,
and some results should be organized in near future.
These are two hypothesis obtained from recent experiments:
\begin{itemize}
    \item existence of information ``ducts'', observable through similarity measures.
	    These might coincide with asymmetries of edges between vertice pairs,
		with homophily or with message-response threads, to point just a few possibilities.
    \item Valuable insights might be obtained from the self-similarity of messages by same author,
	    of messages sent at the same period of the day, etc.
	    This includes incidences of word sizes, incidences of tags and morphosintactic classes,
	    incidences of particular wordnet synset characteristics and distances.
\end{itemize}

Current results suggest that diversity and self-similarity should vary with respect to connectivity. 
Literature usually assumes that periphery holds greater diversity~\cite{easley},
which can be further verified, for example through the diversity of entries (e.g. tokens, sentence sizes).

Other potential next steps are:
\begin{itemize}
    \item The observation of most incident words and word types,
	    such as words related to cursing, food or body parts.
    \item Interpretation of the constant difference found from incident and existent tokens histograms,
	    exposed in Section~\ref{subsec:sii}.
    \item Extend word class observations, e.g. to include plurals, gender, common prefixes and suffixes.
    \item The observation of date and time in relation to textual production of interaction networks and
	    to activity characteristics (e.g. dispersion of sent time along the day or weekdays).
    \item A careful analysis of each textual features distribution which is likely to reveal multimodal outlines and other non-trivial characteristics.
    \item Extend analysis to the windowed approach along the timelines used in this work, where hub, peripheral and intermediary sectors where topologically characterized~\cite{evoSN}.
    \item For ELE list, the more connected the sector, the longer the messages are.
	    This is the inverse of what was found in the other lists,
	    and was considered a peculiarity of the culture bonded with the political subject of ELE list.
	    This hypothesis should be further verified.
    \item Tackle the same analysis on networks with languages other than English.
	    This is especially important for easing applications~\cite{ensaio}
	    and should rely on dedicated implementation of 
	    tokenization, lemmatization and attribution of POS tags.
    \item Observe a broader set of human interaction networks and the resulting types
	    of networks and participants with respect to topological and textual features.
    \item Analyse interaction networks from other platforms such as LinkedIn, etc.
    \item Sentiment analysis was not approached in this work, but might be a good endavior since the subject has received considerable attention from the scientific literature but has not included topological features.
\end{itemize}
 
