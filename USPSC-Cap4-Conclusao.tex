\chapter{Conclusion and Future Work}
\label{ch:concl}
The work reported in this thesis was conceived to bring contributions in three aspects. First, in order to mine data from social networks (e-mail lists and others), a number of tools and procedures to deal with large amounts of data were developed. More specific details will be mentioned in connection with linked data and visualization tools. The second contribution is associated with a thorough analysis of the dynamics of social networks represented by email lists. The third contribution arose from further exploring the findings in the second contribution. Having found that a partition of networks into hubs, intermediary and peripheral participants was very stable over time, we then discovered that the pattern of language usage could be typical of the classes of participants. In the following, the overall conclusions from all of these contributions are described in further detail. 

\section{Stability of topology in the networks}

The very small standard deviations of principal components formation
(see Sections~\ref{sec:pca} and~\ref{prevalence}),
the presence of the Erd\"os sectors even in networks with
few participants (see Sections~\ref{sectioning} and~\ref{subsec:pih}),
and the recurrent activity patterns along different timescales (see Sections~\ref{sec:mtime} and~\ref{constDisc}),
go a step further in characterizing scale-free networks in the context
of the interaction of human individuals.
Furthermore, the importance of symmetry-related metrics,
which surpassed that of clustering coefficient,
with respect to dispersion of the system in the topological measures space,
might add to the current understanding of key-differences between digraphs and
undirected graphs in complex networks.
Noteworthy is also the very stable fraction of participants in each Erd\"os sector when the network reaches more than 200 participants.
Benchmarks were derived from email list networks
and the supplied analysis of
networks from Facebook,
Twitter and ParticipaBR in the Appendix might allow for hypothesizing
about the generality of these characteristics.

\section{Textual analysis final remarks}\label{sec:remarks}
%Human interaction networks yield diverse linguistic peculiarities reported by its members.
This is a first systematic exploration of the relation between topological and textual
metrics in human interaction networks, as far the author knows.
Different textual features were scrutinized and were found to present
salient patterns, especially in relation to topological measures and the Erd\"os sectors.
Furthermore, results show that peripheral participants use more nouns while hubs use more verbs,
which suggests that less connected participants bring content and concepts,
while hubs propose action on them.
Such findings have potential applications in the collection and diffusion of information,
resources recommendation in linked data contexts,
and open processes of document elaboration and refinement~\cite{ensaio,ops,opa,stab,pnud4,pnud3}.

Most importantly, it is reasonable to conclude, from all the distinct textual characteristics
between the Erd\"os sectors, that, as a rule of thumb, the texts from each of the sectors differ.
Surely there should be exceptions and it is a fact that we left out of the analysis
more subtle textual aspects e.g. those related to low percentages (<5\%) or to small
differences.
These might be the subject of future studies.

\section{Linked data final remarks}
The database presented in this thesis
constitutes a large database with diverse provenance.
Even so, the database should be expanded upon need or requests from feedback.
All data should be available online in the \url{http://linkedopensocialdata.org}
address in the near future to fulfill the purpose of being a common
repertoire in current research.
One should reach the diagrams and tables of the 
articles produced in this research~\cite{stab,rcText,versinus,losd}
for further directions
on the available structures and for an overview complement.


\section{Future work}\label{sec:fw}

Further work should expand the analysis to include
more types of networks and more metrics.
The data and software needed to attain these results
received dedicated and in-depth
documentation as they enable a greater level of transparency
and work share,
which is adequate for both benchmarking
and specifically for the study of systems constituted
by human individuals (see Section~\ref{sec:data}).
The derived typology of hub, intermediary and peripheral participants
has been applied for semantic web and participatory democracy efforts,
and these developments might be enhanced to yield scientific knowledge~\cite{pnud4,opa,losd}.
Also, we plan to further explore and publish the audiovisualizations
used for this research~\cite{versinus,animacoes} and
the linguistic differences found in each of the Erd\"os sectors~\cite{rcText}.

Similarity measures of texts in message-response threads have been conceived, 
and some results should be organized in the near future.
In this respect, there are two core hypotheses obtained from recent experiments:
\begin{itemize}
\item the existence of information ``ducts'', observable through similarity measures.
These might coincide with asymmetries of edges between vertex pairs,
with homophily or with message-response threads, to point just a few possibilities.
\item Valuable insights might be obtained from the self-similarity of messages by the same author,
of messages sent at the same period of the day, etc.
This includes incidences of word sizes, incidences of tags and morphosyntactic classes,
incidences of particular Wordnet synset characteristics and distances.
\end{itemize}

The current results suggest that diversity and self-similarity should vary with respect to connectivity. In the literature, it is usually assumed 
that the periphery holds greater diversity~\cite{easly},
which can be further verified, for example through the diversity of entries (e.g. tokens, sentence sizes).

Other potential next steps are:
\begin{itemize}
\item The observation of most incident words and word types,
such as words related to cursing, food or body parts.
\item Interpretation of the constant difference found from incident and existent tokens histograms,
mentioned in Section~\ref{subsec:sii}.
\item Extend word class observations, e.g. to include plurals, gender, common prefixes and suffixes.
\item The observation of date and time in relation to textual production of interaction networks and
to activity characteristics (e.g. dispersion of sent time along the day or weekdays).
\item A careful analysis of each textual feature distribution which is likely to reveal multimodal outlines and other non-trivial characteristics.
\item Extend analysis of textual measurements to the windowed approach along the timelines, where hub, peripheral and intermediary sectors were topologically characterized~\cite{stab}.
\item For ELE list, the more connected the sector, the longer the messages are.
This is the inverse of what was found in the other lists,
and was considered a peculiarity of the culture bonded with the political subject of ELE list.
This hypothesis should be further verified.
\item Tackle the same analysis on networks with languages other than English.
This is especially important for applications~\cite{ensaio}
and should rely on dedicated implementation of 
tokenization, lemmatization and attribution of POS tags.
\item Observe a broader set of human interaction networks and the resulting types
of networks and participants with respect to topological and textual features.
\item Analyze interaction networks from other platforms such as LinkedIn, Diáspora, etc.
\item Sentiment analysis was not used in this work, but might be a good endeavor since the subject has received considerable attention from the scientific literature but has not included topological features as far as we know.
\item This thesis focused on differences of texts among Erd\"os sectors but we envisage that comparison of texts from social networks with canonical texts (e.g. Shakespeare or the King James Bible) might yield other powerful insights.
\item The significant differences found from the texts of the sectors raises the question as to why we are not conscious of these differences.
One possibility is that they are part of our instinctive and unconscious social coordination and this hypothesis might be the goal of future endeavors.
\end{itemize}

% Multiscale analysis


% Versinus


% Collection and diffusion of information

% Inclusion of more metrics

